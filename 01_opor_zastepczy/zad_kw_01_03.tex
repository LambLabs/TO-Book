\begin{task}
Wyznacz impedancje zastępczą $Z_{AB}$ poniższego układu pomiędzy zaciskami A i B. Podaj wzór na impedancje zastępczą oraz jego wartość. Przyjmij że $Z=2+j\,\Omega$
\begin{schemat}
\label{schemat:01:03:kw:Z}
\draw
 node[ocirc,label=A] (A) at (0, 0) {}
 node[ocirc,label=below:B] (B) at (0,-6) {}

 (A) to[short] (1,0)
     to[Z,l_=$Z_4$,*-] (4,0)
     to[Z,l_=$Z_5$] (4,-3)
     to[Z,l_=$Z_6$] (4,-6)
     to[short] (B)
 (1,0) to[Z,l_=$Z_1$,*-*] (1,-3)
       to[Z,l_=$Z_2$,*-*] (1,-6)
 (1,-3) to[Z,l_=$Z_3$,*-*] (4,-3)     
;
\end{schemat}

\subsubsection{Rozwiązanie}
Analizę układu należy rozpocząć od obserwacji iż żadne dwie impedancje nie są połączone szeregowo ani równolegle. Można natomiast zauważyć iż zaznaczone na schemacie \ref{schemat:01:03:kw:A} impedancje $Z_1$, $Z_2$ oraz $Z_3$ są połączone w gwiazdę. Można zastąpić je układem trzech impedancji $Z_A$, $Z_B$ i $Z_C$ połączonych w trójkąt.
\begin{schemat}
\label{schemat:01:03:kw:A}
\draw
node[ocirc,label=A] (A) at (0, 0) {}
node[ocirc,label=below:B] (B) at (0,-6) {}

(A) to[short] (1,0)
to[Z,l_=$Z_4$,*-] (4,0)
to[Z,l_=$Z_5$] (4,-3)
to[Z,l_=$Z_6$] (4,-6)
to[short] (B)
(1,0) to[Z,l_=$Z_1$,*-*] (1,-3)
to[Z,l_=$Z_2$,*-*] (1,-6)
(1,-3) to[Z,l_=$Z_3$,*-*] (4,-3)     
;
\draw[color=red] (1.6,-3.0) ellipse (1.8 and 2.7);
\end{schemat}

\begin{align*}
R_A=R_1+R_3+\frac{R_1 \cdot R_3}{R_2}\\
R_B=R_2+R_3+\frac{R_2 \cdot R_3}{R_1}\\
R_C=R_1+R_2+\frac{R_1 \cdot R_2}{R_3}
\end{align*}

\begin{schemat}
\label{schemat:01:03:kw:B}
\draw
node[ocirc,label=A] (A) at (0, 0) {}
node[ocirc,label=below:B] (B) at (0,-6) {}

(A)   to[short] (1,0)
      to[Z,l_=$Z_4$,*-] (4,0)
      to[Z,l_=$Z_5$] (4,-3)
      to[Z,l_=$Z_6$] (4,-6)
      to[short] (B)
(1,0) to[Z,l=$Z_A$,*-*] (4,-3)
      to[Z,l=$Z_B$,*-*] (1,-6)
      to[Z,l_=$Z_C$,*-*] (1, 0)     
;
\end{schemat}
%-----------------------------------------------
Następnie należy zauważyć iż impedancje $Z_{4}$ i $Z_{5}$ są połączone szeregowo i zastąpić je jedną impedancją zastępczą o wartości $Z_{45}$
\begin{schemat}
\label{schemat:01:03:kw:C}
\draw
node[ocirc,label=A] (A) at (0, 0) {}
node[ocirc,label=below:B] (B) at (0,-6) {}

(A)   to[short] (1,0)
to[Z,l_=$Z_4$,*-] (4,0)
to[Z,l_=$Z_5$] (4,-3)
to[Z,l_=$Z_6$] (4,-6)
to[short] (B)
(1,0) to[Z,l=$Z_A$,*-*] (4,-3)
to[Z,l=$Z_B$,*-*] (1,-6)
to[Z,l_=$Z_C$,*-*] (1, 0)     
;
\draw[color=red,rotate around={45:(3.5,-0.7)}] (3.5,-0.7) ellipse (0.8 and 2.0);
\end{schemat}

\begin{align*}
Z_{45}&=Z_{4}+Z_{5}
\end{align*}

\begin{schemat}
\label{schemat:01:03:kw:D}
\draw
node[ocirc,label=A] (A) at (0, 0) {}
node[ocirc,label=below:B] (B) at (0,-6) {}

(A)   to[short] (2.0,0)
      to[Z,l=$Z_{45}$] (4,-2.0)
      to[short] (4,-3)
      to[Z,l_=$Z_6$] (4,-6)
      to[short] (B)
(1,0) to[Z,l_=$Z_A$,*-*] (4,-3)
      to[Z,l=$Z_B$,*-*] (1,-6)
      to[Z,l_=$Z_C$,*-*] (1, 0)     
;
\end{schemat}
%-----------------------------------------------
Następnie można zauważyć iż impedancje $Z_{45}$ oraz $Z_{A}$ są połączone równolegle i zastąpić je jedną impedancją zastępczą o wartości $Z_{45A}$
\begin{schemat}
\label{schemat:01:03:kw:E}
\draw
node[ocirc,label=A] (A) at (0, 0) {}
node[ocirc,label=below:B] (B) at (0,-6) {}

(A)   to[short] (2.0,0)
to[Z,l=$Z_{45}$] (4,-2.0)
to[short] (4,-3)
to[Z,l_=$Z_6$] (4,-6)
to[short] (B)
(1,0) to[Z,l_=$Z_A$,*-*] (4,-3)
to[Z,l=$Z_B$,*-*] (1,-6)
to[Z,l_=$Z_C$,*-*] (1, 0)     
;
\draw[color=red,rotate around={-45:(3,-1.0)}] (3,-1.0) ellipse (1.0 and 1.5);
\end{schemat}

\begin{align*}
\frac{1}{Z_{45A}}&=\frac{1}{Z_{45}}+\frac{1}{Z_{A}}=\\
&=\frac{Z_{A}}{Z_{45} \cdot Z_{A}}+\frac{Z_{45}}{Z_{45} \cdot Z_{A}}=\\
&=\frac{Z_{45} + Z_{A}}{Z_{45} \cdot Z_{A}}\\
Z_{45A}&=\frac{Z_{45} \cdot Z_{A}}{Z_{45} + Z_{A}}
\end{align*}

Wstawiają wcześniej wyznaczone wartości impedancji $Z_{45}$ i $Z_{A}$ otrzymujemy

\begin{align*}
Z_{45A}&=\frac{Z_{45} \cdot Z_{A}}{Z_{45} + Z_{A}}=\\
&=\frac{\left(Z_{4}+Z_{5}\right) \cdot \left( R_1+R_3+\frac{R_1 \cdot R_3}{R_2} \right)}{Z_{4} + Z_{5} + R_1+R_3+\frac{R_1 \cdot R_3}{R_2} }
\end{align*}

\begin{schemat}
\label{schemat:01:03:kw:F}
\draw
node[ocirc,label=A] (A) at (0, 0) {}
node[ocirc,label=below:B] (B) at (0,-6) {}

(A)    to[short] (1.5,0)
       to[Z,l=$Z_{45A}$] (4,-2.5)
       to[short] (4,-3)
       to[Z,l_=$Z_6$] (4,-6)
       to[short] (B)
(4,-3) to[Z,l=$Z_B$,*-*] (1,-6)
       to[Z,l_=$Z_C$,*-*] (1, 0)     
;
\end{schemat}
%-----------------------------------------------
Następnie należy zauważyć iż impedancje $Z_B$ oraz $Z_{6}$ są połączone równolegle. I można je zastąpić jedną impedancją zastępczą o wartości $Z_{B6}$. 
\begin{schemat}
\label{schemat:01:03:kw:G}
\draw
node[ocirc,label=A] (A) at (0, 0) {}
node[ocirc,label=below:B] (B) at (0,-6) {}

(A)    to[short] (1.5,0)
       to[Z,l=$Z_{45A}$] (4,-2.5)
       to[short] (4,-3)
       to[Z,l_=$Z_6$] (4,-6)
       to[short] (B)
(4,-3) to[Z,l=$Z_B$,*-*] (1,-6)
       to[Z,l_=$Z_C$,*-*] (1, 0)     
;
\draw[color=red] (3,-4.5) ellipse (1.5 and 1.0);
\end{schemat}

\begin{align*}
\frac{1}{Z_{B6}}&=\frac{1}{Z_{B}}+\frac{1}{Z_{6}}=\\
&=\frac{Z_{6}+Z_{B}}{Z_{B} \cdot Z_{6}}\\
Z_{B6}&=\frac{Z_{B} \cdot Z_{6}}{Z_{6}+Z_{B}}
\end{align*}

Wstawiają wcześniej wyznaczoną wartości impedancji $Z_{B}$ otrzymujemy

\begin{align*}
Z_{B6}&=\frac{Z_{B} \cdot Z_{6}}{Z_{6}+Z_{B}}=\\
&=\frac{\left( R_2+R_3+\frac{R_2 \cdot R_3}{R_1} \right) \cdot Z_{6}}{Z_{6}+R_2+R_3+\frac{R_2 \cdot R_3}{R_1}}
\end{align*}
%
\begin{schemat}
\label{schemat:01:03:kw:H}
\draw
node[ocirc,label=A] (A) at (0, 0) {}
node[ocirc,label=below:B] (B) at (0,-6) {}

(A)    to[short] (1.5,0)
       to[Z,l=$Z_{45A}$] (4,-2.5)
       to[short] (4,-3.5)
       to[Z,l_=$Z_B6$] (1.5,-6)
       to[short] (B)
(1,-6) to[Z,l_=$Z_C$,*-*] (1, 0)     
;
\end{schemat}
%-----------------------------------------------
Następnie należy zauważyć iż impedancje $Z_{45A}$ oraz $Z_{B6}$ są połączona szeregowo. I można ją zastąpić jedną impedancją zastępczą o wartości $Z_{45AB6}$. 
\begin{schemat}
\label{schemat:01:03:kw:I}
\draw
node[ocirc,label=A] (A) at (0, 0) {}
node[ocirc,label=below:B] (B) at (0,-6) {}

(A)    to[short] (1.5,0)
to[Z,l=$Z_{45A}$] (4,-2.5)
to[short] (4,-3.5)
to[Z,l_=$Z_B6$] (1.5,-6)
to[short] (B)
(1,-6) to[Z,l_=$Z_C$,*-*] (1, 0)     
;
\draw[color=red] (2.8,-3.0) ellipse (1.0 and 3.0);
\end{schemat}

\begin{align*}
Z_{45AB6} = Z_{45A} + Z_{B6}
\end{align*}

Wstawiając obliczone wcześniej wartości impedancji $Z_{45A}$ i $Z_{B6}$

\begin{align*}
Z_{45AB6} &= Z_{45A} + Z_{B6}=\\
&=\frac{\left(Z_{4}+Z_{5}\right) \cdot \left( R_1+R_3+\frac{R_1 \cdot R_3}{R_2} \right)}{Z_{4} + Z_{5} + R_1+R_3+\frac{R_1 \cdot R_3}{R_2} } + \frac{\left( R_2+R_3+\frac{R_2 \cdot R_3}{R_1} \right) \cdot Z_{6}}{Z_{6}+R_2+R_3+\frac{R_2 \cdot R_3}{R_1}}
\end{align*}

\begin{schemat}
\label{schemat:01:03:kw:J}
\draw
node[ocirc,label=A] (A) at (0, 0) {}
node[ocirc,label=below:B] (B) at (0,-6) {}

(A)    to[short] (2.0,0)
to[Z,l=$Z_{45AB6}$] (2.0,-6)
to[short] (B)
(1,-6) to[Z,l=$Z_C$,*-*] (1, 0)     
;
\end{schemat}
%-----------------------------------------------
Ostatecznie można zauważyć iż impedancje $Z_{C}$ oraz $Z_{45AB6}$ są połączona równolegle. I można ją zastąpić jedną impedancją zastępczą o wartości $Z_{C45AB6}$. 
\begin{schemat}
\label{schemat:01:03:kw:K}
\draw
node[ocirc,label=A] (A) at (0, 0) {}
node[ocirc,label=below:B] (B) at (0,-6) {}

(A)    to[short] (2.0,0)
to[Z,l=$Z_{45AB6}$] (2.0,-6)
to[short] (B)
(1,-6) to[Z,l=$Z_C$,*-*] (1, 0)     
;
\draw[color=red] (1.5,-3.0) ellipse (2.0 and 1.0);
\end{schemat}

\begin{align*}
\frac{1}{Z_{C45AB6}} &= \frac{1}{Z_{C}}+\frac{1}{Z_{45AB6}}\\
&=\frac{Z_{C} + Z_{45AB6}}{Z_{C} \cdot Z_{45AB6}}\\
Z_{C45AB6}&=\frac{Z_{C} \cdot Z_{45AB6}}{Z_{C} + Z_{45AB6}}
\end{align*}

Wstawiając obliczone wcześniej wartości impedancji $Z_{C}$ i $Z_{45AB6}$ otrzymujemy

\begin{align*}
Z_{C45AB6}&=\frac{Z_{C} \cdot Z_{45AB6}}{Z_{C} + Z_{45AB6}}=\\
&= dokończyć
\end{align*}

\begin{schemat}
\label{schemat:01:03:kw:L}
\draw
node[ocirc,label=A] (A) at (0, 0) {}
node[ocirc,label=below:B] (B) at (0,-6) {}

(A)    to[short] (1.5,0)
to[Z,l=$Z_{C45AB6}$] (1.5,-6)
to[short] (B)   
;
\end{schemat}
Tak więc impedancja zastępcza obwodu przedstawionego na schemacie \ref{schemat:01:03:kw:Z} równa się $Z_{C45AB6}$
\end{task}