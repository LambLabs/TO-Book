\subsection{Dzielniki napięciowe i prądowe}

\subsubsection{Dzielnik napięciowy}
\begin{schemat}
\draw
 node[ocirc] (A) at ( 0, 0) {}
 node[ocirc] (B) at ( 0,-4) {}
 (A) to[open, v=$U$] (B)
 ( 0, 0) to[short,i=$I$]       ( 2, 0)
 ( 2, 0) to[R,l=$R_1$,v=$U_1$] ( 2,-2)
 ( 2,-2) to[R,l=$R_2$,v=$U_2$] ( 2,-4)
 ( 0,-4) to[short]     ( 2,-4)
;
\end{schemat}

\begin{equation}
U=U_1+U_2
\end{equation}
\begin{equation}
U_1=R_1 \cdot I
\end{equation}
\begin{equation}
U_2=R_2 \cdot I
\end{equation}
\begin{equation}
I=\frac{U}{R_z}=\frac{U}{R_1+R_2}
\end{equation}

\begin{equation}
U_1=\frac{R_1}{R_1+R_2} \cdot U
\end{equation}
\begin{equation}
U_2=\frac{R_2}{R_1+R_2} \cdot U
\end{equation}

\begin{equation}
U_1=\frac{G_2}{G_1+G_2} \cdot U
\end{equation}
\begin{equation}
U_2=\frac{G_1}{G_1+G_2} \cdot U
\end{equation}


\subsubsection{Dzielnik pradowy}
\begin{schemat} 
%przesynąć etykiety węzłów dalej od węzłów
\draw
 node[ocirc] (A) at ( 0, 0) {}
 node[ocirc] (B) at ( 0,-4) {}
 (A) to[open, v=$U$] (B)
 ( 0, 0) to[short,i=$I$]       ( 2, 0)
 ( 2, 0) to[short      ]       ( 2,-1)
 ( 2,-1) to[short      ]       ( 1,-1)
 ( 2,-1) to[short      ]       ( 3,-1) 
 ( 1,-1) to[R,l=$R_1$,i>^=$I_1$] ( 1,-3)
 ( 3,-1) to[R,l=$R_2$,i>^=$I_2$] ( 3,-3)
 ( 2,-3) to[short      ]       ( 1,-3)
 ( 2,-3) to[short      ]       ( 3,-3)
 ( 2,-3) to[short      ]       ( 2,-4)
 ( 0,-4) to[short]     ( 2,-4)
;
\end{schemat}

\begin{equation}
I=I_1+I_2
\end{equation}
\begin{equation}
I=\frac{U}{R_z}=\frac{U}{\frac{R_1 \cdot R_2}{R_1+R_2}}
\end{equation}
\begin{equation}
I_1=\frac{U}{R_1}
\end{equation}
\begin{equation}
I_2=\frac{U}{R_2}
\end{equation}

\begin{equation}
I_1=\frac{R_2}{R_1+R_2} \cdot I
\end{equation}
\begin{equation}
I_2=\frac{R_1}{R_1+R_2} \cdot I
\end{equation}

\begin{equation}
I_1=\frac{G_1}{G_1+G_2} \cdot I
\end{equation}
\begin{equation}
I_2=\frac{G_2}{G_1+G_2} \cdot I
\end{equation}