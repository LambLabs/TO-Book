\begin{task}
Wyznacz impedancje zastępczą $Z_{AB}$ poniższego układu pomiędzy zaciskami A i B. Podaj wzór na impedancje zastępczą oraz jego wartość. Przyjmij że $Z=3+j\,\Omega$
\begin{schemat}
\label{schemat:01:01:kw:Z}
\draw
 node[ocirc,label=A] (A) at (0, 0) {}
 node[ocirc,label=below:B] (B) at (0,-3) {}
 node[circ,label=45:C] (C) at (6, 0) {}
 node[circ,label=below:D] (D) at (6,-3) {}
 
 (A) to[Z,l_=$Z_1$,o-*] (3,0)
     to[Z,l_=$Z_5$,*-*] (C)
     to[Z,l_=$Z_9$,*-*] (9,0)
     to[Z,l_=$Z_{11}$,*-*] (9,-3)     
     to[Z,l_=$Z_{10}$,*-*] (D)  
     to[Z,l_=$Z_8$,*-*] (C)
 (B) to[Z,l_=$Z_2$,o-*] (3,-3)
     to[Z,l=$Z_4$,*-] (3,-1.5)
     to[Z,l=$Z_5$] (3,0)
 (3,-3) to[short,-*] (3.5,-3)
        to[short] (3.5,-2.5)
        to[Z,l_=$Z_6$] (5.5,-2.5)
        to[short] (5.5,-3.5)        
        to[Z,l=$Z_7$] (3.5,-3.5)        
        to[short] (3.5,-3)  
 (5.5,-3) to[short,*-*] (D)
 (D) to[short] (9,0)
     to[short,-*] (10,1)
     to[Z,l=$Z_{12}$] (10,-3)
     to[short] (9,-3)
 (C) to[short] (6,1)
     to[short] (10,1)
;
\end{schemat}

\subsubsection{Rozwiązanie}
Analizę układu należy rozpocząć od obserwacji iż zaznaczone na schemacie \ref{schemat:01:00:kw:A} impedancje $Z_6$ oraz $Z_7$ połączone są równolegle.
\begin{schemat}
\label{schemat:01:01:kw:A}
\draw
 node[ocirc,label=A] (A) at (0, 0) {}
 node[ocirc,label=below:B] (B) at (0,-3) {}
 node[circ,label=45:C] (C) at (6, 0) {}
 node[circ,label=below:D] (D) at (6,-3) {}
 
 (A) to[Z,l_=$Z_1$,o-*] (3,0)
     to[Z,l_=$Z_5$,*-*] (C)
     to[Z,l_=$Z_9$,*-*] (9,0)
     to[Z,l_=$Z_{11}$,*-*] (9,-3)     
     to[Z,l_=$Z_{10}$,*-*] (D)  
     to[Z,l_=$Z_8$,*-*] (C)
 (B) to[Z,l_=$Z_2$,o-*] (3,-3)
     to[Z,l=$Z_4$,*-] (3,-1.5)
     to[Z,l=$Z_5$] (3,0)
 (3,-3) to[short,-*] (3.5,-3)
        to[short] (3.5,-2.5)
        to[Z,l_=$Z_6$] (5.5,-2.5)
        to[short] (5.5,-3.5)        
        to[Z,l=$Z_7$] (3.5,-3.5)        
        to[short] (3.5,-3)  
 (5.5,-3) to[short,*-*] (D)
 (D) to[short] (9,0)
     to[short,-*] (10,1)
     to[Z,l=$Z_{12}$] (10,-3)
     to[short] (9,-3)
 (C) to[short] (6,1)
     to[short] (10,1)
;
\draw[color=red] (4.5,-3) ellipse (1 and 1.0);
\end{schemat}
Impedancje te zastępujemy jedną impedancją o impedancji zastępczej $Z_{67}$
\begin{align*}
\frac{1}{Z_{67}}&=\frac{1}{Z_{6}}+\frac{1}{Z_{7}}=\\
&=\frac{Z_{7}}{Z_{6}\cdot Z_{7}}+\frac{Z_{7}}{Z_{6}\cdot Z_{7}}=\\
&=\frac{Z_{7} + Z_{6}}{Z_{6}\cdot Z_{7}}=\\
Z_{67}&=\frac{Z_{6}\cdot Z_{7}}{Z_{7} + Z_{6}}
\end{align*}

\begin{schemat}
\label{schemat:01:01:kw:B}
\draw
 node[ocirc,label=A] (A) at (0, 0) {}
 node[ocirc,label=below:B] (B) at (0,-3) {}
 node[circ,label=45:C] (C) at (6, 0) {}
 node[circ,label=below:D] (D) at (6,-3) {}
 
 (A) to[Z,l_=$Z_1$,o-*] (3,0)
     to[Z,l_=$Z_5$,*-*] (C)
     to[Z,l_=$Z_9$,*-*] (9,0)
     to[Z,l_=$Z_{11}$,*-*] (9,-3)     
     to[Z,l_=$Z_{10}$,*-*] (D)  
     to[Z,l_=$Z_8$,*-*] (C)
 (B) to[Z,l_=$Z_2$,o-*] (3,-3)
     to[Z,l=$Z_4$,*-] (3,-1.5)
     to[Z,l=$Z_5$] (3,0)
 (3,-3) to[Z,l_=$Z_{67}$] (D)
     to[short] (9,0)
     to[short,-*] (10,1)
     to[Z,l=$Z_{12}$] (10,-3)
     to[short] (9,-3)
 (C) to[short] (6,1)
     to[short] (10,1)
;
\end{schemat}
%-----------------------------------------------
Następnie należy zauważyć iż impedancje $Z_{3}$ i $Z_{4}$ są połączone szeregowo i zastąpić je jedną impedancją zastępczą o wartości $Z_{35}$
\begin{schemat}
\label{schemat:01:01:kw:C}
\draw
 node[ocirc,label=A] (A) at (0, 0) {}
 node[ocirc,label=below:B] (B) at (0,-3) {}
 node[circ,label=45:C] (C) at (6, 0) {}
 node[circ,label=below:D] (D) at (6,-3) {}
 
 (A) to[Z,l_=$Z_1$,o-*] (3,0)
     to[Z,l_=$Z_5$,*-*] (C)
     to[Z,l_=$Z_9$,*-*] (9,0)
     to[Z,l_=$Z_{11}$,*-*] (9,-3)     
     to[Z,l_=$Z_{10}$,*-*] (D)  
     to[Z,l_=$Z_8$,*-*] (C)
 (B) to[Z,l_=$Z_2$,o-*] (3,-3)
     to[Z,l=$Z_4$,*-] (3,-1.5)
     to[Z,l=$Z_5$] (3,0)
 (3,-3) to[Z,l_=$Z_{67}$] (D)
     to[short] (9,0)
     to[short,-*] (10,1)
     to[Z,l=$Z_{12}$] (10,-3)
     to[short] (9,-3)
 (C) to[short] (6,1)
     to[short] (10,1)
;
\draw[color=red] (3,-1.5) ellipse (0.5 and 2.0);
\end{schemat}

\begin{align*}
Z_{34}&=Z_{3}+Z_{4}
\end{align*}

\begin{schemat}
\label{schemat:01:01:kw:D}
\draw
 node[ocirc,label=A] (A) at (0, 0) {}
 node[ocirc,label=below:B] (B) at (0,-3) {}
 node[circ,label=45:C] (C) at (6, 0) {}
 node[circ,label=below:D] (D) at (6,-3) {}
 
 (A) to[Z,l_=$Z_1$,o-*] (3,0)
     to[Z,l_=$Z_5$,*-*] (C)
     to[Z,l_=$Z_9$,*-*] (9,0)
     to[Z,l_=$Z_{11}$,*-*] (9,-3)     
     to[Z,l_=$Z_{10}$,*-*] (D)  
     to[Z,l_=$Z_8$,*-*] (C)
 (B) to[Z,l_=$Z_2$,o-*] (3,-3)
     to[Z,l=$Z_{34}$,*-] (3,0)
 (3,-3) to[Z,l_=$Z_{67}$] (D)
     to[short] (9,0)
     to[short,-*] (10,1)
     to[Z,l=$Z_{12}$] (10,-3)
     to[short] (9,-3)
 (C) to[short] (6,1)
     to[short] (10,1)
;
\end{schemat}
%-----------------------------------------------
Następnie należy zauważyć iż impedancje $Z_{11}$ oraz $Z_{12}$ są połączone równolegle i zastąpić je jedną impedancją zastępczą o wartości $Z_{1112}$
\begin{schemat}
\label{schemat:01:01:kw:E}
\draw
 node[ocirc,label=A] (A) at (0, 0) {}
 node[ocirc,label=below:B] (B) at (0,-3) {}
 node[circ,label=45:C] (C) at (6, 0) {}
 node[circ,label=below:D] (D) at (6,-3) {}
 
 (A) to[Z,l_=$Z_1$,o-*] (3,0)
     to[Z,l_=$Z_5$,*-*] (C)
     to[Z,l_=$Z_9$,*-*] (9,0)
     to[Z,l_=$Z_{11}$,*-*] (9,-3)     
     to[Z,l_=$Z_{10}$,*-*] (D)  
     to[Z,l_=$Z_8$,*-*] (C)
 (B) to[Z,l_=$Z_2$,o-*] (3,-3)
     to[Z,l=$Z_{34}$,*-] (3,0)
 (3,-3) to[Z,l_=$Z_{67}$] (D)
     to[short] (9,0)
     to[short,-*] (10,1)
     to[Z,l=$Z_{12}$] (10,-3)
     to[short] (9,-3)
 (C) to[short] (6,1)
     to[short] (10,1)
;
\draw[color=red] (9.5,-1.5) ellipse (1.0 and 1.0);
\end{schemat}

\begin{align*}
\frac{1}{Z_{1112}}&=\frac{1}{Z_{11}}+\frac{1}{Z_{12}}=\\
&=\frac{Z_{12}}{Z_{11}+Z_{12}}R+\frac{Z_{11}}{Z_{11}+Z_{12}}=\\
&=\frac{Z_{12} \cdot Z_{11}}{Z_{11}+Z_{12}}\\
Z_{1112}&=\frac{Z_{11}+Z_{12}}{Z_{12} \cdot Z_{11}}
\end{align*}

\begin{schemat}
\label{schemat:01:01:kw:F}
\draw
 node[ocirc,label=A] (A) at (0, 0) {}
 node[ocirc,label=below:B] (B) at (0,-3) {}
 node[circ,label=45:C] (C) at (6, 0) {}
 node[circ,label=below:D] (D) at (6,-3) {}
 
 (A) to[Z,l_=$Z_1$,o-*] (3,0)
     to[Z,l_=$Z_5$,*-*] (C)
     to[Z,l_=$Z_9$,*-*] (9,0)
 (9.5,-3) to[Z,l_=$Z_{10}$,-*] (D)  
     to[Z,l_=$Z_8$,*-*] (C)
 (B) to[Z,l_=$Z_2$,o-*] (3,-3)
     to[Z,l=$Z_{34}$,*-] (3,0)
 (3,-3) to[Z,l_=$Z_{67}$] (D)
     to[short] (9,0)
     to[short] (10,1)
 (9.5,0.5) to[Z,l_=$Z_{1112}$,*-*] (9.5,-3) 
 (C) to[short] (6,1)
     to[short] (10,1)
;
\end{schemat}
%-----------------------------------------------
Następnie należy zauważyć iż impedancja $Z_9$ jest połączona równolegle ze zwarciem. I można ją zastąpić jednym zwarciem. 
\begin{schemat}
\label{schemat:01:01:kw:G}
\draw
 node[ocirc,label=A] (A) at (0, 0) {}
 node[ocirc,label=below:B] (B) at (0,-3) {}
 node[circ,label=45:C] (C) at (6, 0) {}
 node[circ,label=below:D] (D) at (6,-3) {}
 
 (A) to[Z,l_=$Z_1$,o-*] (3,0)
     to[Z,l_=$Z_5$,*-*] (C)
     to[Z,l_=$Z_9$,*-*] (9,0)
 (9.5,-3) to[Z,l_=$Z_{10}$,-*] (D)  
     to[Z,l_=$Z_8$,*-*] (C)
 (B) to[Z,l_=$Z_2$,o-*] (3,-3)
     to[Z,l=$Z_{34}$,*-] (3,0)
 (3,-3) to[Z,l_=$Z_{67}$] (D)
     to[short] (9,0)
     to[short] (10,1)
 (9.5,0.5) to[Z,l_=$Z_{1112}$,*-*] (9.5,-3) 
 (C) to[short] (6,1)
     to[short] (10,1)
;
\draw[color=red] (7.5,0.5) ellipse (1.0 and 1.0);
\end{schemat}

\begin{align*}
\frac{1}{Z_{90}}&=\frac{1}{Z_9}+\frac{1}{0}=\\
&=\frac{0}{Z_9 \cdot 0}+\frac{Z_9}{Z_9 \cdot 0}=\\
&=\frac{0 + Z_9}{Z_9 \cdot 0}\\
Z_{90}&=\frac{Z_9 \cdot 0}{0 + Z_9}=\\
&=\frac{0}{Z_9}=0
\end{align*}
%
\begin{schemat}
\label{schemat:01:01:kw:H}
\draw
 node[ocirc,label=A] (A) at (0, 0) {}
 node[ocirc,label=below:B] (B) at (0,-3) {}
 node[circ,label=45:C] (C) at (6, 0) {}
 node[circ,label=below:D] (D) at (6,-3) {}
 
 (A) to[Z,l_=$Z_1$,o-*] (3,0)
     to[Z,l_=$Z_5$,*-*] (C)
     to[short] (6.0,0.5)
     to[short] (9.5,0.5)
 (9.5,-3) to[Z,l_=$Z_{10}$,-*] (D)  
     to[Z,l_=$Z_8$,*-*] (C)
 (B) to[Z,l_=$Z_2$,o-*] (3,-3)
     to[Z,l=$Z_{34}$,*-] (3,0)
 (3,-3) to[Z,l_=$Z_{67}$] (D)
     to[short] (9,0)
     to[short] (9.5,0.5) 
     to[Z,l_=$Z_{1112}$,*-*] (9.5,-3) 
;
\end{schemat}
%-----------------------------------------------
Następnie należy zauważyć iż impedancje $Z_{10}$ oraz $Z_{1112}$ jest połączona szeregowo. I można ją zastąpić jedną impedancją zastępczą o wartości $Z_{101112}$. 
\begin{schemat}
\label{schemat:01:01:kw:I}
\draw
 node[ocirc,label=A] (A) at (0, 0) {}
 node[ocirc,label=below:B] (B) at (0,-3) {}
 node[circ,label=45:C] (C) at (6, 0) {}
 node[circ,label=below:D] (D) at (6,-3) {}
 
 (A) to[Z,l_=$Z_1$,o-*] (3,0)
     to[Z,l_=$Z_5$,*-*] (C)
     to[short] (6.0,0.5)
     to[short] (9.5,0.5)
 (9.5,-3) to[Z,l_=$Z_{10}$,-*] (D)  
     to[Z,l_=$Z_8$,*-*] (C)
 (B) to[Z,l_=$Z_2$,o-*] (3,-3)
     to[Z,l=$Z_{34}$,*-] (3,0)
 (3,-3) to[Z,l_=$Z_{67}$] (D)
     to[short] (9,0)
     to[short] (9.5,0.5) 
     to[Z,l_=$Z_{1112}$,*-*] (9.5,-3) 
;
\draw[color=red,rotate around={-45:(8.5,-2.0)}] (8.5,-2.0) ellipse (1.0 and 2.0);
\end{schemat}

\begin{align*}
Z_{101112}&=Z_{10}+Z_{1112}\\
\end{align*}

Wstawiając obliczoną wcześniej wartość impedancji $Z_{1112}$

\begin{align*}
Z_{101112}&=Z_{10}+Z_{1112}=\\
&=Z_{10}+\frac{Z_{11}+Z_{12}}{Z_{12} \cdot Z_{11}}=\\
&=\frac{Z_{10} \cdot \left( Z_{12} \cdot Z_{11} \right)}{Z_{12} \cdot Z_{11}}+\frac{Z_{11}+Z_{12}}{Z_{12} \cdot Z_{11}}=\\
&=\frac{Z_{10} \cdot \left( Z_{12} \cdot Z_{11} \right)+ Z_{11} + Z_{12}}{Z_{12} \cdot Z_{11}}
\end{align*}

\begin{schemat}
\label{schemat:01:01:kw:J}
\draw
 node[ocirc,label=A] (A) at (0, 0) {}
 node[ocirc,label=below:B] (B) at (0,-3) {}
 node[circ,label=45:C] (C) at (6, 0) {}
 node[circ,label=below:D] (D) at (6,-3) {}
 
 (A) to[Z,l_=$Z_1$,o-*] (3,0)
     to[Z,l_=$Z_5$,*-*] (C)
     to[short] (6.0,0.5)
     to[short] (9.5,0.5)
 (D) to[Z,l_=$Z_8$,*-*] (C)
 (B) to[Z,l_=$Z_2$,o-*] (3,-3)
     to[Z,l=$Z_{34}$,*-] (3,0)
 (3,-3) to[Z,l_=$Z_{67}$] (D)
     to[short] (9,0)
     to[short] (9.5,0.5) 
     to[short] (9.5,-0.5)
     to[Z,l_=$Z_{101112}$] (7.0,-3)
     to[short] (D) 
;
\end{schemat}
%-----------------------------------------------
Następnie należy zauważyć iż impedancja $Z_{101112}$ połączona jest równolegle ze zwarciem. A więc można ją zastąpić zwarciem. 
\begin{schemat}
\label{schemat:01:01:kw:K}
\draw
 node[ocirc,label=A] (A) at (0, 0) {}
 node[ocirc,label=below:B] (B) at (0,-3) {}
 node[circ,label=45:C] (C) at (6, 0) {}
 node[circ,label=below:D] (D) at (6,-3) {}
 
 (A) to[Z,l_=$Z_1$,o-*] (3,0)
     to[Z,l_=$Z_5$,*-*] (C)
     to[short] (6.0,0.5)
     to[short] (9.5,0.5)
 (D) to[Z,l_=$Z_8$,*-*] (C)
 (B) to[Z,l_=$Z_2$,o-*] (3,-3)
     to[Z,l=$Z_{34}$,*-] (3,0)
 (3,-3) to[Z,l_=$Z_{67}$] (D)
     to[short] (9,0)
     to[short] (9.5,0.5) 
     to[short] (9.5,-0.5)
     to[Z,l_=$Z_{101112}$] (7.0,-3)
     to[short] (D) 
;
\draw[color=red,rotate around={45:(8.0,-1.5)}] (8.0,-1.5) ellipse (1.0 and 1.0);
\end{schemat}

\begin{align*}
\frac{1}{Z_{1011120}}&=\frac{1}{0} + \frac{1}{Z_{101112}}=\\
&=\frac{Z_{101112}}{0 \cdot Z_{101112}} + \frac{0}{0 \cdot Z_{101112}}=\\
&=\frac{Z_{101112}+0}{0 \cdot Z_{101112}}\\
Z_{1011120} &= \frac{0 \cdot Z_{101112}}{Z_{101112}+0}=0\\
\end{align*}

\begin{schemat}
\label{schemat:01:01:kw:L}
\draw
 node[ocirc,label=A] (A) at (0, 0) {}
 node[ocirc,label=below:B] (B) at (0,-3) {}
 node[circ,label=45:C] (C) at (6, 0) {}
 node[circ,label=below:D] (D) at (6,-3) {}
 
 (A) to[Z,l_=$Z_1$,o-*] (3,0)
     to[Z,l_=$Z_5$,*-*] (C)
     to[short] (6.0,0.5)
     to[short] (9.5,0.5)
 (D) to[Z,l_=$Z_8$,*-*] (C)
 (B) to[Z,l_=$Z_2$,o-*] (3,-3)
     to[Z,l=$Z_{34}$,*-] (3,0)
 (3,-3) to[Z,l_=$Z_{67}$] (D)
 (9.5,0.5) to[short] (9.5,0.0)
     to[short] (6.5,-3)
     to[short] (D) 
;
\end{schemat}
%-----------------------------------------------
Następnie należy zauważyć iż impedancja $Z_{8}$ połączona jest równolegle ze zwarciem. A więc można ją zastąpić zwarciem. 
\begin{schemat}
\label{schemat:01:01:kw:M}
\draw
 node[ocirc,label=A] (A) at (0, 0) {}
 node[ocirc,label=below:B] (B) at (0,-3) {}
 node[circ,label=45:C] (C) at (6, 0) {}
 node[circ,label=below:D] (D) at (6,-3) {}
 
 (A) to[Z,l_=$Z_1$,o-*] (3,0)
     to[Z,l_=$Z_5$,*-*] (C)
     to[short] (6.0,0.5)
     to[short] (9.5,0.5)
 (D) to[Z,l_=$Z_8$,*-*] (C)
 (B) to[Z,l_=$Z_2$,o-*] (3,-3)
     to[Z,l=$Z_{34}$,*-] (3,0)
 (3,-3) to[Z,l_=$Z_{67}$] (D)
 (9.5,0.5) to[short] (9.5,0.0)
     to[short] (6.5,-3)
     to[short] (D) 
;
\draw[color=red] (7.0,-1.5) ellipse (2.0 and 1.0);
\end{schemat}

\begin{align*}
\frac{1}{Z_{80}}&=\frac{1}{0} + \frac{1}{Z_{8}}=\\
&=\frac{Z_{8}}{0 \cdot Z_{8}} + \frac{0}{0 \cdot Z_{8}}=\\
&=\frac{Z_{8}+0}{0 \cdot Z_{8}}\\
Z_{80} &= \frac{0 \cdot Z_{8}}{Z_{8}+0}=0\\
\end{align*}

\begin{schemat}
\label{schemat:01:01:kw:N}
\draw
 node[ocirc,label=A] (A) at (0, 0) {}
 node[ocirc,label=below:B] (B) at (0,-3) {}
 node[circ,label=45:C] (C) at (6, 0) {}
 node[circ,label=below:D] (D) at (6,-3) {}
 
 (A) to[Z,l_=$Z_1$,o-*] (3,0)
     to[Z,l_=$Z_5$,*-*] (C)
 (D) to[short] (C)
 (B) to[Z,l_=$Z_2$,o-*] (3,-3)
     to[Z,l=$Z_{34}$,*-] (3,0)
 (3,-3) to[Z,l_=$Z_{67}$] (D)
;
\end{schemat}
%-----------------------------------------------
Następnie należy zauważyć iż impedancje $Z_{5}$ i $Z_{67}$ są połączone szeregowo. A więc można ją zastąpić jedną impedancją zastępczą o wartości $Z_{567}$. 
\begin{schemat}
\label{schemat:01:01:kw:O}
\draw
 node[ocirc,label=A] (A) at (0, 0) {}
 node[ocirc,label=below:B] (B) at (0,-3) {}
 node[circ,label=45:C] (C) at (6, 0) {}
 node[circ,label=below:D] (D) at (6,-3) {}
 
 (A) to[Z,l_=$Z_1$,o-*] (3,0)
     to[Z,l_=$Z_5$,*-*] (C)
 (D) to[short] (C)
 (B) to[Z,l_=$Z_2$,o-*] (3,-3)
     to[Z,l=$Z_{34}$,*-] (3,0)
 (3,-3) to[Z,l_=$Z_{67}$] (D)
;
\draw[color=red] (4.5,-1.5) ellipse (1.0 and 3.0);
\end{schemat}

\begin{align*}
Z_{567} &= Z_{5}+Z_{67}\\
\end{align*}

Wstawiając wcześniej obliczoną wartość impedancji $Z_{67}$

\begin{align*}
Z_{567} &= Z_{5}+Z_{67}=\\
&=Z_{5} + \frac{Z_{6}\cdot Z_{7}}{Z_{7} + Z_{6}} = \\
&=\frac{Z_{5} \cdot \left( Z_{7} + Z_{6} \right)}{Z_{7} + Z_{6}} + \frac{Z_{6}\cdot Z_{7}}{Z_{7} + Z_{6}} = \\
&=\frac{Z_{5} \cdot \left( Z_{7} + Z_{6} \right)+Z_{6}\cdot Z_{7} }{Z_{7} + Z_{6}}
\end{align*}


\begin{schemat}
\label{schemat:01:01:kw:P}
\draw
 node[ocirc,label=A] (A) at (0, 0) {}
 node[ocirc,label=below:B] (B) at (0,-3) {}

 (A) to[Z,l_=$Z_1$,o-*] (3,0)
     to[short,*-] (4,0)
     to[Z,l=$Z_{567}$] (4,-3)
     to[short] (3,-3)
 (B) to[Z,l_=$Z_2$,o-*] (3,-3)
     to[Z,l=$Z_{34}$,*-] (3,0)
;
\end{schemat}
%-----------------------------------------------
Następnie należy zauważyć iż impedancje $Z_{34}$ i $Z_{567}$ są połączone równolegle. A więc można ją zastąpić jedną impedancją zastępczą o wartości $Z_{34567}$. 
\begin{schemat}
\label{schemat:01:01:kw:R}
\draw
 node[ocirc,label=A] (A) at (0, 0) {}
 node[ocirc,label=below:B] (B) at (0,-3) {}

 (A) to[Z,l_=$Z_1$,o-*] (3,0)
     to[short,*-] (4,0)
     to[Z,l=$Z_{567}$] (4,-3)
     to[short] (3,-3)
 (B) to[Z,l_=$Z_2$,o-*] (3,-3)
     to[Z,l=$Z_{34}$,*-] (3,0)
;
\draw[color=red] (3.5,-1.5) ellipse (2.0 and 1.0);
\end{schemat}

\begin{align*}
\frac{1}{Z_{34567}} &= \frac{1}{Z_{34}}+\frac{1}{Z_{567}}\\
\end{align*}

Wstawiając wcześniej obliczone wartości impedancji $Z_{34}$ oraz $Z_{567}$

\begin{align*}
\frac{1}{Z_{34567}} &= \frac{1}{Z_{34}}+\frac{1}{Z_{567}}=\\
&=\frac{1}{Z_{3}+Z_{4}}+\frac{1}{\frac{Z_{5} \cdot \left( Z_{7} + Z_{6} \right)+Z_{6}\cdot Z_{7} }{Z_{7} + Z_{6}}}=\\
&=\frac{1}{Z_{3}+Z_{4}}+\frac{Z_{7} + Z_{6}}{Z_{5} \cdot \left( Z_{7} + Z_{6} \right)+Z_{6}\cdot Z_{7} }=\\
&=\frac{Z_{5} \cdot \left( Z_{7} + Z_{6} \right)+Z_{6}\cdot Z_{7}}{\left( Z_{3}+Z_{4} \right) \cdot \left( Z_{5} \cdot \left( Z_{7} + Z_{6} \right)+Z_{6}\cdot Z_{7} \right)} + \frac{Z_{3}+Z_{4}}{\left( Z_{3}+Z_{4} \right) \cdot \left( Z_{5} \cdot \left( Z_{7} + Z_{6} \right)+Z_{6}\cdot Z_{7} \right)}=\\
&=\frac{Z_{5} \cdot \left( Z_{7} + Z_{6} \right)+Z_{6}\cdot Z_{7}+Z_{3}+Z_{4}}{\left( Z_{3}+Z_{4} \right) \cdot \left( Z_{5} \cdot \left( Z_{7} + Z_{6} \right)+Z_{6}\cdot Z_{7} \right)}\\
Z_{34567}&=\frac{\left( Z_{3}+Z_{4} \right) \cdot \left( Z_{5} \cdot \left( Z_{7} + Z_{6} \right)+Z_{6}\cdot Z_{7} \right)}{Z_{5} \cdot \left( Z_{7} + Z_{6} \right)+Z_{6}\cdot Z_{7}+Z_{3}+Z_{4}}
\end{align*}

\begin{schemat}
\label{schemat:01:01:kw:S}
\draw
 node[ocirc,label=A] (A) at (0, 0) {}
 node[ocirc,label=below:B] (B) at (0,-3) {}

 (A) to[Z,l_=$Z_1$,o-] (3,0)
     to[short] (3.5,0)
     to[Z,l=$Z_{34567}$] (3.5,-3)
     to[short] (3,-3)
     to[Z,l_=$Z_2$,-o] (B)
;
\end{schemat}
%-----------------------------------------------
Ostatecznie należy zauważyć iż impedancje $Z_1$, $Z_{34567}$ oraz $Z_{2}$ są połączone szeregowo i zastąpić je jedną impedancją zastępczą o wartości $R_{1234567}$
\begin{schemat}
\label{schemat:01:01:kw:T}
\draw
 node[ocirc,label=A] (A) at (0, 0) {}
 node[ocirc,label=below:B] (B) at (0,-3) {}

 (A) to[Z,l_=$Z_1$,o-] (3,0)
     to[short] (3.5,0)
     to[Z,l=$Z_{34567}$] (3.5,-3)
     to[short] (3,-3)
     to[Z,l_=$Z_2$,-o] (B)
;
\draw[color=red] (2.5,-1.5) ellipse (3.0 and 3.0);
\end{schemat}

\begin{align*}
Z_{1234567}&=Z_{1}+Z_{34567}+Z_{2}\\
\end{align*}

Podstawiając obliczoną wcześniej wartość impedancji $Z_{34567}$

\begin{align*}
Z_{1234567}&=Z_{1}+Z_{34567}+Z_{2}=\\
&=Z_{1} + \frac{\left( Z_{3}+Z_{4} \right) \cdot \left( Z_{5} \cdot \left( Z_{7} + Z_{6} \right)+Z_{6}\cdot Z_{7} \right)}{Z_{5} \cdot \left( Z_{7} + Z_{6} \right)+Z_{6}\cdot Z_{7}+Z_{3}+Z_{4}} + Z_{2}
\end{align*}

\begin{schemat}
\label{schemat:01:01:kw:U}
\draw
 node[ocirc,label=A] (A) at (0, 0) {}
 node[ocirc,label=below:B] (B) at (0,-3) {}

 (A) to[short] (0.5,0)
     to[Z,l_=$Z_{1234567}$] (0.5,-3)
     to[short] (B)
;
\end{schemat}
Tak więc impedancja zastępcza obwodu przedstawionego na schemacie \ref{schemat:01:01:kw:Z} równa się $Z_{1234567}$
\end{task}