\begin{task}
Proszę wyznaczyć wartość mocy wydzielanej w rezystorze $R_1$ i $R_2$ w układzie pokazanym poniżej.
\begin{schemat} \draw
	(0,0)  to [Vsrc,l=$5$] (0,2)
	to [R,l=$R_1$, i=$I_A$,-*] (2,2)
	to [R,l=$R_2$,-*] (2,0)
	to [short] (0,0)
(2,4)	to [cVsrc,l=$20 \cdot I_A$] (2,2)
(2,4)	to [short] (4,4)
	to [R,l=$R_3$] (4,0)
	to [short] (0,0)
	;\end{schemat}
Wartości elementów wynoszą: $R_1=560 \ohm$, $R_2=110\ohm$, $R_3=240\ohm$.
\subsubsection{Rozwiązanie}

Wyznaczmy, metodą prądów oczkowych, prądy w układzie:
\begin{schemat} \draw
	(0,0)  to [Vsrc,l=$5$] (0,2)
	to [R,l=$R_1$, i=$I_A$,-*] (2,2)
	to [R,l=$R_2$,i=$I_{R2}$,-*] (2,0)
	to [short] (0,0)
	(2,4)	to [cVsrc,l=$20 \cdot I_A$] (2,2)
	(2,4)	to [short] (4,4)
	to [R,l=$R_3$] (4,0)
	to [short] (0,0);
	\draw[thick, rounded corners=12pt, ->, red]
	(0.3,0.3) -- (0.3,1.7) -- (1.7,1.7) -- (1.7,0.3) -- (1.1,0.3) node[anchor=south] {${I_{I}}$};
	\draw[thick, rounded corners=12pt, ->, red]
	(2.3,0.3) -- (2.3,3.7) -- (3.7,3.7) -- (3.7,0.3) -- (3.1,0.3) node[anchor=south] {${I_{II}}$};
	\draw[thick, rounded corners=12pt, ->, blue]
	(2.8,0.3) -- (2.8,1.7) node[anchor=west] {${U_{R2}}$};
	\draw[thick, rounded corners=12pt, ->, blue]
	(1.6,2.8) -- (0.3,2.8) node[anchor=north] {${U_{R1}}$};
	\end{schemat}
\begin{align}
\left[ \begin{array}{cc}
560+110 & -110\\
-110 & 240+110
\end{array} \right] \cdot 
\left[ \begin{array}{c}
I_I\\
I_{II}
\end{array} \right] = 
\left[ \begin{array}{c}
5\\
-20 \cdot I_A
\end{array} \right]
\end{align}
Przyglądając się układowi, zauważyć można, że $I_A=I_I$, zatem ostatecznie możemy przekształcić równanie macierzowe do postaci:
\begin{align}
\left[ \begin{array}{cc}
670 & -110\\
-110 & 350
\end{array} \right] \cdot 
\left[ \begin{array}{c}
I_I\\
I_{II}
\end{array} \right] = 
\left[ \begin{array}{c}
5\\
-20 \cdot I_I
\end{array} \right]\\
\left[ \begin{array}{cc}
670 & -110\\
-110 +20 & 350
\end{array} \right] \cdot 
\left[ \begin{array}{c}
I_I\\
I_{II}
\end{array} \right] = 
\left[ \begin{array}{c}
5\\
0
\end{array} \right]\\
\left[ \begin{array}{cc}
670 & -110\\
-90 & 350
\end{array} \right] \cdot 
\left[ \begin{array}{c}
I_I\\
I_{II}
\end{array} \right] = 
\left[ \begin{array}{c}
5\\
0
\end{array} \right]
\end{align}
Rozwiązując układ otrzymujemy
\begin{align}
\left[ \begin{array}{cc}
670 & -110\\
-90 & 350
\end{array} \right]^{-1} \cdot 
\left[ \begin{array}{cc}
670 & -110\\
-90 & 350
\end{array} \right] \cdot 
\left[ \begin{array}{c}
I_I\\
I_{II}
\end{array} \right] = 
\left[ \begin{array}{cc}
670 & -110\\
-90 & 350
\end{array} \right]^{-1} \cdot 
\left[ \begin{array}{c}
5\\
0
\end{array} \right]\\
\frac{1}{1000}\cdot \left[ \begin{array}{cc}
1,5583 & 0,4898\\
0,4007 & 2,9831
\end{array} \right] \cdot 
\left[ \begin{array}{cc}
670 & -110\\
-90 & 350
\end{array} \right] \cdot 
\left[ \begin{array}{c}
I_I\\
I_{II}
\end{array} \right] = 
\frac{1}{1000}\cdot \left[ \begin{array}{cc}
1,5583 & 0,4898\\
0,4007 & 2,9831
\end{array} \right] \cdot 
\left[ \begin{array}{c}
5\\
0
\end{array} \right]\\
\left[ \begin{array}{cc}
1 & 0\\
0 & 1
\end{array} \right] \cdot 
\left[ \begin{array}{c}
I_I\\
I_{II}
\end{array} \right] = 
\frac{1}{1000}\cdot  
\left[ \begin{array}{c}
7,7916\\
2,0036
\end{array} \right]
\end{align}
Zatem prąd $I_I=7,79 \: mA$, natomiast prąd $I_{II}=2 \: mA$. Możemy teraz wyznaczyć wartość mocy wydzielającej się na rezystorze $R_1$. Wiadomo, że dla prądu stałego, moc wydzielana na elemencie jest równa iloczynowi wartości prądu płynącego przez element i wartości napięcia na tym elemencie. Prąd płynący przez rezystor $R_1$ to ten sam prąd, który jest na schemacie oznaczony jako $I_A$. Zatem $I_{R1}=I_A=I_I=7,79 \: mA$.
\begin{align}
P_{R1} = I_{R1} \cdot U_{R1} = I_I \cdot ( I_I \cdot R_1) = 33,98mW
\end{align}
Na rezystorze $R_1$ wydziela się moc około 34mW.

Z kolei dla rezystora $R_2$ mamy:
\begin{align}
P_{R2} = I_{R2} \cdot U_{R2} = (I_I-I_{II}) \cdot ( (I_I-I_{II}) \cdot R_2) = 3,69mW
\end{align}
Na rezystorze $R_2$ wydziela się moc około 3,7mW.
\end{task}