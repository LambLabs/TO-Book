\externaldocument{zad_kk_07_12.tex}
\begin{task}
Na podstawie wyników uzyskanych w Zadaniu \ref{klima_zadanie1} , proszę wyznaczyć wartość współczynnika mocy (cos($\varphi$)) dla przypadku 2 rozważanego w tym zadaniu.
\subsubsection{Rozwiązanie}
%TBD
Wartość współczynnika mocy definiuje się jako stosunek mocy czynnej do wartości bezwzględnej mocy pozornej
\begin{align}
\cos(\varphi)=\frac{P}{|S|}
\end{align}
W przypadku 2 z Zadania \ref{klima_zadanie1} moc czynna miała wartość 6$\:$W, a moc pozorna $(6-2 \jmath)\:$VA, zatem 
\begin{align}
\cos(\varphi)=\frac{6}{|6-2 \jmath|}=\frac{6}{\sqrt{6^2+2^2}}=\frac{6}{6,3246}=0,9487
\end{align}
Warto również zauważyć, że ten sam wynik otrzymamy, gdy wartość bezwzględna mocy pozornej wyznaczona będzie przez pomnożenie modułu skutecznego prądu zespolonego i skutecznego napięcia zespolonego:
\begin{align}
|3,1113+0,5657j|\cdot |1,9799-0,2828j|=6,3246
\end{align}
co oznacza, że moc pozorna ma wartość równą mocy czynnej, jaka wydzieliłaby się na badanym elemencie, gdyby przy zachowaniu bez zmian wartości amplitud, prąd i napięcie były zgodne w fazie.
\end{task}