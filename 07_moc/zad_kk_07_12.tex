\begin{task}
\label{klima_zadanie1}
W prostym układzie pokazanym poniżej występuje źródło napięcia sinusoidalnego o amplitudzie wynoszącej $10V$ i o rezystancji wewnętrznej $R_S = 2 \ohm$. Źródło jest obciążone odbiornikiem o impedancji wynoszącej $Z_L$. Rozważmy dwa oddzielne przypadki.\bigskip\\
\bigskip
Przypadek 1: $Z_{L1} = 3+0 \jmath$\\
\bigskip
Przypadek 2: $Z_{L2} = 1,5+0,5 \jmath$\\
Dla każdego z przypadków proszę wyznaczyć:\\
1) moc pozorną i czynną wydzielaną w obciążeniu,\\
2) moc pozorną i czynną wydzielaną (traconą) w rezystancji wewnętrznej źródła.

%\begin{align}
%Z_w=2+2 \jmath \\
%C = 5 \farad \\
%R = 10 \ohm \\
%j(t) = 2 sin(t)
%\end{align}

%\begin{schemat} \draw
%(0,0)  to [Z,l=$Z_w$] (0,2)
%       to [short,-o] (2,2)
%       to [Z,l=$Z_1$] (4,2)
%       to [short] (5,2)
%       to [C,l=$C$] (5,0)
%       to [short,-o] (2,0)
%       to [short] (0,0)
%(1,0)  to [Isrc,l_=$j(t)$] (1,2)
%(4,0)  to [R,l=$R$] (4,2)
%;\end{schemat}

\begin{schemat} \draw
(0,0)  to [Vsrc,l=$u(t)$] (0,2)
	to [R,l=$R_S$] (2,2)
	to [short,-o] (3,2)
	to [short] (4,2)
	to [R,l=$Z_L$] (4,0)
	to [short,-o] (3,0)
	to [short] (0,0)
	;\end{schemat}

\subsubsection{Rozwiązanie}
%TBD

Wiemy, że moc pozorna wydzielana w dowolnym elemencie określana jest wzorem 
\begin{align}
\label{klima_S:1}
S=\underline{I_{sk}} \cdot \underline{U_{sk}}^*
\end{align} 
natomiast moc czynna jest częścią rzeczywistą mocy pozornej
\begin{align}
\label{klima_P:1}
P=\Re{(S)}=\Re{(\underline{I_{sk}} \cdot \underline{U_{sk}}^*)}
\end{align} 
gdzie $\underline{I_{sk}}$ to zespolona wartość skuteczna prądu, a $\underline{U_{sk}}^*$ to sprzężona wartość zespolonej wartości skutecznej napięcia.
Wiadomo również, że dla elementu o impedancji $Z$ słuszna jest zależność
\begin{align}
\label{klima_usk:1}
\underline{U_{sk}} = \underline{I_{sk}} \cdot Z
\end{align}
gdzie $Z$ to impedancja elementu.\\
Wyznaczmy zatem wartość skuteczną prądu płynącego w obwodzie $\underline{I_{sk}}$, według poniższego schematu. 
\begin{schemat} \draw
	(0,0)  to [Vsrc,l=$u(t)$] (0,2)
	to [R,l=$R_S$] (2,2)
	to [short,-o] (3,2)
	to [short] (4,2)
	to [R,l=$Z_L$] (4,0)
	to [short,-o] (3,0)
	to [short] (0,0);
	\draw[thick, rounded corners=12pt, ->, red]
	(0.3,0.3) -- (0.3,1.7) -- (3.7,1.7) -- (3.7,0.3) -- (3.1,0.3) node[anchor=south] {$\underline {I_{sk}}$};
	\draw[thick, rounded corners=12pt, ->, blue]
	(-0.3,0.3) -- (-0.3,1.7) node[anchor=east] {$\underline {U_{sk}}$};
	\draw[thick, rounded corners=12pt, ->, blue]
	(1.7,2.8) -- (0.3,2.8) node[anchor=south] {$\underline {U_{Ssk}}$};
	\draw[thick, rounded corners=12pt, ->, blue]
	(4.8,0.3) -- (4.8,1.7) node[anchor=west] {$\underline {U_{Lsk}}$};
	\end{schemat}

Zgodnie z napięciowym prawem Kirchoffa, $\underline{I_{sk}}$ będzie wynosił:
\begin{align}
\underline{I_{sk}}=\frac {\underline{U_{sk}} }{R_S + Z_L}
\end{align}
Przebieg napięcia jest określony jako sinusoidalny o amplitudzie 10V, zatem $\underline{U_{sk}}$ wynosi $\frac{10}{\sqrt{2}}$V.
Podstawiając wartości $R_s$ oraz $Z_L$ otrzymujemy kolejno wartości skuteczne prądu dla obu przypadków:
\begin{align}
\underline{I_{sk1}}=\frac{\frac{10}{\sqrt{2}} \:V}{(2+3+0 \jmath) \: \ohm}=\frac{2}{\sqrt{2}} \: A\approx 1,4142 \: A\\
\underline{I_{sk2}}=\frac{\frac{10}{\sqrt{2}} \:V}{(2+1,5+0,5 \jmath)  \:\ohm}=\frac{\frac{10}{\sqrt{2}} \:V}{(3,5+0,5 \jmath) \ohm}\approx (1,9799 - 0,2828 \jmath) \: A
\end{align}
Teraz możemy skorzystać z zależności \eqref{klima_usk:1} i wyznaczyć napięcia skuteczne występujące na obciążeniu
\begin{align}
\underline{U_{Lsk1}}=1,4142 \: A\cdot (3+0 \jmath)\ohm = 4,2426 \: V\\
\underline{U_{Lsk2}}=(1,9799 - 0,2828 \jmath) \: A \cdot (1,5+0,5 \jmath) \:\ohm = (3,1113 + 0,5657 \jmath) \: V
\end{align}
aby w końcu wyznaczyć, zgodnie ze wzorem \eqref{klima_S:1} wartości mocy pozornej na odbiorniku
\begin{align}
S_{L1} = 1,4142 \: A \cdot 4,2426 \: V \approx 6 \: VA
\end{align}
\begin{multline}
S_{L2} = (1,9799 - 0,2828 \jmath) \: A\cdot (3,1113 + 0,5657 \jmath)^* \: V=\\=(1,9799 - 0,2828 \jmath) \: A \cdot (3,1113 - 0,5657 \jmath) \: V \approx (6-2 \jmath) \: VA
\end{multline}
i zgodnie z \eqref{klima_P:1} wartość mocy czynnej wydzielanej na odbiorniku
\begin{align}
\label{klima_wynikPL:1}
P_{L1} = 6 W\\
\label{klima_wynikPL:2}
P_{L2} = 6 W
\end{align}
Następnie możemy wyznaczyć napięcie skuteczne na rezystancji wewnętrznej źródła:
\begin{align}
\underline{U_{Ssk1}}=1,4142 \: A\cdot (2+0 \jmath) \:\ohm = 2,8284 \: V\\
\underline{U_{Ssk2}}=(1,9799 - 0,2828 \jmath) \: A\cdot (2+0 \jmath) \:\ohm = (3,9598 - 0,5656 \jmath)  \:V
\end{align}
oraz ostatecznie moc pozorną:
\begin{align}
S_{S1} = 1,4142 \: A\cdot  2,8284 \: V = 4  \:VA \;
\end{align}
\begin{multline}
S_{S2} =  (1,9799 - 0,2828 \jmath) \: A\cdot (3,9598 - 0,5656 \jmath)^* \: V = \\=(1,9799 - 0,2828 \jmath) \: A\cdot (3,9598 + 0,5656 \jmath) \: V = 8 \: VA
\end{multline}
i moc czynną wydzielaną na rezystancji wewnętrznej źródła:
\begin{align}
\label{klima_wynikPS:1}
P_{S1} = 4 W\\
\label{klima_wynikPS:2}
P_{S2} = 8 W
\end{align}
Porównanie wyników uzyskanych dla dwóch powyższych przypadków pozwala zauważyć, że moc czynna dostarczana do odbiornika jest w obu przypadkach taka sama \eqref{klima_wynikPL:1}, \eqref{klima_wynikPL:2}, jednak moc czynna tracona w rezystancji wewnętrznej źródła jest dwukrotnie większa dla drugiego przypadku \eqref{klima_wynikPS:2} niż dla pierwszego przypadku \eqref{klima_wynikPS:1} Sprawność przekazywania energii do obciążenia jest zatem mniejsza dla układu, w którym obciążenie charakteryzuje się niezerową wartością reaktancji.
\end{task}