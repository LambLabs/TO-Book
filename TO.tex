\documentclass[a4paper,11pt]{LambBook} %6 inch 800x600

\usepackage[T1]{fontenc}
\usepackage[polish]{babel}
\usepackage[utf8]{inputenc}
\usepackage{lmodern}
\usepackage{mathtools}
\usepackage{amsmath}
\usepackage{cancel}
\usepackage{graphicx} 
\usepackage{array}
\usepackage{float} %Wymuszenie wstawienia obrazka w miejscu wstawienia
%\usepackage[europeancurrents, europeanvoltages, europeanresistors, americaninductors, europeanports]{circuitikz} %Do rysowania obwodów elektrycznych
\usepackage[americancurrents, americanvoltages, europeanresistors, americaninductors, europeanports]{circuitikz} %Do rysowania obwodów elektrycznych
\usepackage{polishcircuitikz} %Polskie symbole elementów obwodów (Źródła i inne nasze)
\usepackage[ISBN=978-83-939620-2-0,SC2]{ean13isbn} %Drukowanie numeru ISBN
\usepackage{tikz}
\selectlanguage{polish}
\usepackage{xr}

\allowdisplaybreaks
%https://www.codecogs.com/latex/eqneditor.php
%http://www.sciweavers.org/free-online-latex-equation-editor

\def\booktitle{Teoria Obwodów w zadaniach}
\def\bookauthors{Jakub Stankowski, Agnieszka Wardzińska, Krzysztof Wegner, Krzysztof Klimaszewski}

\usepackage[pdftex,
        unicode=true, % Aby działały polskie literki
        colorlinks=true,
        urlcolor=rltblue,       % \href{...}{...} external (URL)
        filecolor=rltgreen,     % \href{...} local file
        linkcolor=rltred,       % \ref{...} and \pageref{...}
        citecolor=blue,
        pdfstartview={FitV},
        pdftitle={\booktitle},
        pdfauthor={\bookauthors},
        pdfsubject={Teoria Obwodów},
        pdfkeywords={Teoria Obwodów, Zadania},
        pdfproducer={pdfLaTeX},
        %pdfadjustspacing=1,
        pagebackref=false, %activate back references inside bibliography. Must be specified as part
        bookmarksopen=true]{hyperref}

\setcounter{secnumdepth}{2}

%\setdebug
\setrelease
%====================================================
\begin{document}
%====================================================
\definecolor{rltred}{rgb}{0.75,0,0} %Definicja kolorów
\definecolor{rltgreen}{rgb}{0,0.5,0}
\definecolor{rltblue}{rgb}{0,0,0.75}
%====================================================
\title{\booktitle}
%\author{\bookauthors}
\author{J. Stankowski, A. Wardzińska, K. Wegner, K. Klimaszewski}
%====================================================
%Strona Tytułowa
\label{page:titlepage}
%====================================================
% Title Page definition
\thispagestyle{empty}
\begin{titlepage}
  \begin{center}
    \baselineskip=100pt
    {\fontsize{100}{120}\selectfont  Teoria Obwodów} \\[1em] {\fontsize{80}{100}\selectfont w zadaniach}
    \baselineskip=40pt%10pt
     
    %\\[6em]
    \bookauthors
    
    \vfill
    
    % Bottom of the page
    %{\large \date }
    {\large \today }
    
  \end{center}
\end{titlepage}
\cleardoublepage
%====================================================
%====================================================
%Druga Strona z ISBN'em
\label{page:firstpage}
\input{FirstPage.tex}
%====================================================
\chapter{Prawo Ohma, rezystancja zastępcza, dzielniki pradowe i napięciowe, zwijanie obwodu}

\section{Teoria}
\subsection{Prawo Ohma}

Rezystancja (opór) - R[Ohm] - R[$\Omega$]
\begin{equation}
R=\frac{U}{I}
\end{equation}
\begin{equation}
U=R \cdot I
\end{equation}
\begin{equation}
I=\frac{U}{R}
\end{equation}

Konduktancja - G[Siemens] - G[S]
\begin{equation}
G=\frac{1}{R}
\end{equation}
\begin{equation}
R=\frac{1}{G}
\end{equation}

















\subsection{Opór zastępczy}
\subsubsection{Szeregowe łączenie rezystorów}
\begin{schemat}
\draw
 node[ocirc] (A) at (0,0) {}
 node[ocirc] (B) at (4,0) {}
 (0,0) to[R,l=$R_1$] (2,0) 
 (2,0) to[R,l=$R_2$] (4,0)
;
\end{schemat}

\begin{equation}
R_z=R_1+R_2
\end{equation}

\begin{schemat} 
\draw
 node[ocirc] (A) at (0,0) {}
 node[ocirc] (B) at (8,0) {}
 (0,0) to[R,l=$R_1$] (2,0) 
 (2,0) to[R,l=$R_2$] (4,0)
 (6,0) to[R,l=$R_n$] (8,0)
;
\end{schemat}
\begin{equation}
R_z=R_1+R_2+...+R_n
\end{equation}

\subsubsection{Równoległe łączenie rezystorów}

\begin{schemat} 
\draw
 node[ocirc] (A) at ( 0, 0) {}
 node[ocirc] (B) at ( 4, 0) {}
 ( 0, 0) to[short]     ( 1, 0)
 ( 1, 0) to[short]     ( 1, 1)
 ( 1, 0) to[short]     ( 1,-1)
 ( 1, 1) to[R,l=$R_1$] ( 3, 1) 
 ( 1,-1) to[R,l=$R_2$] ( 3,-1)
 ( 3, 1) to[short]     ( 3, 0)
 ( 3,-1) to[short]     ( 3, 0)
 ( 3, 0) to[short]     ( 4, 0)
;
\end{schemat}

\begin{equation}
R_z=\frac{R_1 \cdot R_2}{R_1+R_2}
\end{equation}

\begin{schemat}
\draw
 node[ocirc] (A) at ( 0, 0) {}
 node[ocirc] (B) at ( 4, 0) {}
 ( 0, 0) to[short]     ( 1, 0)
 ( 1, 0) to[short]     ( 1, 1)
 ( 1, 0) to[short]     ( 1,-0.5)
 ( 1, -1) to[short]     ( 1,-1.5)
 ( 1, 1) to[R,l=$R_1$] ( 3, 1) 
 ( 1, 0) to[R,l=$R_2$] ( 3, 0)
 ( 1, -1.5) to[R,l=$R_n$] ( 3, -1.5)
 ( 3, 1) to[short]     ( 3, 0)
 ( 3,-0.5) to[short]   ( 3, 0)
 ( 3,-1) to[short]   ( 3, -1.5)
 ( 3, 0) to[short]     ( 4, 0)
;
\end{schemat}

\begin{equation}
\frac{1}{R_z}=\frac{1}{R_1} + \frac{1}{R_2} + ... + \frac{1}{R_n}
\end{equation}
\subsection{Dzielniki napięciowe i prądowe}

\subsubsection{Dzielnik napięciowy}
\begin{schemat}
\draw
 node[ocirc] (A) at ( 0, 0) {}
 node[ocirc] (B) at ( 0,-4) {}
 (A) to[open, v=$U$] (B)
 ( 0, 0) to[short,i=$I$]       ( 2, 0)
 ( 2, 0) to[R,l=$R_1$,v=$U_1$] ( 2,-2)
 ( 2,-2) to[R,l=$R_2$,v=$U_2$] ( 2,-4)
 ( 0,-4) to[short]     ( 2,-4)
;
\end{schemat}

\begin{equation}
U=U_1+U_2
\end{equation}
\begin{equation}
U_1=R_1 \cdot I
\end{equation}
\begin{equation}
U_2=R_2 \cdot I
\end{equation}
\begin{equation}
I=\frac{U}{R_z}=\frac{U}{R_1+R_2}
\end{equation}

\begin{equation}
U_1=\frac{R_1}{R_1+R_2} \cdot U
\end{equation}
\begin{equation}
U_2=\frac{R_2}{R_1+R_2} \cdot U
\end{equation}

\begin{equation}
U_1=\frac{G_2}{G_1+G_2} \cdot U
\end{equation}
\begin{equation}
U_2=\frac{G_1}{G_1+G_2} \cdot U
\end{equation}


\subsubsection{Dzielnik pradowy}
\begin{schemat} 
%przesynąć etykiety węzłów dalej od węzłów
\draw
 node[ocirc] (A) at ( 0, 0) {}
 node[ocirc] (B) at ( 0,-4) {}
 (A) to[open, v=$U$] (B)
 ( 0, 0) to[short,i=$I$]       ( 2, 0)
 ( 2, 0) to[short      ]       ( 2,-1)
 ( 2,-1) to[short      ]       ( 1,-1)
 ( 2,-1) to[short      ]       ( 3,-1) 
 ( 1,-1) to[R,l=$R_1$,i>^=$I_1$] ( 1,-3)
 ( 3,-1) to[R,l=$R_2$,i>^=$I_2$] ( 3,-3)
 ( 2,-3) to[short      ]       ( 1,-3)
 ( 2,-3) to[short      ]       ( 3,-3)
 ( 2,-3) to[short      ]       ( 2,-4)
 ( 0,-4) to[short]     ( 2,-4)
;
\end{schemat}

\begin{equation}
I=I_1+I_2
\end{equation}
\begin{equation}
I=\frac{U}{R_z}=\frac{U}{\frac{R_1 \cdot R_2}{R_1+R_2}}
\end{equation}
\begin{equation}
I_1=\frac{U}{R_1}
\end{equation}
\begin{equation}
I_2=\frac{U}{R_2}
\end{equation}

\begin{equation}
I_1=\frac{R_2}{R_1+R_2} \cdot I
\end{equation}
\begin{equation}
I_2=\frac{R_1}{R_1+R_2} \cdot I
\end{equation}

\begin{equation}
I_1=\frac{G_1}{G_1+G_2} \cdot I
\end{equation}
\begin{equation}
I_2=\frac{G_2}{G_1+G_2} \cdot I
\end{equation}
\subsection{Przekształcenie trójkat-gwiazda i gwiazda-trójkąt}

\subsubsection{Trójkąt}
\begin{schemat}
\draw
 node[ocirc] (A) at (   0, 3.5  ) {$A$}
 node[ocirc] (B) at (-3.5,-2.5  ) {$B$}
 node[ocirc] (C) at ( 3.5,-2.5  ) {$C$}
 
 node[circ] (Aa) at ( 0, 2) {}
 node[circ] (Bb) at (-2, -1.5) {}
 node[circ] (Cc) at ( 2, -1.5) {}
 
 (A) to[short] (Aa)
 (B) to[short] (Bb)
 (C) to[short] (Cc)
 
 (Aa) to[R,l=$R_1$] (Bb)
 (Aa) to[R,l=$R_2$] (Cc)
 (Bb) to[R,l=$R_3$] (Cc) 
;
\end{schemat}

\subsubsection{Gwiazda}
\begin{schemat}
\draw
 node[ocirc] (A) at (   0, 3.5  ) {$A$}
 node[ocirc] (B) at (-3.5,-2.5  ) {$B$}
 node[ocirc] (C) at ( 3.5,-2.5  ) {$C$}
 node[circ ] (D) at (   0,-0.5  ) {   }
 (A) to[R,l=$R_A$] (D)
 (B) to[R,l=$R_B$] (D)
 (C) to[R,l=$R_C$] (D) 
;
\end{schemat}

\subsubsection{Przekształcenie trójkat-gwiazda}
\begin{equation}
R_A=\frac{R_1 \cdot R_2}{R_1 + R_2 + R_3}
\end{equation}
\begin{equation}
R_B=\frac{R_1 \cdot R_3}{R_1 + R_2 + R_3}
\end{equation}
\begin{equation}
R_C=\frac{R_2 \cdot R_3}{R_1 + R_2 + R_3}
\end{equation}

\subsubsection{Przekształcenie gwiazda-trójkąt}
\begin{equation}
R_1=R_A+R_B+\frac{R_A \cdot R_B}{R_C}
\end{equation}
\begin{equation}
R_2=R_A+R_C+\frac{R_A \cdot R_C}{R_B}
\end{equation}
\begin{equation}
R_3=R_b+R_C+\frac{R_B \cdot R_C}{R_A}
\end{equation}

\section{Zadania}
\begin{task}
Wyznacz opór zastępczy układu
\begin{schemat}
\draw
 node[ocirc] (A) at ( 0,-0.5) {}
 node[ocirc] (B) at (9.5,-0.5) {}
 
 (0, -0.5) to[R,l=$R$] (2, -0.5) 
 (2, 0) to[R,l=$R$] (8.5, 0)
 (2,-1) to[R,l=$R$] (4,-1)
 (4,-1) to[R,l=$R$] (6,-1) 
 (6,-1) to[R,l=$R$] (8,-1) 
 (4,-2) to[R,l=$R$] (8,-2)
 
 (2, -0.5) to[short] (2, 0)
 (2, -0.5) to[short] (2,-1) 
 (4,-1) to[short] (4,-2)
 (8,-1) to[short] (8,-2)
 (8,-1) to[short] (8.5,-1)
 (8.5,-1) to[short] (8.5, 0)
 (8.5, -0.5) to[short] (9.5, -0.5)
;
\end{schemat}
\subsubsection{Rozwiązanie}
TBD
\end{task}
\begin{task}
Wyznacz opór zastępczy widziany z zacisków A i F. R=6$\Omega$
\begin{schemat}
\draw
 node[circ] (A) at ( 0, 0) {$A$}
 node[circ] (B) at ( 2, 0) {$B$}
 node[circ] (C) at ( 4, 0) {$C$}
 node[circ] (D) at ( 6, 0) {$D$}
 node[circ] (E) at ( 0,-2) {$E$}
 node[circ] (F) at ( 2,-2) {$F$}
 node[circ] (G) at ( 4,-2) {$G$}
 node[circ] (H) at ( 6,-2) {$H$}
 
 (A) to[R,l=$R$] (B) 
 (B) to[R,l=$R$] (C) 
 (C) to[R,l=$R$] (D) 
 (A) to[R,l=$R$] (E) 
 (B) to[R,l=$R$] (F) 
 (C) to[R,l=$R$] (G) 
 (D) to[R,l=$R$] (H) 
 (E) to[R,l=$R$] (F) 
 (F) to[R,l=$R$] (G) 
 (G) to[R,l=$R$] (H) 
;
\end{schemat}
\subsubsection{Rozwiązanie}
TBD
\end{task}
\begin{task}
Oblicz rezystancję zastępczą widziana z zacisków a i b.
\begin{schemat}
\draw
 node[ocirc] (a) at ( 0, 0) {$a$}
 node[ocirc] (b) at ( 0,-3) {$b$}
  
 node[circ] (n0) at ( 2, 0) {}
 node[circ] (n1) at ( 4, 0) {}
 node[circ] (n2) at ( 6, 0) {}
 
 (a) to[R,l=$R$] (n0)
 (n0) to[R,l=$2R$] (n1) 
 (n1) to[R,l=$2R$] (n2) 
 (n2) to[R,l=$2R$] (8, 0)
 
 (n0) to[short] (2,1)
 (2,1) to[short] (6,1)
 (6,1) to[short] (6,0)
 
 (n1) to[short] (4,-1)
 (4,-1) to[short] (8,-1)
 
 (n0) to[short] (2,-2)
 (2,-2) to[R,l=$2R$] (8, -2)
 
 (8,0) to[short] (8,-3)
 (8,-3) to[short] (2,-3)
 (b) to[R,l=$R$] (2,-3) 
;
\end{schemat}
\subsubsection{Rozwiązanie}
TBD
\end{task}
\begin{task}
Oblicz rezystancję zastępczą widziana z zacisków a i b.
\begin{schemat} 
\draw
 node[ocirc] (a) at ( 0, 0) {$a$}
 node[ocirc] (b) at ( 0,-2) {$b$}
  
 (a) to[R,l=$R$] ( 2, 0)
 ( 2, 0) to[R,l=$10R$] ( 4, 0)
 ( 4, 0) to[R,l=$10R$] ( 6, 0) 
 ( 6, 0) to[R,l=$10R$] ( 8, 0)
 ( 8, 0) to[R,l=$10R$] (10, 0)
 
 ( 2,0) to[short] ( 2, 1)
 ( 6,0) to[short] ( 6, 1) 
 (10,1) to[short] (10,-2)
 ( 2,1) to[short] (10, 1) 

 ( 4, 0) to[short] ( 4,-1)
 ( 8, 0) to[short] ( 8,-1)
 ( 4,-1) to[short] ( 8,-1)
 ( 6,-1) to[short] ( 6,-2)
 
 ( 6,-2) to[R,l=$10R$] ( 8,-2)
 ( 8,-2) to[short]     (10,-2)
 (b) to[R,l=$2R$] ( 2, -2)
 ( 2, -2) to[short]  ( 6,-2)
;
\end{schemat}
\subsubsection{Rozwiązanie}
TBD
\end{task}
\begin{task}
Oblicz rezystancję zastępczą widziana z zacisków a i b. \\*
$R_1=R_3=20\Omega, R_2=R_4=40\Omega, R_5=30\Omega, R_6-R_7=30\Omega$

\begin{schemat}
\draw
 node[ocirc] (a) at ( 0, 0) {$a$}
 node[ocirc] (b) at ( 0,-4) {$b$}
  
 (a) to[short] ( 6, 0)
 (b) to[short] ( 6,-4)
 
 ( 2, 0) to[R,l=$R_1$] ( 2,-2)
 ( 2,-2) to[R,l=$R_5$] ( 2,-4)
 ( 2,-2) to[R,l=$R_3$] ( 4,-2)
 ( 4, 0) to[short    ] ( 4,-2)
 ( 4,-2) to[R,l=$R_7$] ( 4,-4)
 ( 4,-2) to[R,l=$R_4$] ( 6,-2)
 ( 6, 0) to[R,l=$R_2$] ( 6,-2)
 ( 6,-2) to[R,l=$R_6$] ( 6,-4)
;
\end{schemat}
\subsubsection{Rozwiązanie}
TBD
\end{task}
\begin{task}
Oblicz rezystancję zastępczą widziana z zacisków a i b. \\*
$R_1=10\Omega, R_2=20\Omega, R_3=30\Omega, R_4=40\Omega, R_5=50\Omega$

\begin{schemat}
\draw
 node[ocirc] (a) at ( 0, 0) {$a$}
 node[ocirc] (b) at ( 6, 0) {$b$}
  
 (a) to[short] ( 1, 0)
 (b) to[short] ( 5, 0)
 
 ( 1, 0) to[R,l=$R_1$] ( 3, 2)
 ( 1, 0) to[R,l=$R_3$] ( 3,-2)
 ( 3, 2) to[R,l=$R_5$] ( 3,-2)
 ( 5, 0) to[R,l=$R_2$] ( 3, 2)
 ( 5, 0) to[R,l=$R_4$] ( 3,-2)
;
\end{schemat}
\subsubsection{Rozwiązanie}
TBD
\end{task}
\begin{task}
Oblicz rezystancję zastępczą widziana z zacisków a i b. \\*

\begin{schemat}
\draw
 node[ocirc] (a) at ( 0, 0) {$a$}
 node[ocirc] (b) at ( 8, 0) {$b$}
  
 (a) to[short] ( 1, 0)
 (b) to[short] ( 7, 0)
 
 ( 1, 0) to[R,l=$3R$] ( 7, 0)
 ( 1, 0) to[R,l=$3R$] ( 4, 4)
 ( 4, 4) to[R,l=$3R$] ( 7, 0)
 
 ( 1, 0) to[R,l=$R$] ( 4, 1.5)
 (4, 1.5) to[R,l=$R$] ( 7, 0)
 (4, 1.5) to[R,l=$R$] ( 4, 4)
;
\end{schemat}
\subsubsection{Rozwiązanie}
TBD
\end{task}

\begin{task}
Wyznacz opór zastępczy poniższego układu. Podaj wzór na opór zastępczy oraz jego wartość. Przyjmij że $R=3\Omega$
\begin{schemat}
\label{schemat:01:00:kw:Z}
\draw
 (0,0) to[R,l_=$R$,o-*] (3,0) 
 	   to[short] (3,-1)
       to[R,l_=$R$,-*] (5,-1)
       to[R,l_=$R$] (7,-1)
       to[R,l_=$R$,-*] (9,-1)
       to[short] (9.5,-1)
       to[short,-*] (9.5,0)
       to[short,-o] (10.5,0)
 (5,-1) to[short] (5,-2)
 		to[R,l_=$R$] (9,-2)
 		to[short] (9,-1)
 (3,0) to[short] (3,1)
       to[R,l_=$R$] (9.5,1) 	
       to[short] (9.5,0)
;
\end{schemat}

\subsubsection{Rozwiązanie}
Należy zauważyć iż zaznaczone na schemacie \ref{schemat:01:00:kw:A} oporniki połączone są szeregowo.
\begin{schemat}
\label{schemat:01:00:kw:A}
\draw
 (0,0) to[R,l_=$R$,o-*] (3,0) 
 	   to[short] (3,-1)
       to[R,l_=$R$,-*] (5,-1)
       to[R,l_=$R$] (7,-1)
       to[R,l_=$R$,-*] (9,-1)
       to[short] (9.5,-1)
       to[short,-*] (9.5,0)
       to[short,-o] (10.5,0)
 (5,-1) to[short] (5,-2)
 		to[R,l_=$R$] (9,-2)
 		to[short] (9,-1)
 (3,0) to[short] (3,1)
       to[R,l_=$R$] (9.5,1) 	
       to[short] (9.5,0)
;
\draw[color=red] (7,-1) ellipse (2 and 0.5);
\end{schemat}
Oporniki te zastępujemy jednym opornikiem o oporze zastępczym $R_{Z1}$
\begin{equation*}
R_{Z1}=R+R=2R
\end{equation*}
\begin{schemat}
\label{schemat:01:00:kw:B}
\draw
 (0,0) to[R,l_=$R$,o-*] (3,0) 
 	   to[short] (3,-1)
       to[R,l_=$R$,-*] (5,-1)
       to[R,l_=$R_{Z1} \equal 2R$,-*] (9,-1)
       to[short] (9.5,-1)
       to[short,-*] (9.5,0)
       to[short,-o] (10.5,0)
 (5,-1) to[short] (5,-2)
 		to[R,l_=$R$] (9,-2)
 		to[short] (9,-1)
 (3,0) to[short] (3,1)
       to[R,l_=$R$] (9.5,1) 	
       to[short] (9.5,0)
;
\end{schemat}
Następnie należy zauważyć iż zaznaczone na schemacie \ref{schemat:01:00:kw:C} są połączone równoległe i zastąpić je jednym oporem zastępczym o wartości $R_{Z2}$
\begin{schemat}
\label{schemat:01:00:kw:C}
\draw
 (0,0) to[R,l_=$R$,o-*] (3,0) 
 	   to[short] (3,-1)
       to[R,l_=$R$,-*] (5,-1)
       to[R,l_=$R_{Z1} \equal 2R$,-*] (9,-1)
       to[short] (9.5,-1)
       to[short,-*] (9.5,0)
       to[short,-o] (10.5,0)
 (5,-1) to[short] (5,-2)
 		to[R,l_=$R$] (9,-2)
 		to[short] (9,-1)
 (3,0) to[short] (3,1)
       to[R,l_=$R$] (9.5,1) 	
       to[short] (9.5,0)
;
\draw[color=red] (7,-1.5) ellipse (2 and 1.0);
\end{schemat}
\begin{align*}
\frac{1}{R_{Z2}}&=\frac{1}{R_{Z1}}+\frac{1}{R}\\
\frac{1}{R_{Z2}}&=\frac{1}{2R}+\frac{1}{R}=\\
&=\frac{1}{2R}+\frac{2}{2R}=\\
&=\frac{3}{2R}\\
R_{Z2}=\frac{2}{3}R
\end{align*}
\begin{schemat}
\label{schemat:01:00:kw:D}
\draw
 (0,0) to[R,l_=$R$,o-*] (3,0) 
 	   to[short] (3,-1)
       to[R,l_=$R$] (5,-1)
       to[short] (5,-1.5) 
       to[R,l_=$R_{Z2} \equal \frac{2}{3}R$] (9,-1.5)
       to[short] (9,-1) 
       to[short] (9.5,-1)
       to[short,-*] (9.5,0)
       to[short,-o] (10.5,0)
 (3,0) to[short] (3,1)
       to[R,l_=$R$] (9.5,1) 	
       to[short] (9.5,0)
;
\end{schemat}
Następnie należy zauważyć iż zaznaczone na schemacie \ref{schemat:01:00:kw:E} są połączone szeregowo i zastąpić je jednym oporem zastępczym o wartości $R_{Z3}$
\begin{schemat}
\label{schemat:01:00:kw:E}
\draw
 (0,0) to[R,l_=$R$,o-*] (3,0) 
 	   to[short] (3,-1)
       to[R,l_=$R$] (5,-1)
       to[short] (5,-1.5) 
       to[R,l_=$R_{Z2} \equal \frac{2}{3}R$] (9,-1.5)
       to[short] (9,-1) 
       to[short] (9.5,-1)
       to[short,-*] (9.5,0)
       to[short,-o] (10.5,0)
 (3,0) to[short] (3,1)
       to[R,l_=$R$] (9.5,1) 	
       to[short] (9.5,0)
;
\draw[color=red] (5.5,-1.25) ellipse (2.5 and 1.0);
\end{schemat}
\begin{align*}
R_{Z3}&=\frac{2}{3}R+R\\
R_{Z3}&=\frac{2}{3}R+\frac{3}{3}R\\
&=\frac{5}{3}R
\end{align*}
\begin{schemat}
\label{schemat:01:00:kw:F}
\draw
 (0,0) to[R,l_=$R$,o-*] (3,0) 
 	   to[short] (3,-1)
       to[R,l_=$R_{Z3} \equal \frac{5}{3}R$] (9.5,-1)
       to[short,-*] (9.5,0)
       to[short,-o] (10.5,0)
 (3,0) to[short] (3,1)
       to[R,l_=$R$] (9.5,1) 	
       to[short] (9.5,0)
;
\end{schemat}
Następnie należy zauważyć iż zaznaczone na schemacie \ref{schemat:01:00:kw:G} są połączone równolegle i zastąpić je jednym oporem zastępczym o wartości $R_{Z4}$
\begin{schemat}
\label{schemat:01:00:kw:G}
\draw
 (0,0) to[R,l_=$R$,o-*] (3,0) 
 	   to[short] (3,-1)
       to[R,l_=$R_{Z3} \equal \frac{5}{3}R$] (9.5,-1)
       to[short,-*] (9.5,0)
       to[short,-o] (10.5,0)
 (3,0) to[short] (3,1)
       to[R,l_=$R$] (9.5,1) 	
       to[short] (9.5,0)
;
\draw[color=red] (6.25,0.0) ellipse (2.0 and 2.0);
\end{schemat}
\begin{align*}
\frac{1}{R_{Z4}}&=\frac{1}{R}+\frac{1}{R_{Z3}}\\
\frac{1}{R_{Z4}}&=\frac{1}{R}+\frac{1}{\frac{5}{3}R}=\\
&=\frac{1}{R}+\frac{3}{5R}=\\
&=\frac{5}{5R}+\frac{3}{5R}=\\
&=\frac{8}{5R}\\
R_{Z4}&=\frac{5}{8}R\\
\end{align*}
\begin{schemat}
\label{schemat:01:00:kw:H}
\draw
 (0,0) to[R,l_=$R$,o-] (3,0) 
       to[R,l_=$R_{Z4} \equal \frac{5}{8}R$] (9.5,0)
       to[short,-o] (10.5,0)
;
\end{schemat}
Ostatecznie należy zauważyć iż zaznaczone na schemacie \ref{schemat:01:00:kw:I} są połączone szeregowo i zastąpić je jednym oporem zastępczym o wartości $R_{Z5}$
\begin{schemat}
\label{schemat:01:00:kw:I}
\draw
 (0,0) to[R,l_=$R$,o-] (3,0) 
       to[R,l_=$R_{Z4} \equal \frac{5}{8}R$] (9.5,0)
       to[short,-o] (10.5,0)
;
\draw[color=red] (3.75,0.0) ellipse (4.0 and 0.5);
\end{schemat}
\begin{align*}
R_{Z5}&=R+R_{Z4}\\
R_{Z5}&=R+\frac{5}{8}R=\\
&=\frac{8}{8}R+\frac{5}{8}R=\\
&=\frac{13}{8}R\\
\end{align*}
\begin{schemat}
\label{schemat:01:00:kw:J}
\draw
 (0,0) to[R,l_=$R_{Z5} \equal \frac{13}{8}R$,o-o] (10.5,0) 
;
\end{schemat}
Tak więc opór zastępczy obwodu przedstawionego na schemacie \ref{schemat:01:00:kw:Z} równa się $\frac{13}{8}R$
\end{task}
\begin{task}
Wyznacz impedancje zastępczą $Z_{AB}$ poniższego układu pomiędzy zaciskami A i B. Podaj wzór na impedancje zastępczą oraz jego wartość. Przyjmij że $Z=3+j\,\Omega$
\begin{schemat}
\label{schemat:01:01:kw:Z}
\draw
 node[ocirc,label=A] (A) at (0, 0) {}
 node[ocirc,label=below:B] (B) at (0,-3) {}
 node[circ,label=45:C] (C) at (6, 0) {}
 node[circ,label=below:D] (D) at (6,-3) {}
 
 (A) to[Z,l_=$Z_1$,o-*] (3,0)
     to[Z,l_=$Z_5$,*-*] (C)
     to[Z,l_=$Z_9$,*-*] (9,0)
     to[Z,l_=$Z_{11}$,*-*] (9,-3)     
     to[Z,l_=$Z_{10}$,*-*] (D)  
     to[Z,l_=$Z_8$,*-*] (C)
 (B) to[Z,l_=$Z_2$,o-*] (3,-3)
     to[Z,l=$Z_4$,*-] (3,-1.5)
     to[Z,l=$Z_5$] (3,0)
 (3,-3) to[short,-*] (3.5,-3)
        to[short] (3.5,-2.5)
        to[Z,l_=$Z_6$] (5.5,-2.5)
        to[short] (5.5,-3.5)        
        to[Z,l=$Z_7$] (3.5,-3.5)        
        to[short] (3.5,-3)  
 (5.5,-3) to[short,*-*] (D)
 (D) to[short] (9,0)
     to[short,-*] (10,1)
     to[Z,l=$Z_{12}$] (10,-3)
     to[short] (9,-3)
 (C) to[short] (6,1)
     to[short] (10,1)
;
\end{schemat}

\subsubsection{Rozwiązanie}
Analizę układu należy rozpocząć od obserwacji iż zaznaczone na schemacie \ref{schemat:01:00:kw:A} impedancje $Z_6$ oraz $Z_7$ połączone są równolegle.
\begin{schemat}
\label{schemat:01:01:kw:A}
\draw
 node[ocirc,label=A] (A) at (0, 0) {}
 node[ocirc,label=below:B] (B) at (0,-3) {}
 node[circ,label=45:C] (C) at (6, 0) {}
 node[circ,label=below:D] (D) at (6,-3) {}
 
 (A) to[Z,l_=$Z_1$,o-*] (3,0)
     to[Z,l_=$Z_5$,*-*] (C)
     to[Z,l_=$Z_9$,*-*] (9,0)
     to[Z,l_=$Z_{11}$,*-*] (9,-3)     
     to[Z,l_=$Z_{10}$,*-*] (D)  
     to[Z,l_=$Z_8$,*-*] (C)
 (B) to[Z,l_=$Z_2$,o-*] (3,-3)
     to[Z,l=$Z_4$,*-] (3,-1.5)
     to[Z,l=$Z_5$] (3,0)
 (3,-3) to[short,-*] (3.5,-3)
        to[short] (3.5,-2.5)
        to[Z,l_=$Z_6$] (5.5,-2.5)
        to[short] (5.5,-3.5)        
        to[Z,l=$Z_7$] (3.5,-3.5)        
        to[short] (3.5,-3)  
 (5.5,-3) to[short,*-*] (D)
 (D) to[short] (9,0)
     to[short,-*] (10,1)
     to[Z,l=$Z_{12}$] (10,-3)
     to[short] (9,-3)
 (C) to[short] (6,1)
     to[short] (10,1)
;
\draw[color=red] (4.5,-3) ellipse (1 and 1.0);
\end{schemat}
Impedancje te zastępujemy jedną impedancją o impedancji zastępczej $Z_{67}$
\begin{align*}
\frac{1}{Z_{67}}&=\frac{1}{Z_{6}}+\frac{1}{Z_{7}}=\\
&=\frac{Z_{7}}{Z_{6}\cdot Z_{7}}+\frac{Z_{7}}{Z_{6}\cdot Z_{7}}=\\
&=\frac{Z_{7} + Z_{6}}{Z_{6}\cdot Z_{7}}=\\
Z_{67}&=\frac{Z_{6}\cdot Z_{7}}{Z_{7} + Z_{6}}
\end{align*}

\begin{schemat}
\label{schemat:01:01:kw:B}
\draw
 node[ocirc,label=A] (A) at (0, 0) {}
 node[ocirc,label=below:B] (B) at (0,-3) {}
 node[circ,label=45:C] (C) at (6, 0) {}
 node[circ,label=below:D] (D) at (6,-3) {}
 
 (A) to[Z,l_=$Z_1$,o-*] (3,0)
     to[Z,l_=$Z_5$,*-*] (C)
     to[Z,l_=$Z_9$,*-*] (9,0)
     to[Z,l_=$Z_{11}$,*-*] (9,-3)     
     to[Z,l_=$Z_{10}$,*-*] (D)  
     to[Z,l_=$Z_8$,*-*] (C)
 (B) to[Z,l_=$Z_2$,o-*] (3,-3)
     to[Z,l=$Z_4$,*-] (3,-1.5)
     to[Z,l=$Z_5$] (3,0)
 (3,-3) to[Z,l_=$Z_{67}$] (D)
     to[short] (9,0)
     to[short,-*] (10,1)
     to[Z,l=$Z_{12}$] (10,-3)
     to[short] (9,-3)
 (C) to[short] (6,1)
     to[short] (10,1)
;
\end{schemat}
%-----------------------------------------------
Następnie należy zauważyć iż impedancje $Z_{3}$ i $Z_{4}$ są połączone szeregowo i zastąpić je jedną impedancją zastępczą o wartości $Z_{35}$
\begin{schemat}
\label{schemat:01:01:kw:C}
\draw
 node[ocirc,label=A] (A) at (0, 0) {}
 node[ocirc,label=below:B] (B) at (0,-3) {}
 node[circ,label=45:C] (C) at (6, 0) {}
 node[circ,label=below:D] (D) at (6,-3) {}
 
 (A) to[Z,l_=$Z_1$,o-*] (3,0)
     to[Z,l_=$Z_5$,*-*] (C)
     to[Z,l_=$Z_9$,*-*] (9,0)
     to[Z,l_=$Z_{11}$,*-*] (9,-3)     
     to[Z,l_=$Z_{10}$,*-*] (D)  
     to[Z,l_=$Z_8$,*-*] (C)
 (B) to[Z,l_=$Z_2$,o-*] (3,-3)
     to[Z,l=$Z_4$,*-] (3,-1.5)
     to[Z,l=$Z_5$] (3,0)
 (3,-3) to[Z,l_=$Z_{67}$] (D)
     to[short] (9,0)
     to[short,-*] (10,1)
     to[Z,l=$Z_{12}$] (10,-3)
     to[short] (9,-3)
 (C) to[short] (6,1)
     to[short] (10,1)
;
\draw[color=red] (3,-1.5) ellipse (0.5 and 2.0);
\end{schemat}

\begin{align*}
Z_{34}&=Z_{3}+Z_{4}
\end{align*}

\begin{schemat}
\label{schemat:01:01:kw:D}
\draw
 node[ocirc,label=A] (A) at (0, 0) {}
 node[ocirc,label=below:B] (B) at (0,-3) {}
 node[circ,label=45:C] (C) at (6, 0) {}
 node[circ,label=below:D] (D) at (6,-3) {}
 
 (A) to[Z,l_=$Z_1$,o-*] (3,0)
     to[Z,l_=$Z_5$,*-*] (C)
     to[Z,l_=$Z_9$,*-*] (9,0)
     to[Z,l_=$Z_{11}$,*-*] (9,-3)     
     to[Z,l_=$Z_{10}$,*-*] (D)  
     to[Z,l_=$Z_8$,*-*] (C)
 (B) to[Z,l_=$Z_2$,o-*] (3,-3)
     to[Z,l=$Z_{34}$,*-] (3,0)
 (3,-3) to[Z,l_=$Z_{67}$] (D)
     to[short] (9,0)
     to[short,-*] (10,1)
     to[Z,l=$Z_{12}$] (10,-3)
     to[short] (9,-3)
 (C) to[short] (6,1)
     to[short] (10,1)
;
\end{schemat}
%-----------------------------------------------
Następnie należy zauważyć iż impedancje $Z_{11}$ oraz $Z_{12}$ są połączone równolegle i zastąpić je jedną impedancją zastępczą o wartości $Z_{1112}$
\begin{schemat}
\label{schemat:01:01:kw:E}
\draw
 node[ocirc,label=A] (A) at (0, 0) {}
 node[ocirc,label=below:B] (B) at (0,-3) {}
 node[circ,label=45:C] (C) at (6, 0) {}
 node[circ,label=below:D] (D) at (6,-3) {}
 
 (A) to[Z,l_=$Z_1$,o-*] (3,0)
     to[Z,l_=$Z_5$,*-*] (C)
     to[Z,l_=$Z_9$,*-*] (9,0)
     to[Z,l_=$Z_{11}$,*-*] (9,-3)     
     to[Z,l_=$Z_{10}$,*-*] (D)  
     to[Z,l_=$Z_8$,*-*] (C)
 (B) to[Z,l_=$Z_2$,o-*] (3,-3)
     to[Z,l=$Z_{34}$,*-] (3,0)
 (3,-3) to[Z,l_=$Z_{67}$] (D)
     to[short] (9,0)
     to[short,-*] (10,1)
     to[Z,l=$Z_{12}$] (10,-3)
     to[short] (9,-3)
 (C) to[short] (6,1)
     to[short] (10,1)
;
\draw[color=red] (9.5,-1.5) ellipse (1.0 and 1.0);
\end{schemat}

\begin{align*}
\frac{1}{Z_{1112}}&=\frac{1}{Z_{11}}+\frac{1}{Z_{12}}=\\
&=\frac{Z_{12}}{Z_{11}+Z_{12}}R+\frac{Z_{11}}{Z_{11}+Z_{12}}=\\
&=\frac{Z_{12} \cdot Z_{11}}{Z_{11}+Z_{12}}\\
Z_{1112}&=\frac{Z_{11}+Z_{12}}{Z_{12} \cdot Z_{11}}
\end{align*}

\begin{schemat}
\label{schemat:01:01:kw:F}
\draw
 node[ocirc,label=A] (A) at (0, 0) {}
 node[ocirc,label=below:B] (B) at (0,-3) {}
 node[circ,label=45:C] (C) at (6, 0) {}
 node[circ,label=below:D] (D) at (6,-3) {}
 
 (A) to[Z,l_=$Z_1$,o-*] (3,0)
     to[Z,l_=$Z_5$,*-*] (C)
     to[Z,l_=$Z_9$,*-*] (9,0)
 (9.5,-3) to[Z,l_=$Z_{10}$,-*] (D)  
     to[Z,l_=$Z_8$,*-*] (C)
 (B) to[Z,l_=$Z_2$,o-*] (3,-3)
     to[Z,l=$Z_{34}$,*-] (3,0)
 (3,-3) to[Z,l_=$Z_{67}$] (D)
     to[short] (9,0)
     to[short] (10,1)
 (9.5,0.5) to[Z,l_=$Z_{1112}$,*-*] (9.5,-3) 
 (C) to[short] (6,1)
     to[short] (10,1)
;
\end{schemat}
%-----------------------------------------------
Następnie należy zauważyć iż impedancja $Z_9$ jest połączona równolegle ze zwarciem. I można ją zastąpić jednym zwarciem. 
\begin{schemat}
\label{schemat:01:01:kw:G}
\draw
 node[ocirc,label=A] (A) at (0, 0) {}
 node[ocirc,label=below:B] (B) at (0,-3) {}
 node[circ,label=45:C] (C) at (6, 0) {}
 node[circ,label=below:D] (D) at (6,-3) {}
 
 (A) to[Z,l_=$Z_1$,o-*] (3,0)
     to[Z,l_=$Z_5$,*-*] (C)
     to[Z,l_=$Z_9$,*-*] (9,0)
 (9.5,-3) to[Z,l_=$Z_{10}$,-*] (D)  
     to[Z,l_=$Z_8$,*-*] (C)
 (B) to[Z,l_=$Z_2$,o-*] (3,-3)
     to[Z,l=$Z_{34}$,*-] (3,0)
 (3,-3) to[Z,l_=$Z_{67}$] (D)
     to[short] (9,0)
     to[short] (10,1)
 (9.5,0.5) to[Z,l_=$Z_{1112}$,*-*] (9.5,-3) 
 (C) to[short] (6,1)
     to[short] (10,1)
;
\draw[color=red] (7.5,0.5) ellipse (1.0 and 1.0);
\end{schemat}

\begin{align*}
\frac{1}{Z_{90}}&=\frac{1}{Z_9}+\frac{1}{0}=\\
&=\frac{0}{Z_9 \cdot 0}+\frac{Z_9}{Z_9 \cdot 0}=\\
&=\frac{0 + Z_9}{Z_9 \cdot 0}\\
Z_{90}&=\frac{Z_9 \cdot 0}{0 + Z_9}=\\
&=\frac{0}{Z_9}=0
\end{align*}
%
\begin{schemat}
\label{schemat:01:01:kw:H}
\draw
 node[ocirc,label=A] (A) at (0, 0) {}
 node[ocirc,label=below:B] (B) at (0,-3) {}
 node[circ,label=45:C] (C) at (6, 0) {}
 node[circ,label=below:D] (D) at (6,-3) {}
 
 (A) to[Z,l_=$Z_1$,o-*] (3,0)
     to[Z,l_=$Z_5$,*-*] (C)
     to[short] (6.0,0.5)
     to[short] (9.5,0.5)
 (9.5,-3) to[Z,l_=$Z_{10}$,-*] (D)  
     to[Z,l_=$Z_8$,*-*] (C)
 (B) to[Z,l_=$Z_2$,o-*] (3,-3)
     to[Z,l=$Z_{34}$,*-] (3,0)
 (3,-3) to[Z,l_=$Z_{67}$] (D)
     to[short] (9,0)
     to[short] (9.5,0.5) 
     to[Z,l_=$Z_{1112}$,*-*] (9.5,-3) 
;
\end{schemat}
%-----------------------------------------------
Następnie należy zauważyć iż impedancje $Z_{10}$ oraz $Z_{1112}$ jest połączona szeregowo. I można ją zastąpić jedną impedancją zastępczą o wartości $Z_{101112}$. 
\begin{schemat}
\label{schemat:01:01:kw:I}
\draw
 node[ocirc,label=A] (A) at (0, 0) {}
 node[ocirc,label=below:B] (B) at (0,-3) {}
 node[circ,label=45:C] (C) at (6, 0) {}
 node[circ,label=below:D] (D) at (6,-3) {}
 
 (A) to[Z,l_=$Z_1$,o-*] (3,0)
     to[Z,l_=$Z_5$,*-*] (C)
     to[short] (6.0,0.5)
     to[short] (9.5,0.5)
 (9.5,-3) to[Z,l_=$Z_{10}$,-*] (D)  
     to[Z,l_=$Z_8$,*-*] (C)
 (B) to[Z,l_=$Z_2$,o-*] (3,-3)
     to[Z,l=$Z_{34}$,*-] (3,0)
 (3,-3) to[Z,l_=$Z_{67}$] (D)
     to[short] (9,0)
     to[short] (9.5,0.5) 
     to[Z,l_=$Z_{1112}$,*-*] (9.5,-3) 
;
\draw[color=red,rotate around={-45:(8.5,-2.0)}] (8.5,-2.0) ellipse (1.0 and 2.0);
\end{schemat}

\begin{align*}
Z_{101112}&=Z_{10}+Z_{1112}\\
\end{align*}

Wstawiając obliczoną wcześniej wartość impedancji $Z_{1112}$

\begin{align*}
Z_{101112}&=Z_{10}+Z_{1112}=\\
&=Z_{10}+\frac{Z_{11}+Z_{12}}{Z_{12} \cdot Z_{11}}=\\
&=\frac{Z_{10} \cdot \left( Z_{12} \cdot Z_{11} \right)}{Z_{12} \cdot Z_{11}}+\frac{Z_{11}+Z_{12}}{Z_{12} \cdot Z_{11}}=\\
&=\frac{Z_{10} \cdot \left( Z_{12} \cdot Z_{11} \right)+ Z_{11} + Z_{12}}{Z_{12} \cdot Z_{11}}
\end{align*}

\begin{schemat}
\label{schemat:01:01:kw:J}
\draw
 node[ocirc,label=A] (A) at (0, 0) {}
 node[ocirc,label=below:B] (B) at (0,-3) {}
 node[circ,label=45:C] (C) at (6, 0) {}
 node[circ,label=below:D] (D) at (6,-3) {}
 
 (A) to[Z,l_=$Z_1$,o-*] (3,0)
     to[Z,l_=$Z_5$,*-*] (C)
     to[short] (6.0,0.5)
     to[short] (9.5,0.5)
 (D) to[Z,l_=$Z_8$,*-*] (C)
 (B) to[Z,l_=$Z_2$,o-*] (3,-3)
     to[Z,l=$Z_{34}$,*-] (3,0)
 (3,-3) to[Z,l_=$Z_{67}$] (D)
     to[short] (9,0)
     to[short] (9.5,0.5) 
     to[short] (9.5,-0.5)
     to[Z,l_=$Z_{101112}$] (7.0,-3)
     to[short] (D) 
;
\end{schemat}
%-----------------------------------------------
Następnie należy zauważyć iż impedancja $Z_{101112}$ połączona jest równolegle ze zwarciem. A więc można ją zastąpić zwarciem. 
\begin{schemat}
\label{schemat:01:01:kw:K}
\draw
 node[ocirc,label=A] (A) at (0, 0) {}
 node[ocirc,label=below:B] (B) at (0,-3) {}
 node[circ,label=45:C] (C) at (6, 0) {}
 node[circ,label=below:D] (D) at (6,-3) {}
 
 (A) to[Z,l_=$Z_1$,o-*] (3,0)
     to[Z,l_=$Z_5$,*-*] (C)
     to[short] (6.0,0.5)
     to[short] (9.5,0.5)
 (D) to[Z,l_=$Z_8$,*-*] (C)
 (B) to[Z,l_=$Z_2$,o-*] (3,-3)
     to[Z,l=$Z_{34}$,*-] (3,0)
 (3,-3) to[Z,l_=$Z_{67}$] (D)
     to[short] (9,0)
     to[short] (9.5,0.5) 
     to[short] (9.5,-0.5)
     to[Z,l_=$Z_{101112}$] (7.0,-3)
     to[short] (D) 
;
\draw[color=red,rotate around={45:(8.0,-1.5)}] (8.0,-1.5) ellipse (1.0 and 1.0);
\end{schemat}

\begin{align*}
\frac{1}{Z_{1011120}}&=\frac{1}{0} + \frac{1}{Z_{101112}}=\\
&=\frac{Z_{101112}}{0 \cdot Z_{101112}} + \frac{0}{0 \cdot Z_{101112}}=\\
&=\frac{Z_{101112}+0}{0 \cdot Z_{101112}}\\
Z_{1011120} &= \frac{0 \cdot Z_{101112}}{Z_{101112}+0}=0\\
\end{align*}

\begin{schemat}
\label{schemat:01:01:kw:L}
\draw
 node[ocirc,label=A] (A) at (0, 0) {}
 node[ocirc,label=below:B] (B) at (0,-3) {}
 node[circ,label=45:C] (C) at (6, 0) {}
 node[circ,label=below:D] (D) at (6,-3) {}
 
 (A) to[Z,l_=$Z_1$,o-*] (3,0)
     to[Z,l_=$Z_5$,*-*] (C)
     to[short] (6.0,0.5)
     to[short] (9.5,0.5)
 (D) to[Z,l_=$Z_8$,*-*] (C)
 (B) to[Z,l_=$Z_2$,o-*] (3,-3)
     to[Z,l=$Z_{34}$,*-] (3,0)
 (3,-3) to[Z,l_=$Z_{67}$] (D)
 (9.5,0.5) to[short] (9.5,0.0)
     to[short] (6.5,-3)
     to[short] (D) 
;
\end{schemat}
%-----------------------------------------------
Następnie należy zauważyć iż impedancja $Z_{8}$ połączona jest równolegle ze zwarciem. A więc można ją zastąpić zwarciem. 
\begin{schemat}
\label{schemat:01:01:kw:M}
\draw
 node[ocirc,label=A] (A) at (0, 0) {}
 node[ocirc,label=below:B] (B) at (0,-3) {}
 node[circ,label=45:C] (C) at (6, 0) {}
 node[circ,label=below:D] (D) at (6,-3) {}
 
 (A) to[Z,l_=$Z_1$,o-*] (3,0)
     to[Z,l_=$Z_5$,*-*] (C)
     to[short] (6.0,0.5)
     to[short] (9.5,0.5)
 (D) to[Z,l_=$Z_8$,*-*] (C)
 (B) to[Z,l_=$Z_2$,o-*] (3,-3)
     to[Z,l=$Z_{34}$,*-] (3,0)
 (3,-3) to[Z,l_=$Z_{67}$] (D)
 (9.5,0.5) to[short] (9.5,0.0)
     to[short] (6.5,-3)
     to[short] (D) 
;
\draw[color=red] (7.0,-1.5) ellipse (2.0 and 1.0);
\end{schemat}

\begin{align*}
\frac{1}{Z_{80}}&=\frac{1}{0} + \frac{1}{Z_{8}}=\\
&=\frac{Z_{8}}{0 \cdot Z_{8}} + \frac{0}{0 \cdot Z_{8}}=\\
&=\frac{Z_{8}+0}{0 \cdot Z_{8}}\\
Z_{80} &= \frac{0 \cdot Z_{8}}{Z_{8}+0}=0\\
\end{align*}

\begin{schemat}
\label{schemat:01:01:kw:N}
\draw
 node[ocirc,label=A] (A) at (0, 0) {}
 node[ocirc,label=below:B] (B) at (0,-3) {}
 node[circ,label=45:C] (C) at (6, 0) {}
 node[circ,label=below:D] (D) at (6,-3) {}
 
 (A) to[Z,l_=$Z_1$,o-*] (3,0)
     to[Z,l_=$Z_5$,*-*] (C)
 (D) to[short] (C)
 (B) to[Z,l_=$Z_2$,o-*] (3,-3)
     to[Z,l=$Z_{34}$,*-] (3,0)
 (3,-3) to[Z,l_=$Z_{67}$] (D)
;
\end{schemat}
%-----------------------------------------------
Następnie należy zauważyć iż impedancje $Z_{5}$ i $Z_{67}$ są połączone szeregowo. A więc można ją zastąpić jedną impedancją zastępczą o wartości $Z_{567}$. 
\begin{schemat}
\label{schemat:01:01:kw:O}
\draw
 node[ocirc,label=A] (A) at (0, 0) {}
 node[ocirc,label=below:B] (B) at (0,-3) {}
 node[circ,label=45:C] (C) at (6, 0) {}
 node[circ,label=below:D] (D) at (6,-3) {}
 
 (A) to[Z,l_=$Z_1$,o-*] (3,0)
     to[Z,l_=$Z_5$,*-*] (C)
 (D) to[short] (C)
 (B) to[Z,l_=$Z_2$,o-*] (3,-3)
     to[Z,l=$Z_{34}$,*-] (3,0)
 (3,-3) to[Z,l_=$Z_{67}$] (D)
;
\draw[color=red] (4.5,-1.5) ellipse (1.0 and 3.0);
\end{schemat}

\begin{align*}
Z_{567} &= Z_{5}+Z_{67}\\
\end{align*}

Wstawiając wcześniej obliczoną wartość impedancji $Z_{67}$

\begin{align*}
Z_{567} &= Z_{5}+Z_{67}=\\
&=Z_{5} + \frac{Z_{6}\cdot Z_{7}}{Z_{7} + Z_{6}} = \\
&=\frac{Z_{5} \cdot \left( Z_{7} + Z_{6} \right)}{Z_{7} + Z_{6}} + \frac{Z_{6}\cdot Z_{7}}{Z_{7} + Z_{6}} = \\
&=\frac{Z_{5} \cdot \left( Z_{7} + Z_{6} \right)+Z_{6}\cdot Z_{7} }{Z_{7} + Z_{6}}
\end{align*}


\begin{schemat}
\label{schemat:01:01:kw:P}
\draw
 node[ocirc,label=A] (A) at (0, 0) {}
 node[ocirc,label=below:B] (B) at (0,-3) {}

 (A) to[Z,l_=$Z_1$,o-*] (3,0)
     to[short,*-] (4,0)
     to[Z,l=$Z_{567}$] (4,-3)
     to[short] (3,-3)
 (B) to[Z,l_=$Z_2$,o-*] (3,-3)
     to[Z,l=$Z_{34}$,*-] (3,0)
;
\end{schemat}
%-----------------------------------------------
Następnie należy zauważyć iż impedancje $Z_{34}$ i $Z_{567}$ są połączone równolegle. A więc można ją zastąpić jedną impedancją zastępczą o wartości $Z_{34567}$. 
\begin{schemat}
\label{schemat:01:01:kw:R}
\draw
 node[ocirc,label=A] (A) at (0, 0) {}
 node[ocirc,label=below:B] (B) at (0,-3) {}

 (A) to[Z,l_=$Z_1$,o-*] (3,0)
     to[short,*-] (4,0)
     to[Z,l=$Z_{567}$] (4,-3)
     to[short] (3,-3)
 (B) to[Z,l_=$Z_2$,o-*] (3,-3)
     to[Z,l=$Z_{34}$,*-] (3,0)
;
\draw[color=red] (3.5,-1.5) ellipse (2.0 and 1.0);
\end{schemat}

\begin{align*}
\frac{1}{Z_{34567}} &= \frac{1}{Z_{34}}+\frac{1}{Z_{567}}\\
\end{align*}

Wstawiając wcześniej obliczone wartości impedancji $Z_{34}$ oraz $Z_{567}$

\begin{align*}
\frac{1}{Z_{34567}} &= \frac{1}{Z_{34}}+\frac{1}{Z_{567}}=\\
&=\frac{1}{Z_{3}+Z_{4}}+\frac{1}{\frac{Z_{5} \cdot \left( Z_{7} + Z_{6} \right)+Z_{6}\cdot Z_{7} }{Z_{7} + Z_{6}}}=\\
&=\frac{1}{Z_{3}+Z_{4}}+\frac{Z_{7} + Z_{6}}{Z_{5} \cdot \left( Z_{7} + Z_{6} \right)+Z_{6}\cdot Z_{7} }=\\
&=\frac{Z_{5} \cdot \left( Z_{7} + Z_{6} \right)+Z_{6}\cdot Z_{7}}{\left( Z_{3}+Z_{4} \right) \cdot \left( Z_{5} \cdot \left( Z_{7} + Z_{6} \right)+Z_{6}\cdot Z_{7} \right)} + \frac{Z_{3}+Z_{4}}{\left( Z_{3}+Z_{4} \right) \cdot \left( Z_{5} \cdot \left( Z_{7} + Z_{6} \right)+Z_{6}\cdot Z_{7} \right)}=\\
&=\frac{Z_{5} \cdot \left( Z_{7} + Z_{6} \right)+Z_{6}\cdot Z_{7}+Z_{3}+Z_{4}}{\left( Z_{3}+Z_{4} \right) \cdot \left( Z_{5} \cdot \left( Z_{7} + Z_{6} \right)+Z_{6}\cdot Z_{7} \right)}\\
Z_{34567}&=\frac{\left( Z_{3}+Z_{4} \right) \cdot \left( Z_{5} \cdot \left( Z_{7} + Z_{6} \right)+Z_{6}\cdot Z_{7} \right)}{Z_{5} \cdot \left( Z_{7} + Z_{6} \right)+Z_{6}\cdot Z_{7}+Z_{3}+Z_{4}}
\end{align*}

\begin{schemat}
\label{schemat:01:01:kw:S}
\draw
 node[ocirc,label=A] (A) at (0, 0) {}
 node[ocirc,label=below:B] (B) at (0,-3) {}

 (A) to[Z,l_=$Z_1$,o-] (3,0)
     to[short] (3.5,0)
     to[Z,l=$Z_{34567}$] (3.5,-3)
     to[short] (3,-3)
     to[Z,l_=$Z_2$,-o] (B)
;
\end{schemat}
%-----------------------------------------------
Ostatecznie należy zauważyć iż impedancje $Z_1$, $Z_{34567}$ oraz $Z_{2}$ są połączone szeregowo i zastąpić je jedną impedancją zastępczą o wartości $R_{1234567}$
\begin{schemat}
\label{schemat:01:01:kw:T}
\draw
 node[ocirc,label=A] (A) at (0, 0) {}
 node[ocirc,label=below:B] (B) at (0,-3) {}

 (A) to[Z,l_=$Z_1$,o-] (3,0)
     to[short] (3.5,0)
     to[Z,l=$Z_{34567}$] (3.5,-3)
     to[short] (3,-3)
     to[Z,l_=$Z_2$,-o] (B)
;
\draw[color=red] (2.5,-1.5) ellipse (3.0 and 3.0);
\end{schemat}

\begin{align*}
Z_{1234567}&=Z_{1}+Z_{34567}+Z_{2}\\
\end{align*}

Podstawiając obliczoną wcześniej wartość impedancji $Z_{34567}$

\begin{align*}
Z_{1234567}&=Z_{1}+Z_{34567}+Z_{2}=\\
&=Z_{1} + \frac{\left( Z_{3}+Z_{4} \right) \cdot \left( Z_{5} \cdot \left( Z_{7} + Z_{6} \right)+Z_{6}\cdot Z_{7} \right)}{Z_{5} \cdot \left( Z_{7} + Z_{6} \right)+Z_{6}\cdot Z_{7}+Z_{3}+Z_{4}} + Z_{2}
\end{align*}

\begin{schemat}
\label{schemat:01:01:kw:U}
\draw
 node[ocirc,label=A] (A) at (0, 0) {}
 node[ocirc,label=below:B] (B) at (0,-3) {}

 (A) to[short] (0.5,0)
     to[Z,l_=$Z_{1234567}$] (0.5,-3)
     to[short] (B)
;
\end{schemat}
Tak więc impedancja zastępcza obwodu przedstawionego na schemacie \ref{schemat:01:01:kw:Z} równa się $Z_{1234567}$
\end{task}
\begin{task}
Wyznacz impedancje zastępczą $Z_{AB}$ poniższego układu pomiędzy zaciskami A i B. Podaj wzór na impedancje zastępczą oraz jego wartość. Przyjmij że $Z=2+j\,\Omega$
\begin{schemat}
\label{schemat:01:02:kw:Z}
\draw
 node[ocirc,label=A] (A) at (0, 0) {}
 node[ocirc,label=below:B] (B) at (0,-6) {}

 (A) to[short] (1,0)
     to[Z,l_=$Z_4$,*-*] (4,0)
     to[short] (5,0)
     to[short] (5,-3)
     to[short] (5,-6)
     to[short] (B)
 (1,0) to[Z,l_=$Z_1$,*-*] (1,-3)
       to[Z,l_=$Z_2$,*-*] (1,-6)
 (4,0) to[Z,l_=$Z_5$,*-*] (4,-3)
       to[Z,l_=$Z_6$,*-*] (4,-6)  
 (1,-3) to[Z,l_=$Z_3$,*-*] (4,-3)     
;
\end{schemat}

\subsubsection{Rozwiązanie}
Analizę układu należy rozpocząć od obserwacji iż zaznaczone na schemacie \ref{schemat:01:02:kw:A} impedancje $Z_5$ oraz $Z_6$ połączone są równolegle.
\begin{schemat}
\label{schemat:01:02:kw:A}
\draw
 node[ocirc,label=A] (A) at (0, 0) {}
 node[ocirc,label=below:B] (B) at (0,-6) {}

 (A) to[short] (1,0)
     to[Z,l_=$Z_4$,*-*] (4,0)
     to[short] (5,0)
     to[short] (5,-3)
     to[short] (5,-6)
     to[short] (B)
 (1,0) to[Z,l_=$Z_1$,*-*] (1,-3)
       to[Z,l_=$Z_2$,*-*] (1,-6)
 (4,0) to[Z,l_=$Z_5$,*-*] (4,-3)
       to[Z,l_=$Z_6$,*-*] (4,-6)  
 (1,-3) to[Z,l_=$Z_3$,*-*] (4,-3)    
;
\draw[color=red] (4.0,-3.0) ellipse (1.0 and 3.0);
\end{schemat}
Impedancje tą zastępujemy jedną impedancją zastępczą o wartości $Z_{56}$
\begin{align*}
\frac{1}{Z_{56}}&=\frac{1}{Z_{5}}+\frac{1}{Z_{6}}=\\
&=\frac{Z_{6}}{Z_{5}\cdot Z_{6}}+\frac{Z_{5}}{Z_{5}\cdot Z_{6}}=\\
&=\frac{Z_{6} + Z_{5}}{Z_{5}\cdot Z_{6}}\\
Z_{56}&=\frac{Z_{5}\cdot Z_{6}}{Z_{6} + Z_{5}}
\end{align*}

\begin{schemat}
\label{schemat:01:02:kw:B}
\draw
 node[ocirc,label=A] (A) at (0, 0) {}
 node[ocirc,label=below:B] (B) at (0,-6) {}

 (A) to[short] (1,0)
     to[Z,l_=$Z_4$,*-*] (4,0)
     to[short] (5,0)
     to[short] (5,-3)
     to[short] (5,-6)
     to[short] (B)
 (1,0) to[Z,l_=$Z_1$,*-*] (1,-3)
       to[Z,l_=$Z_2$,*-*] (1,-6)
 (4,0) to[Z,l_=$Z_{56}$,*-] (4,-3)
       to[Z,l_=$Z_3$,-*]  (1,-3)    
;
\end{schemat}
%-----------------------------------------------
Następnie należy zauważyć iż impedancje $Z_{3}$ i $Z_{56}$ są połączone szeregowo i zastąpić je jedną impedancją zastępczą o wartości $Z_{356}$
\begin{schemat}
\label{schemat:01:02:kw:C}
\draw
 node[ocirc,label=A] (A) at (0, 0) {}
 node[ocirc,label=below:B] (B) at (0,-6) {}

 (A) to[short] (1,0)
     to[Z,l_=$Z_4$,*-*] (4,0)
     to[short] (5,0)
     to[short] (5,-3)
     to[short] (5,-6)
     to[short] (B)
 (1,0) to[Z,l_=$Z_1$,*-*] (1,-3)
       to[Z,l_=$Z_2$,*-*] (1,-6)
 (4,0) to[Z,l_=$Z_{56}$,*-] (4,-3)
       to[Z,l_=$Z_3$,-*]  (1,-3)    
;
\draw[color=red,rotate around={-45:(3,-2.0)}] (3,-2.0) ellipse (1.0 and 2.0);
\end{schemat}

\begin{align*}
Z_{356}&=Z_{3}+Z_{56}=\\
&=Z_{3}+\frac{Z_{5}\cdot Z_{6}}{Z_{6} + Z_{5}}
\end{align*}

\begin{schemat}
\label{schemat:01:02:kw:D}
\draw
 node[ocirc,label=A] (A) at (0, 0) {}
 node[ocirc,label=below:B] (B) at (0,-6) {}

 (A) to[short] (1,0)
     to[Z,l_=$Z_4$,*-*] (4,0)
     to[short] (5,0)
     to[short] (5,-3)
     to[short] (5,-6)
     to[short] (B)
 (1,0) to[Z,l_=$Z_1$,*-*] (1,-3)
       to[Z,l_=$Z_2$,*-*] (1,-6)
 (4,0) to[Z,l_=$Z_{356}$,*-*] (1,-3)    
;
\end{schemat}
%-----------------------------------------------
Następnie należy przerysować układ delikatnie wyprostowując impedancje $Z_{356}$. Można wtedy zauważyć iż impedancje $Z_{2}$ oraz $Z_{356}$ są połączone równolegle i zastąpić je jedną impedancją zastępczą o wartości $Z_{2356}$
\begin{schemat}
\label{schemat:01:02:kw:E}
\draw
 node[ocirc,label=A] (A) at (0, 0) {}
 node[ocirc,label=below:B] (B) at (0,-6) {}

 (A) to[short] (1,0)
     to[Z,l_=$Z_4$,*-] (4,0)
     to[short] (5,0)
     to[short] (5,-3)
     to[short] (5,-6)
     to[short] (B)
 (1,0) to[Z,l_=$Z_1$,*-*] (1,-3)
       to[Z,l_=$Z_2$,*-*] (1,-6)
 (5,-3) to[Z,l_=$Z_{356}$,*-*] (1,-3)    
;
\draw[color=red,rotate around={-45:(2,-4.0)}] (2,-4.0) ellipse (1.0 and 2.0);
\end{schemat}

\begin{align*}
\frac{1}{Z_{2356}}&=\frac{1}{Z_{2}}+\frac{1}{Z_{356}}=\\
&=\frac{Z_{356}}{Z_{2} \cdot Z_{356}}+\frac{Z_{2}}{Z_{2} \cdot Z_{356}}=\\
&=\frac{Z_{356} + Z_{2}}{Z_{2} \cdot Z_{356}}\\
Z_{2356}&=\frac{Z_{2} \cdot Z_{356}}{Z_{356} + Z_{2}}
\end{align*}

Wstawiają wcześniej wyznaczoną wartości impedancji $Z_{356}$ otrzymujemy

\begin{align*}
Z_{2356}&=\frac{Z_{2} \cdot Z_{356}}{Z_{356} + Z_{2}}=\\
&=\frac{Z_{2} \cdot \left( Z_{3}+\frac{Z_{5}\cdot Z_{6}}{Z_{6} + Z_{5}} \right)}{Z_{3}+\frac{Z_{5}\cdot Z_{6}}{Z_{6} + Z_{5}} + Z_{2}}=\\
&=\frac{Z_{2} \cdot Z_{3}+Z_{2} \cdot \frac{Z_{5}\cdot Z_{6}}{Z_{6} + Z_{5}}}{Z_{3}+\frac{Z_{5}\cdot Z_{6}}{Z_{6} + Z_{5}} + Z_{2}}=\\
&=\frac{Z_2 \cdot Z_3 \cdot Z_5 + Z_2 \cdot \left( Z_3+Z_5 \right) \cdot Z_6}{\left( Z_2+Z_3\right) \cdot Z_5 + \left( Z_2+Z_3+Z_5 \right) \cdot Z_6}
\end{align*}

\begin{schemat}
\label{schemat:01:02:kw:F}
\draw
 node[ocirc,label=A] (A) at (0, 0) {}
 node[ocirc,label=below:B] (B) at (0,-6) {}

 (A) to[short] (1,0)
     to[Z,l_=$Z_4$,*-] (4,0)
     to[short] (5,0)
     to[short] (5,-3)
     to[short] (5,-6)
     to[short] (B)
 (1,0) to[Z,l_=$Z_1$,*-] (1,-3)
       to[Z,l_=$Z_{2356}$,-*] (1,-6)    
;
\end{schemat}
%-----------------------------------------------
Następnie należy zauważyć iż impedancje $Z_1$ oraz $Z_{2356}$ są połączone szeregowo. I można je zastąpić jedną impedancją zastępczą o wartości $Z_{12356}$. 
\begin{schemat}
\label{schemat:01:02:kw:G}
\draw
 node[ocirc,label=A] (A) at (0, 0) {}
 node[ocirc,label=below:B] (B) at (0,-6) {}

 (A) to[short] (1,0)
     to[Z,l_=$Z_4$,*-] (4,0)
     to[short] (5,0)
     to[short] (5,-3)
     to[short] (5,-6)
     to[short] (B)
 (1,0) to[Z,l_=$Z_1$,*-] (1,-3)
       to[Z,l_=$Z_{2356}$,-*] (1,-6)    
;
\draw[color=red] (1,-3.0) ellipse (1.0 and 2.5);
\end{schemat}

\begin{align*}
Z_{12356}&=Z_{1} + Z_{2356}\\
\end{align*}

Wstawiają wcześniej wyznaczoną wartości impedancji $Z_{2356}$ otrzymujemy

\begin{align*}
Z_{12356}&=Z_{1} + Z_{2356}=\\
&=Z_{1} + \frac{Z_2 \cdot Z_3 \cdot Z_5 + Z_2 \cdot \left( Z_3+Z_5 \right) \cdot Z_6}{\left( Z_2+Z_3\right) \cdot Z_5 + \left( Z_2+Z_3+Z_5 \right) \cdot Z_6}
\end{align*}
%
\begin{schemat}
\label{schemat:01:02:kw:H}
\draw
 node[ocirc,label=A] (A) at (0, 0) {}
 node[ocirc,label=below:B] (B) at (0,-6) {}

 (A) to[short] (1,0)
     to[Z,l_=$Z_4$,*-] (4,0)
     to[short] (5,0)
     to[short] (5,-3)
     to[short] (5,-6)
     to[short] (B)
 (1,0) to[Z,l_=$Z_{12356}$,*-*] (1,-6)    
;
\end{schemat}
%-----------------------------------------------
Następnie należy zauważyć iż impedancje $Z_{4}$ oraz $Z_{12356}$ są połączona równolegle. I można ją zastąpić jedną impedancją zastępczą o wartości $Z_{123456}$. 
\begin{schemat}
\label{schemat:01:02:kw:I}
\draw
 node[ocirc,label=A] (A) at (0, 0) {}
 node[ocirc,label=below:B] (B) at (0,-6) {}

 (A) to[short] (1,0)
     to[Z,l_=$Z_4$,*-] (4,0)
     to[short] (5,0)
     to[short] (5,-3)
     to[short] (5,-6)
     to[short] (B)
 (1,0) to[Z,l_=$Z_{12356}$,*-*] (1,-6)    
;
\draw[color=red,rotate around={-30:(2.0,-2.0)}] (2.0,-2.0) ellipse (1.0 and 2.0);
\end{schemat}

\begin{align*}
\frac{1}{Z_{123456}}&=\frac{1}{Z_{4}}+\frac{1}{Z_{12356}}
\end{align*}

Wstawiając obliczoną wcześniej wartość impedancji $Z_{12356}$

\begin{align*}
\frac{1}{Z_{123456}}&=\frac{1}{Z_{4}}+\frac{1}{Z_{12356}}\\
Z_{123456}&=\frac{Z_1 \cdot Z_4 \cdot \left( \left(Z_2 + Z_3\right) \cdot Z_5 + \left( Z_2 + Z_3 + Z_5 \right) \cdot Z_6 \right) + 
   Z_2 \cdot Z_4 \cdot \left( Z_5 \cdot Z_6 + Z_3 \cdot \left( Z_5 + Z_6 \right) \right)}{Z_1 \cdot \left( Z_2 + Z_3 \right) \cdot Z_5 + Z_3 \cdot Z_4 \cdot Z_5 + 
   Z_2 \cdot \left( Z_3 + Z_4 \right) \cdot Z_5 + Z_4 \cdot \left( Z_3 + Z_5 \right) \cdot Z_6 + Z_1 \cdot \left( Z_2 + Z_3 + Z_5 \right) \cdot Z_6 + 
   Z_2 \cdot \left( Z_3 + Z_4 + Z_5 \right) \cdot Z_6}
\end{align*}

\begin{schemat}
\label{schemat:01:02:kw:J}
\draw
 node[ocirc,label=A] (A) at (0, 0) {}
 node[ocirc,label=below:B] (B) at (0,-6) {}

 (A) to[short] (1,0)
     to[Z,l_=$Z_{123456}$] (1,-6)         
     to[short] (B)
;
\end{schemat}
Tak więc impedancja zastępcza obwodu przedstawionego na schemacie \ref{schemat:01:02:kw:Z} równa się $Z_{123456}$
\end{task}
\begin{task}
Wyznacz impedancje zastępczą $Z_{AB}$ poniższego układu pomiędzy zaciskami A i B. Podaj wzór na impedancje zastępczą oraz jego wartość. Przyjmij że $Z=2+j\,\Omega$
\begin{schemat}
\label{schemat:01:03:kw:Z}
\draw
 node[ocirc,label=A] (A) at (0, 0) {}
 node[ocirc,label=below:B] (B) at (0,-6) {}

 (A) to[short] (1,0)
     to[Z,l_=$Z_4$,*-] (4,0)
     to[Z,l_=$Z_5$] (4,-3)
     to[Z,l_=$Z_6$] (4,-6)
     to[short] (B)
 (1,0) to[Z,l_=$Z_1$,*-*] (1,-3)
       to[Z,l_=$Z_2$,*-*] (1,-6)
 (1,-3) to[Z,l_=$Z_3$,*-*] (4,-3)     
;
\end{schemat}

\subsubsection{Rozwiązanie}
Analizę układu należy rozpocząć od obserwacji iż żadne dwie impedancje nie są połączone szeregowo ani równolegle. Można natomiast zauważyć iż zaznaczone na schemacie \ref{schemat:01:03:kw:A} impedancje $Z_1$, $Z_2$ oraz $Z_3$ są połączone w gwiazdę. Można zastąpić je układem trzech impedancji $Z_A$, $Z_B$ i $Z_C$ połączonych w trójkąt.
\begin{schemat}
\label{schemat:01:03:kw:A}
\draw
node[ocirc,label=A] (A) at (0, 0) {}
node[ocirc,label=below:B] (B) at (0,-6) {}

(A) to[short] (1,0)
to[Z,l_=$Z_4$,*-] (4,0)
to[Z,l_=$Z_5$] (4,-3)
to[Z,l_=$Z_6$] (4,-6)
to[short] (B)
(1,0) to[Z,l_=$Z_1$,*-*] (1,-3)
to[Z,l_=$Z_2$,*-*] (1,-6)
(1,-3) to[Z,l_=$Z_3$,*-*] (4,-3)     
;
\draw[color=red] (1.6,-3.0) ellipse (1.8 and 2.7);
\end{schemat}

\begin{align*}
R_A=R_1+R_3+\frac{R_1 \cdot R_3}{R_2}\\
R_B=R_2+R_3+\frac{R_2 \cdot R_3}{R_1}\\
R_C=R_1+R_2+\frac{R_1 \cdot R_2}{R_3}
\end{align*}

\begin{schemat}
\label{schemat:01:03:kw:B}
\draw
node[ocirc,label=A] (A) at (0, 0) {}
node[ocirc,label=below:B] (B) at (0,-6) {}

(A)   to[short] (1,0)
      to[Z,l_=$Z_4$,*-] (4,0)
      to[Z,l_=$Z_5$] (4,-3)
      to[Z,l_=$Z_6$] (4,-6)
      to[short] (B)
(1,0) to[Z,l=$Z_A$,*-*] (4,-3)
      to[Z,l=$Z_B$,*-*] (1,-6)
      to[Z,l_=$Z_C$,*-*] (1, 0)     
;
\end{schemat}
%-----------------------------------------------
Następnie należy zauważyć iż impedancje $Z_{4}$ i $Z_{5}$ są połączone szeregowo i zastąpić je jedną impedancją zastępczą o wartości $Z_{45}$
\begin{schemat}
\label{schemat:01:03:kw:C}
\draw
node[ocirc,label=A] (A) at (0, 0) {}
node[ocirc,label=below:B] (B) at (0,-6) {}

(A)   to[short] (1,0)
to[Z,l_=$Z_4$,*-] (4,0)
to[Z,l_=$Z_5$] (4,-3)
to[Z,l_=$Z_6$] (4,-6)
to[short] (B)
(1,0) to[Z,l=$Z_A$,*-*] (4,-3)
to[Z,l=$Z_B$,*-*] (1,-6)
to[Z,l_=$Z_C$,*-*] (1, 0)     
;
\draw[color=red,rotate around={45:(3.5,-0.7)}] (3.5,-0.7) ellipse (0.8 and 2.0);
\end{schemat}

\begin{align*}
Z_{45}&=Z_{4}+Z_{5}
\end{align*}

\begin{schemat}
\label{schemat:01:03:kw:D}
\draw
node[ocirc,label=A] (A) at (0, 0) {}
node[ocirc,label=below:B] (B) at (0,-6) {}

(A)   to[short] (2.0,0)
      to[Z,l=$Z_{45}$] (4,-2.0)
      to[short] (4,-3)
      to[Z,l_=$Z_6$] (4,-6)
      to[short] (B)
(1,0) to[Z,l_=$Z_A$,*-*] (4,-3)
      to[Z,l=$Z_B$,*-*] (1,-6)
      to[Z,l_=$Z_C$,*-*] (1, 0)     
;
\end{schemat}
%-----------------------------------------------
Następnie można zauważyć iż impedancje $Z_{45}$ oraz $Z_{A}$ są połączone równolegle i zastąpić je jedną impedancją zastępczą o wartości $Z_{45A}$
\begin{schemat}
\label{schemat:01:03:kw:E}
\draw
node[ocirc,label=A] (A) at (0, 0) {}
node[ocirc,label=below:B] (B) at (0,-6) {}

(A)   to[short] (2.0,0)
to[Z,l=$Z_{45}$] (4,-2.0)
to[short] (4,-3)
to[Z,l_=$Z_6$] (4,-6)
to[short] (B)
(1,0) to[Z,l_=$Z_A$,*-*] (4,-3)
to[Z,l=$Z_B$,*-*] (1,-6)
to[Z,l_=$Z_C$,*-*] (1, 0)     
;
\draw[color=red,rotate around={-45:(3,-1.0)}] (3,-1.0) ellipse (1.0 and 1.5);
\end{schemat}

\begin{align*}
\frac{1}{Z_{45A}}&=\frac{1}{Z_{45}}+\frac{1}{Z_{A}}=\\
&=\frac{Z_{A}}{Z_{45} \cdot Z_{A}}+\frac{Z_{45}}{Z_{45} \cdot Z_{A}}=\\
&=\frac{Z_{45} + Z_{A}}{Z_{45} \cdot Z_{A}}\\
Z_{45A}&=\frac{Z_{45} \cdot Z_{A}}{Z_{45} + Z_{A}}
\end{align*}

Wstawiają wcześniej wyznaczone wartości impedancji $Z_{45}$ i $Z_{A}$ otrzymujemy

\begin{align*}
Z_{45A}&=\frac{Z_{45} \cdot Z_{A}}{Z_{45} + Z_{A}}=\\
&=\frac{\left(Z_{4}+Z_{5}\right) \cdot \left( R_1+R_3+\frac{R_1 \cdot R_3}{R_2} \right)}{Z_{4} + Z_{5} + R_1+R_3+\frac{R_1 \cdot R_3}{R_2} }
\end{align*}

\begin{schemat}
\label{schemat:01:03:kw:F}
\draw
node[ocirc,label=A] (A) at (0, 0) {}
node[ocirc,label=below:B] (B) at (0,-6) {}

(A)    to[short] (1.5,0)
       to[Z,l=$Z_{45A}$] (4,-2.5)
       to[short] (4,-3)
       to[Z,l_=$Z_6$] (4,-6)
       to[short] (B)
(4,-3) to[Z,l=$Z_B$,*-*] (1,-6)
       to[Z,l_=$Z_C$,*-*] (1, 0)     
;
\end{schemat}
%-----------------------------------------------
Następnie należy zauważyć iż impedancje $Z_B$ oraz $Z_{6}$ są połączone równolegle. I można je zastąpić jedną impedancją zastępczą o wartości $Z_{B6}$. 
\begin{schemat}
\label{schemat:01:03:kw:G}
\draw
node[ocirc,label=A] (A) at (0, 0) {}
node[ocirc,label=below:B] (B) at (0,-6) {}

(A)    to[short] (1.5,0)
       to[Z,l=$Z_{45A}$] (4,-2.5)
       to[short] (4,-3)
       to[Z,l_=$Z_6$] (4,-6)
       to[short] (B)
(4,-3) to[Z,l=$Z_B$,*-*] (1,-6)
       to[Z,l_=$Z_C$,*-*] (1, 0)     
;
\draw[color=red] (3,-4.5) ellipse (1.5 and 1.0);
\end{schemat}

\begin{align*}
\frac{1}{Z_{B6}}&=\frac{1}{Z_{B}}+\frac{1}{Z_{6}}=\\
&=\frac{Z_{6}+Z_{B}}{Z_{B} \cdot Z_{6}}\\
Z_{B6}&=\frac{Z_{B} \cdot Z_{6}}{Z_{6}+Z_{B}}
\end{align*}

Wstawiają wcześniej wyznaczoną wartości impedancji $Z_{B}$ otrzymujemy

\begin{align*}
Z_{B6}&=\frac{Z_{B} \cdot Z_{6}}{Z_{6}+Z_{B}}=\\
&=\frac{\left( R_2+R_3+\frac{R_2 \cdot R_3}{R_1} \right) \cdot Z_{6}}{Z_{6}+R_2+R_3+\frac{R_2 \cdot R_3}{R_1}}
\end{align*}
%
\begin{schemat}
\label{schemat:01:03:kw:H}
\draw
node[ocirc,label=A] (A) at (0, 0) {}
node[ocirc,label=below:B] (B) at (0,-6) {}

(A)    to[short] (1.5,0)
       to[Z,l=$Z_{45A}$] (4,-2.5)
       to[short] (4,-3.5)
       to[Z,l_=$Z_B6$] (1.5,-6)
       to[short] (B)
(1,-6) to[Z,l_=$Z_C$,*-*] (1, 0)     
;
\end{schemat}
%-----------------------------------------------
Następnie należy zauważyć iż impedancje $Z_{45A}$ oraz $Z_{B6}$ są połączona szeregowo. I można ją zastąpić jedną impedancją zastępczą o wartości $Z_{45AB6}$. 
\begin{schemat}
\label{schemat:01:03:kw:I}
\draw
node[ocirc,label=A] (A) at (0, 0) {}
node[ocirc,label=below:B] (B) at (0,-6) {}

(A)    to[short] (1.5,0)
to[Z,l=$Z_{45A}$] (4,-2.5)
to[short] (4,-3.5)
to[Z,l_=$Z_B6$] (1.5,-6)
to[short] (B)
(1,-6) to[Z,l_=$Z_C$,*-*] (1, 0)     
;
\draw[color=red] (2.8,-3.0) ellipse (1.0 and 3.0);
\end{schemat}

\begin{align*}
Z_{45AB6} = Z_{45A} + Z_{B6}
\end{align*}

Wstawiając obliczone wcześniej wartości impedancji $Z_{45A}$ i $Z_{B6}$

\begin{align*}
Z_{45AB6} &= Z_{45A} + Z_{B6}=\\
&=\frac{\left(Z_{4}+Z_{5}\right) \cdot \left( R_1+R_3+\frac{R_1 \cdot R_3}{R_2} \right)}{Z_{4} + Z_{5} + R_1+R_3+\frac{R_1 \cdot R_3}{R_2} } + \frac{\left( R_2+R_3+\frac{R_2 \cdot R_3}{R_1} \right) \cdot Z_{6}}{Z_{6}+R_2+R_3+\frac{R_2 \cdot R_3}{R_1}}
\end{align*}

\begin{schemat}
\label{schemat:01:03:kw:J}
\draw
node[ocirc,label=A] (A) at (0, 0) {}
node[ocirc,label=below:B] (B) at (0,-6) {}

(A)    to[short] (2.0,0)
to[Z,l=$Z_{45AB6}$] (2.0,-6)
to[short] (B)
(1,-6) to[Z,l=$Z_C$,*-*] (1, 0)     
;
\end{schemat}
%-----------------------------------------------
Ostatecznie można zauważyć iż impedancje $Z_{C}$ oraz $Z_{45AB6}$ są połączona równolegle. I można ją zastąpić jedną impedancją zastępczą o wartości $Z_{C45AB6}$. 
\begin{schemat}
\label{schemat:01:03:kw:K}
\draw
node[ocirc,label=A] (A) at (0, 0) {}
node[ocirc,label=below:B] (B) at (0,-6) {}

(A)    to[short] (2.0,0)
to[Z,l=$Z_{45AB6}$] (2.0,-6)
to[short] (B)
(1,-6) to[Z,l=$Z_C$,*-*] (1, 0)     
;
\draw[color=red] (1.5,-3.0) ellipse (2.0 and 1.0);
\end{schemat}

\begin{align*}
\frac{1}{Z_{C45AB6}} &= \frac{1}{Z_{C}}+\frac{1}{Z_{45AB6}}\\
&=\frac{Z_{C} + Z_{45AB6}}{Z_{C} \cdot Z_{45AB6}}\\
Z_{C45AB6}&=\frac{Z_{C} \cdot Z_{45AB6}}{Z_{C} + Z_{45AB6}}
\end{align*}

Wstawiając obliczone wcześniej wartości impedancji $Z_{C}$ i $Z_{45AB6}$ otrzymujemy

\begin{align*}
Z_{C45AB6}&=\frac{Z_{C} \cdot Z_{45AB6}}{Z_{C} + Z_{45AB6}}=\\
&= dokończyć
\end{align*}

\begin{schemat}
\label{schemat:01:03:kw:L}
\draw
node[ocirc,label=A] (A) at (0, 0) {}
node[ocirc,label=below:B] (B) at (0,-6) {}

(A)    to[short] (1.5,0)
to[Z,l=$Z_{C45AB6}$] (1.5,-6)
to[short] (B)   
;
\end{schemat}
Tak więc impedancja zastępcza obwodu przedstawionego na schemacie \ref{schemat:01:03:kw:Z} równa się $Z_{C45AB6}$
\end{task}













\chapter{Źródła sterowane i zamiana źródeł} 
\section{Teoria}
\subsection{Źródła idealne}

\subsubsection{Źródło napięciowe}
\begin{schemat}
\draw
 ( 0, 0) to[polish voltage source, l=$U$]       ( 2, 0) 
;
\end{schemat}

\subsubsection{Źródło prądowe}
\begin{schemat}
\draw
 ( 0, 0) to[polish current source, l=$I$]       ( 2, 0) 
;
\end{schemat}

\subsection{Źródła rzeczywiste}

\subsubsection{Źródło napięciowe}
\begin{schemat}
\draw
 ( 0, 0) to[polish voltage source, l=$e$]       ( 2, 0) 
 ( 2, 0) to[R, l=$R_w$]                         ( 4, 0) 
;
\end{schemat}

\subsubsection{Źródło prądowe}
\begin{schemat}
\draw
 ( 0, 0) to[polish current source, l=$j$]       ( 2, 0) 
 ( 0, 2) to[R, l=$R_w$]                         ( 2, 2) 
 ( 0, 0) to[short]( 0, 2)
 ( 2, 0) to[short]( 2, 2)
 ( -1, 1) to[short]( 0, 1)
 ( 2, 1) to[short]( 3, 1)
;
\end{schemat}

\subsection{Źródła sterowane}

\subsubsection{Źródło napięciowe sterowane napięciem}
\begin{schemat}
\draw
 node[ocirc] (A) at ( 0, 0) {}
 node[ocirc] (B) at ( 0, 2) {}
 ( 0, 0) to[open, v=$U_1$]                                ( 0, 2)
 ( 4, 0) to[polish controlled voltage source, l_=${e=k \cdot U_1}$]      ( 4, 2) 
;
\end{schemat}

\subsubsection{Źródło napięciowe sterowane prądem}

\begin{schemat}
\draw
 node[ocirc] (A) at ( 0, 0) {}
 node[ocirc] (B) at ( 0, 2) {}
 ( 0, 0) to[short, i=$I_1$]                                ( 0, 2)
 ( 4, 0) to[polish controlled voltage source, l_=${e=r \cdot I_1}$]      ( 4, 2) 
;
\end{schemat}

\subsubsection{Źródło prądowe sterowane napięciem}

\begin{schemat}
\draw
 node[ocirc] (A) at ( 0, 0) {}
 node[ocirc] (B) at ( 0, 2) {}
 ( 0, 0) to[open, v=$U_1$]                              ( 0, 2)
 ( 4, 0) to[polish controlled current source, l_=${j=g \cdot U_1}$]      ( 4, 2) 
;
\end{schemat}

\subsubsection{Źródło prądowe sterowane prądem}

\begin{schemat}
\draw
 node[ocirc] (A) at ( 0, 0) {}
 node[ocirc] (B) at ( 0, 2) {}
 ( 0, 0) to[short, i=$I_1$]                                ( 0, 2)
 ( 4, 0) to[polish controlled current source, l_=${j= \alpha \cdot I_1}$]      ( 4, 2) 
;
\end{schemat}

\subsection{Zamiana źródeł}

\begin{schemat}
\draw
 ( 0, 0) to[polish voltage source, l=$e$]       ( 2, 0) 
 ( 2, 0) to[R, l=$R_{w1}$]                         ( 4, 0) 
;
\end{schemat}

\begin{schemat}
\draw
 ( 0, 0) to[polish current source, l=$j$]       ( 2, 0) 
 ( 0, 2) to[R, l=$R_{w2}$]                         ( 2, 2) 
 ( 0, 0) to[short]( 0, 2)
 ( 2, 0) to[short]( 2, 2)
 ( -1, 1) to[short]( 0, 1)
 ( 2, 1) to[short]( 3, 1)
;
\end{schemat}

\begin{equation}
R_{w1} = R_{w2} = R_w
\end{equation}
\begin{equation}
e = R_w \cdot j
\end{equation}
\begin{equation}
j = \frac{e}{R_w}
\end{equation}

\subsection{Łączenie źródeł}

\subsection{Szeregowe łączenie źródeł napięciowych}

\begin{schemat}
\draw
 ( 0,  0) to[polish voltage source, l=$e_1$]     ( 0, -2) 
 ( 0, -2) to[R, l=$R_1$]                         ( 0, -4)
 ( 0, -4) to[polish voltage source, l=$e_2$]     ( 0, -6) 
 ( 0, -6) to[R, l=$R_2$]                         ( 0, -8)  
 ( 0, 0) to[short]( 4, 0)
 ( 0,-8) to[short]( 4,-8)
 node[ocirc] (A) at ( 4, 0) {}
 node[ocirc] (B) at ( 4,-8) {} 
;
\end{schemat}

\begin{schemat}
\draw
 ( 0,  0) to[polish voltage source, l=$e_1$]     ( 0, -2) 
 ( 0, -4) to[R, l=$R_1$]                         ( 0, -6)
 ( 0, -2) to[polish voltage source, l=$e_2$]     ( 0, -4) 
 ( 0, -6) to[R, l=$R_2$]                         ( 0, -8)  
 ( 0, 0) to[short]( 4, 0)
 ( 0,-8) to[short]( 4,-8)
 node[ocirc] (A) at ( 4, 0) {}
 node[ocirc] (B) at ( 4,-8) {} 
;
\end{schemat}

\begin{equation}
e = e_1 + e_2
\end{equation}
\begin{equation}
R = R_1 + R_2
\end{equation}

\begin{schemat}
\draw
 ( 0,  0) to[polish voltage source, l=$e$]     ( 0, -2) 
 ( 0, -2) to[R, l=$R$]                         ( 0, -4)
 ( 0, 0) to[short]( 4, 0)
 ( 0,-4) to[short]( 4,-4)
 node[ocirc] (A) at ( 4, 0) {}
 node[ocirc] (B) at ( 4,-4) {} 
;
\end{schemat}

\subsection{Równoległe łączenie źródeł pradowych}

\begin{schemat}
\draw
 ( 0, 0) to[polish current source, l=$j_1$]     ( 0, 2) 
 ( 2, 0) to[R, l=$R_1$]                         ( 2, 2) 
 ( 4, 0) to[polish current source, l=$j_2$]     ( 4, 2) 
 ( 6, 0) to[R, l=$R_2$]                         ( 6, 2)  
 ( 0, 0) to[short]( 8, 0)
 ( 0, 2) to[short]( 8, 2)
 node[ocirc] (A) at ( 8, 0) {}
 node[ocirc] (B) at ( 8, 2) {} 
;
\end{schemat}

\begin{schemat}
\draw
 ( 0, 0) to[polish current source, l=$j_1$]     ( 0, 2) 
 ( 4, 0) to[R, l=$R_1$]                         ( 4, 2) 
 ( 2, 0) to[polish current source, l=$j_2$]     ( 2, 2) 
 ( 6, 0) to[R, l=$R_2$]                         ( 6, 2)  
 ( 0, 0) to[short]( 8, 0)
 ( 0, 2) to[short]( 8, 2)
 node[ocirc] (A) at ( 8, 0) {}
 node[ocirc] (B) at ( 8, 2) {} 
;
\end{schemat}

\begin{equation}
j = j_1 + j_2
\end{equation}
\begin{equation}
R = R_1 || R_2 =\frac{R_1 \cdot R_2}{R_1+R_2}
\end{equation}

\begin{schemat}
\draw
 ( 0, 0) to[polish current source, l=$j$]     ( 0, 2) 
 ( 2, 0) to[R, l=$R$]                         ( 2, 2) 
 ( 0, 0) to[short]( 8, 0)
 ( 0, 2) to[short]( 8, 2)
 node[ocirc] (A) at ( 8, 0) {}
 node[ocirc] (B) at ( 8, 2) {}
;
\end{schemat}

\subsection{Równoległe łączenie źródeł napięciowych}

\begin{schemat}
\draw
 ( 0,  0) to[polish voltage source, l=$e_1$]     ( 0, -2) 
 ( 0, -2) to[R, l=$R_1$]                         ( 0, -4)
 ( 2, -0) to[polish voltage source, l=$e_2$]     ( 2, -2) 
 ( 2, -2) to[R, l=$R_2$]                         ( 2, -4)  
 ( 0, 0) to[short]( 4, 0)
 ( 0,-4) to[short]( 4,-4)
 node[ocirc] (A) at ( 4, 0) {}
 node[ocirc] (B) at ( 4,-4) {} 
;
\end{schemat}

\begin{equation}
e = \frac{e_1 \cdot R_2 + e_2 \cdot R_1}{R_1+R_2}
\end{equation}
\begin{equation}
R = R_1 || R_2 =\frac{R_1 \cdot R_2}{R_1+R_2}
\end{equation}

\begin{schemat}
\draw
 ( 0,  0) to[polish voltage source, l=$e$]     ( 0, -2) 
 ( 0, -2) to[R, l=$R$]                         ( 0, -4)
 ( 0, 0) to[short]( 4, 0)
 ( 0,-4) to[short]( 4,-4)
 node[ocirc] (A) at ( 4, 0) {}
 node[ocirc] (B) at ( 4,-4) {} 
;
\end{schemat}



\subsection{Szeregowe łączenie źródeł pradowych}

\begin{schemat}
\draw
( 0, 0) to[short] ( 0,-1)
(-1,-1) to[short] ( 1,-1)
(-1,-1) to[Isrc, l=$j_1$] (-1,-3)
( 1,-1) to[R, l=$R_1$] ( 1,-3)
(-1,-3) to[short] ( 1,-3)
( 0,-3) to[short] ( 0,-5)
(-1,-5) to[short] ( 1,-5)
(-1,-5) to[Isrc, l=$j_2$] (-1,-7)
( 1,-5) to[R, l=$R_2$] ( 1,-7)
(-1,-7) to[short] ( 1,-7)
( 0,-7) to[short] ( 0,-8)
 ( 0, 0) to[short]( 4, 0)
 ( 0,-8) to[short]( 4,-8)
 node[ocirc] (A) at ( 4, 0) {}
 node[ocirc] (B) at ( 4,-8) {} 
;
\end{schemat}

\begin{schemat}
\draw
 ( 0,  0) to[polish voltage source, l=$e_1$]     ( 0, -2) 
 ( 0, -2) to[R, l=$R_1$]                         ( 0, -4)
 ( 0, -4) to[polish voltage source, l=$e_2$]     ( 0, -6) 
 ( 0, -6) to[R, l=$R_2$]                         ( 0, -8)  
 ( 0, 0) to[short]( 4, 0)
 ( 0,-8) to[short]( 4,-8)
 node[ocirc] (A) at ( 4, 0) {}
 node[ocirc] (B) at ( 4,-8) {} 
;
\end{schemat}

\begin{schemat}
\draw
 ( 0,  0) to[polish voltage source, l=$e_1$]     ( 0, -2) 
 ( 0, -4) to[R, l=$R_1$]                         ( 0, -6)
 ( 0, -2) to[polish voltage source, l=$e_2$]     ( 0, -4) 
 ( 0, -6) to[R, l=$R_2$]                         ( 0, -8)  
 ( 0, 0) to[short]( 4, 0)
 ( 0,-8) to[short]( 4,-8)
 node[ocirc] (A) at ( 4, 0) {}
 node[ocirc] (B) at ( 4,-8) {} 
;
\end{schemat}

\begin{equation}
R = R_1 + R_2
\end{equation}
\begin{equation}
e = e_1 + e_2 = j_1 \cdot R_1 + j_2 \cdot R_2
\end{equation}
\begin{equation}
j = \frac{e}{R} = \frac{j_1 \cdot R_1 + j_2 \cdot R_2}{R_1 + R_2}
\end{equation}


\begin{schemat}
\draw
 ( 0,  0) to[polish voltage source, l=$e$]     ( 0, -2) 
 ( 0, -2) to[R, l=$R$]                         ( 0, -4)
 ( 0, 0) to[short]( 4, 0)
 ( 0,-4) to[short]( 4,-4)
 node[ocirc] (A) at ( 4, 0) {}
 node[ocirc] (B) at ( 4,-4) {} 
;
\end{schemat}













\section{Zadania}
\begin{task}
Uprość obwód
\begin{schemat}
\draw
 ( 0,-2) to[polish current source, l=$10mA$]       ( 0, 0) 
 ( 2, 0) to[R, l=${2k \Omega}$]                    ( 2,-2)
 ( 2, 0) to[polish voltage source, l=$8V$]         ( 4, 0)
 ( 4, 0) to[R, l=${4k \Omega}$]                    ( 6, 0)
 ( 6, 0) to[R, l=${6k \Omega}$]                    ( 6,-2)
 ( 6, 0) to[R, l=${3k \Omega}$]                    ( 8, 0)
 ( 8,-2) to[polish voltage source, l_=$6V$]        ( 8, 0)
 ( 0, 0) to[short]                                 ( 2, 0)
 ( 0,-2) to[short]                                 ( 8,-2)
;
\end{schemat}
\subsubsection{Rozwiązanie}
TBD
\end{task}
\begin{task}
Uprość obwód
\begin{schemat}
\draw
 ( 0,-4) to[Vsrc,  l=$E$]             ( 0, 0)
 ( 2,-2) to[cVsrc, l=${R \cdot J_1}$] ( 2, 0)
 ( 2,-4) to[R, l=$3R$]                ( 2,-2)
 ( 6,-4) to[R, l=$2R$]                ( 6, 0)
 ( 9,-4) to[Isrc,  l=${J = \frac{E}{2R}}$] ( 9, 0)
 ( 6, 0) to[short]                    ( 9, 0) 
 ( 0, 0) to[R, l=$3R$, i=$J_1$]       ( 2, 0)
 ( 2, 0) to[R, l=$R$]                 ( 4, 0)
 ( 4, 0) to[Vsrc, l=$E$]              ( 6, 0)
 ( 0,-4) to[short]                    ( 9,-4)
;
\end{schemat}
\subsubsection{Rozwiązanie}
TBD
\end{task}









































\chapter{Prawa Kirchhoffa}
\section{Teoria}
\subsection{I prawo Kirchhoffa - prądowe prawo Kirchhoffa}

\begin{center}
\begin{circuitikz}[european] 
\ctikzset{voltage/distance from line=0.2} % można jeszcze poprawić wygląd
\ctikzset{voltage/european label distance=2.00}  
%\ctikzset{voltage/distance from node=.1}
\draw
 (0,0) to[short,i>_=$I_1$,-*] (2,0) 
       to[short,i>_=$I_3$,*-] (2.5,1)
 (2,0) to[short,i<_=$I_5$,*-] (2.5,-1)       
 (2,0) to[short,i>_=$I_4$,*-] (4,0)
 (2,0) to[short,i>_=$I_2$,*-] (1,1) 
;
\end{circuitikz}
\end{center}

Suma prądów wpływających i wypływających z węzła jest równa 0

\begin{equation}
\sum_{i}I_i=0
\end{equation}

Prądy wpływające do węzła ($I_1$, $I_5$) oznaczamy ze znakiem plus, prądy wypływające z węzła ($I_2$, $I_3$, $I_4$) oznaczamy ze znakiem minus.

\begin{equation}
I_1-I_2-I_3-I_4+I_5=0
\end{equation}

\subsection{II prawo Kirchhoffa - napięciowe prawo Kirchhoffa}

\begin{center}
\begin{circuitikz}[european] 
\ctikzset{voltage/distance from line=0.2} % można jeszcze poprawić wygląd
\ctikzset{voltage/european label distance=2.00}  
%\ctikzset{voltage/distance from node=.1}
\draw
 (-2,0) to[R,l_=$R_2$,i>_=$I_2$,v^=$U_2$] (2,0) 
       to[R,l_=$R_3$,v^=$U_3$]  (2,-2)  
       to[american current source,l_=$E_3$] (2,-3)  
       to[short,i>_=$I_3$] (2,-4)
       to[american current source,l_=$E_4$] (0,-4)  
       to[R,l_=$R_4$,v^>=$U_4$]  (-2,-4)
       to[R,l_=$R_1$,i<_=$I_1$,v^>=$U_1$]  (-2,0)  ;
\draw[->] (-1,-2) arc (180:-135:1cm) 
;
\end{circuitikz}
\end{center}

W każdym obwodzie zamkniętym suma spadków napięć na poszczególnych elementach obwodu jest w każdej chwili równa zero

\begin{equation}
\sum_{i}U_i=0
\end{equation}

Należy założyć sobie kierunek obiegu obwodu. Napięcia zgodne z kierunkiem obiegu obwodu mają znak dodatki, napięcia przeciwne z kierunkiem obiegu obwodu mają znak ujemny. 

\begin{equation}
U_1-U_2-U_3+E_3+E_4+U_4=0
\end{equation}

\section{Zadania}
\begin{task}
Wyznacz spadek napięcia na rezystorze $R_1$
\begin{center}\begin{circuitikz}[european] 
\ctikzset{voltage/distance from line=0.2} 
\ctikzset{voltage/european label distance=2.00}  
\draw
 (0,0) to[R,l=$R_1$] (2,0) 
       to[R,l=$R_2$] (2,-2)
       to[short] (0,-2)
       to[american current source,l=$E$] (0,0) 
;
\end{circuitikz}
\end{center}
\subsubsection{Rozwiązanie}
TBD
\end{task}
\begin{task}
W układzie dzielnika prądowego pokazanego na rysunku dobrać tak wartość oporu $R_2$, aby przez opór $R_1$ płynął prąd o natężeniu $p\cdot I$

\begin{center}\begin{circuitikz}[european] 
\ctikzset{voltage/distance from line=0.2} 
\ctikzset{voltage/european label distance=2.00}  
%przesynąć etykiety węzłów dalej od węzłów
\draw
 (0,0) to[short,i>_=$I$,-*] (1,0) 
       to[short] (1,-0.5) 
       to[R,l_=$R_2$] (3,-0.5)
       to[short,-*] (3,0)
       to[short,-*] (4,0)       
 (1,0) to[short] (1,0.5) 
       to[R,l_=$R_1$] (3,0.5)
       to[short] (3,0)       
;
\end{circuitikz}
\end{center}
\subsubsection{Rozwiązanie}
TBD
\end{task}
\begin{task}
Wzynacy wartość siłz elektromotorycznej $E$, która na oporze $R_7$ powoduje spadek napięcia $U$

\begin{center}\begin{circuitikz}[european] 
\ctikzset{voltage/distance from line=0.2} 
\ctikzset{voltage/european label distance=2.00}  
%przesynąć etykiety węzłów dalej od węzłów
\draw
 (0,0) to[R,l_=$R_1$] (2,0)
       to[R,l_=$R_2$] (2,-2)
       to[short] (0,-2)
       to[american current source,l=$E$] (0,0)
 (2,0) to[R,l_=$R_3$,*-] (4,0)
       to[R,l_=$R_4$] (4,-2)
       to[R,l_=$R_5$,-*] (2,-2)
 (4,0) to[R,l_=$R_6$,*-] (6,0)
       to[R,l_=$R_7$] (6,-2)
       to[short,-*] (4,-2)       
;\end{circuitikz}\end{center}

Do obliczeń przyjmij że $U=2V$, $R_1=5 \Omega$, $R_2=6 \Omega$, $R_3=7 \Omega$, $R_4=8 \Omega$, $R_5=9 \Omega$, $R_6=10 \Omega$, $R_7=5 \Omega$, 

\subsubsection{Rozwiązanie}
TBD
\end{task}
\begin{task}
Oblicz rozpływ prądów w układzie podanym na rysunku metodą praw Kirchhoffa. Wyznacz napięcia na rezystorach.

\begin{center}\begin{circuitikz}[european] 
\ctikzset{voltage/distance from line=0.2} 
\ctikzset{voltage/european label distance=2.00}  
%przesynąć etykiety węzłów dalej od węzłów
\draw
 (0,0) to[R,l=$R_1$] (2,0)
       to[R,l_=$R_3$] (2,-2)
       to[short] (0,-2)
       to[american current source,l=$E_1$] (0,0)
 (2,-2) to[short,*-] (4,-2)
       to[american current source,l=$E_2$] (4,0)
       to[R,l_=$R_2$,-*] (2,0)
;\end{circuitikz}\end{center}

Do obliczeń przyjmij że: $E_1=4V$, $E_2=6V$, $R_1=2\Omega$, $R_2=12\Omega$, $R_3=4\Omega$.

\subsubsection{Rozwiązanie}
TBD
\end{task}
\begin{task}
Oblicz rozpływ prądów metodą praw Kirchhoffa

\begin{center}\begin{circuitikz}[european] 
\ctikzset{voltage/distance from line=0.2} 
\ctikzset{voltage/european label distance=2.00}  
%przesynąć etykiety węzłów dalej od węzłów
\draw
 (0,0) to[R,l=$R_1$] (3,3)
       to[R,l_=$R_2$] (6,0)
       to[R,l_=$R_3$] (3,-3)
       to[R,l_=$R_4$] (0,0)
 (0,0) to[short,*-] (1,0)
       to[R,l_=$R_5$] (3,0) 
       to[american current source,l_=$E_5$] (5,0)
       to[short,-*] (6,0)
 (3,-3) to[short,*-] (8,-3)
        to[american current source,l_=$E_6$] (8,0)
        to[R,l_=$R_6$] (8,3) 
        to[short,-*] (3,3)
;\end{circuitikz}\end{center}

Do obliczeń przyjmij: $E_5=6V$, $E_6=4V$, $R_1=3\Omega$, $R_2=2\Omega$, $R_3=4\Omega$, $R_4=5\Omega$, $R_5=1\Omega$, $R_6=2\Omega$.

\subsubsection{Rozwiązanie}
TBD
\end{task}
\begin{task}
W układzie przedstawionym poniżej dobierz opór $R$ w taki sposób aby prąd $I$ był równy zero.

\begin{center}\begin{circuitikz}[european] 
\ctikzset{voltage/distance from line=0.2} 
\ctikzset{voltage/european label distance=2.00}  
%przesynąć etykiety węzłów dalej od węzłów
\draw
 (0,0) to[R,l=$40\Omega$] (2,0)
       to[R,l=$30\Omega$] (2,-3)
       to[short] (0,-3)
       to[american current source,l=$100V$] (0,0)
 (2,0) to[short,*-*] (3,0)
        to[short] (3,0.5)
        to[R,l=$40\Omega$] (5,0.5) 
        to[short] (5,0)
        to[short,*-*] (6,0)
 (3,0)  to[short] (3,-0.5)
        to[R,l=$40\Omega$] (5,-0.5) 
        to[short] (5,0)        
 (2,-3) to[short,*-*] (6,-3)          
 (6,-3) to[short] (8,-3)
        to[short] (8,-2)
        to[american current source,l_=$35V$] (8,-1) 
        to[short,i>_=$I$] (8,0)
        to[R,l_=$40\Omega$] (6,0)
        to[R,l=$R$] (6,-3)
       
;\end{circuitikz}\end{center}

\subsubsection{Rozwiązanie}
TBD
\end{task}
\begin{task}
W układzie przedstawionym poniżej określ rozpływ prądów i rozkład napięć.

\begin{center}\begin{circuitikz}[european] 
\ctikzset{voltage/distance from line=0.2} 
\ctikzset{voltage/european label distance=2.00}  
%przesynąć etykiety węzłów dalej od węzłów
\draw
 (0,0) to[R,l=$R_2$] (2,0)
       to[american current source,l=$E$] (4,0)
       to[R,l_=$R_3$] (4,-3)
       to[short] (0,-3)
       to[R,l_=$R_1$] (0,0) 
 (0,0) to[short] (0,1)
       to[R,l=$R_4$] (4,1)
       to[short] (4,0)    
 (0,-3) to[short] (-1,-3)
        to[american current source,l=$I_1$] (-1,0)
        to[short] (0,0)
 (4,-3) to[short] (5,-3)
        to[american current source,l_=$I_2$] (5,0)
        to[short] (4,0)        
;\end{circuitikz}\end{center}

Do obliczeń przyjmij $R_1=1\Omega$, $R_2=2\Omega$, $R_3=3\Omega$, $R_4=4\Omega$, $E=10V$, $I_1=2A$, $I_2=5A$.

\subsubsection{Rozwiązanie}

Zastosować przekształcenie źródeł

TBD
\end{task}
\begin{task}
W układzie przedstawionym poniżej oblicz wartość prądu $\underline{I_1}$

\begin{schemat}
\label{schemat:03:07:kw:Z}
\draw
 (0,0) to[Z,l=$Z$] (0,2)
       to[Vsrc,l=$\underline{E}$] (0,4)
       to[Z,l_=$Z$] (2,4)
       to[Z,l_=$Z$] (4,4)
       to[short,i=$\underline{I_1}$] (6,4)
       to[Z,l_=$2 \cdot Z$] (6,0) 
       to[short] (2,0)
       to[Z,l_=$Z$] (0,0)
 (2,0) to[Z,l=$2\cdot Z$,*-*] (2,4)
 (4,0) to[Z,l=$Z$,*-] (4,2)
       to[Z,l=$Z$,-*] (4,4)
;
\end{schemat}

Do obliczeń przyjmij $\underline{E}=3\cdot e^{\jmath \cdot \frac{\pi}{4}} \volt$, $Z=2+\jmath\ohm$.

\subsubsection{Rozwiązanie}

TBD
\end{task}
\begin{task}
W układzie przedstawionym poniżej dobierz wartość zespolonej amplitudy napięcia $\underline{E}$ aby wartość zespolonej amplitudy natężenia prądu $\underline{I_x}$ była równa $1+\jmath\amper$.

\begin{schemat}
\label{schemat:03:08:kw:Z}
\draw
 (0,0) to[Z,l=$Z$] (0,2)
       to[Vsrc,l=$\underline{E}$] (0,4)
       to[Z,l_=$Z$] (2,4)
       to[Z,l_=$Z$] (4,4)
       to[short,i=$\underline{I_1}$] (6,4)
       to[Z,l_=$Z$] (6,2) 
       to[Z,l_=$Z$] (6,0)        
       to[short] (2,0)
       to[Z,l_=$Z$] (0,0)
 (2,0) to[Z,l=$3\cdot Z$,*-*] (2,4)
 (4,0) to[Z,l=$2\cdot Z$,*-*] (4,4)
;
\end{schemat}

Do obliczeń przyjmij $Z=1+2\cdot \jmath\ohm$.

\subsubsection{Rozwiązanie}

TBD
\end{task}
\begin{task}
W układzie przedstawionym poniżej wyznacz wartość zespolonej amplitudy natężenia prądu $\underline{I_1}$.

\begin{schemat}
\label{schemat:03:09:kw:Z}
\draw
 (0,0) to[Z,l=$Z_1$] (0,2)
       to[Vsrc,l=$e_1(t)$] (0,4)
       to[short,i=$\underline{I_1}$] (2,4)
       to[short] (4,4)
       to[Vsrc,l_=$e_2(t)$] (6,4)
       to[Z,l_=$Z_4$] (6,0) 
       to[Z,l_=$Z_3$] (4,0)        
       to[short] (0,0)
 (2,0) to[Isrc,l=$j(t)$,*-*] (2,4)
 (4,0) to[Z,l=$Z_2$,*-*] (4,4)
;
\end{schemat}

\begin{align*}
e_1(t)&=2 \cdot sin(2 \cdot t)\volt\\
e_2(t)&= sin(2 \cdot t)\volt\\
j(t)&= cos(2 \cdot t)\amper\\
Z_1&=\jmath\ohm\\
Z_2&=2\jmath+1\ohm\\
Z_3&=\jmath\ohm\\
Z_4&=1\ohm\\
\end{align*}

\subsubsection{Rozwiązanie}

TBD
\end{task}
\begin{task}
W układzie przedstawionym poniżej dobierz wartość zespolonej amplitudy natężenia prądu źródła $\underline{J_2}$ w taki sposób aby wartość zespolonej amplitudy natężenia prądu $\underline{I_1}$ była równa $0$.

\begin{schemat}
\label{schemat:03:10:kw:Z}
\draw
 (0,0) to[Z,l=$Z_1$] (0,2)
       to[Vsrc,l=$e_1(t)$] (0,4)
       to[short] (2,4)
       to[short,i=$\underline{I_1}$] (4,4)
       to[short] (6,4)
 (6,0) to[Isrc,l=$j_2(t)$] (6,4) 
 (6,0) to[short] (4,0)        
       to[Z,l_=$Z_2$] (2,0)
       to[short] (0,0)
 (2,0) to[Isrc,l=$j(t)$,*-*] (2,4)
 (4,0) to[Z,l=$Z_3$,*-*] (4,4)
;
\end{schemat}

\begin{align*}
e_1(t)&=2 \cdot sin(2 \cdot t)\volt\\
j(t)&= cos(2 \cdot t)\amper\\
Z_1&=\jmath\ohm\\
Z_2&=2\jmath+1\ohm\\
Z_3&=\jmath\ohm\\
\end{align*}

\subsubsection{Rozwiązanie}

TBD
\end{task}
\begin{task}
W układzie przedstawionym poniżej dobierz wartość impedancji $Z_1$ w taki sposób aby ...

\begin{align*}
e(t)&=2 \cdot sin(2 \cdot t)\volt\\
j(t)&=2 \cdot sin(2 \cdot t + \frac{\pi}{2})\volt\\
R_1&=2\ohm\\
R_2&=1\ohm\\
L_1&=3\henr\\
C_1&=0.5\farad
\end{align*}


\begin{schemat}
\label{schemat:03:11:kw:Z}
\draw
 (0,0) to[Vsrc,l=$e_1(t)$] (0,3)
       to[short] (0,6)
       to[Vsrc,l=$e_2(t)$] (2,6)
       to[R,l=$R_1$] (4,6)       
       to[short] (4,3)
       to[short] (5,3)
       to[L,l=$L_1$] (5,0)
       to[short] (4,0)
       to[C,l=$C_1$] (0,0)
 (4,0) to[Isrc,l=$j(t)$] (4,3)
 (4,0) to[R,l=$R_1$] (0,3)
       to[Z,l_=$Z_1$] (4,3)
 (4,4.5) to[Isrc,l=$j(t)$] (0,4.5) 
;
\end{schemat}


\subsubsection{Rozwiązanie}

TBD
\end{task}
\begin{task}
Oblicz przesunięcie fazowe pomiędzy natężeniem prądu $\underline{I_1}$ a napięciem $\underline{U_1}$ w poniższym układzie.  

\begin{schemat}
\label{schemat:03:12:kw:Z}
\draw
node[ocirc,label=A] (A) at (2, 4) {}
node[ocirc,label=below:B] (B) at (2,0) {}

 (0,0) to[R,l=$R_1$] (0,2) 
       to[Vsrc,l=$e_1(t)$] (0,4)
       to[short,i>=$\underline{I_1}$] (2,4)
       to[L,l=$L_1$] (4,4)       
       to[short] (5,4)
       to[C,l=$C$] (5,0)
       to[short] (2,0)
       to[L,l=$L_2$] (0,0)
 (4,0) to[R,l=$R_2$] (4,4)
 (B)   to[open,v^>=$\underline{U_1}$] (A)

;
\end{schemat}


\subsubsection{Rozwiązanie}

TBD
\end{task}
\begin{task}
Wyznacz wartość natężenia prądu $i_1(t)$ oraz $i_1(t)$ w obwodzie na rysunku poniżej.

\begin{align*}
L&=1 n \henr \\
C&=250 p \farad \\
e_1(t)&=100 \cdot sin\left(10^9 \cdot t + \frac{\pi}{4} \right) \\
e_2(t)&=100 \cdot sin\left(10^9 \cdot t + \frac{\pi}{2} \right)
\end{align*}

\begin{schemat}
\label{schemat:03:13:kw:Z}
\draw
 (0,0) to[Vsrc,l=$e_1(t)$] (0,3)
       to[L,l=$L$,i>^=$\underline{I_1}$] (3,3)
       to[L,l=$L$] (6,3)
       to[C,l=$C$] (6,0)       
       to[Vsrc,l=$e_2(t)$] (3,0)
       to[C,l=$C$] (0,0)
 (3,0) to[L,l=$L$,i<=$\underline{I_2}$] (3,3)
;
\end{schemat}


\subsubsection{Rozwiązanie}

TBD
\end{task}
\begin{task}
Wyznacz wartość napięcia $u_c(t)$ na kondensatorze w obwodzie na rysunku poniżej.

\begin{align*}
e_1(t)&=sin\left(2 \cdot 10^9 \cdot t + \frac{3\pi}{4} \right) \\
e_2(t)&=sin\left(2 \cdot 10^9 \cdot t + \frac{3\pi}{2} \right) \\
e_2(t)&=sin\left(2 \cdot 10^9 \cdot t + \pi \right) \\
L_1&=1 n \henr \\
L_2&=0,5 n \henr \\
R_1&=R_2=R_3=1 \ohm \\
C_1&=10,5 n \farad 
\end{align*}

\begin{schemat}
\label{schemat:03:14:kw:Z}
\draw
 (0,0) to[Vsrc,l=$e_1(t)$] (0,2)
       to[L,l=$L_1$] (2,2)
       to[L,l=$L_2$] (6,2)
 (0,0) to[C,l=$C_1$] (2,0)
       to[Vsrc,l_=$e_2(t)$] (4,0)
       to[R,l_=$R_2$] (6,0)
 (2,2) to[R,l=$R_1$] (2,0)
 (0,0) to[short] (0,-2)
       to[R,l=$R_3$] (6,-2)
       to[short] (6,0)
       to[Vsrc,l=$e_3(t)$] (6,2)
;
\end{schemat}


\subsubsection{Rozwiązanie}

TBD
\end{task}
\begin{task}
Wyznacz wartość napięcia $u_{c1}(t)$ na kondensatorze w obwodzie na rysunku poniżej.

\begin{align*}
e_1(t)&=sin\left(2 \cdot 10^9 \cdot t + \frac{3\pi}{4} \right) \\
e_2(t)&=sin\left(2 \cdot 10^9 \cdot t + \frac{3\pi}{2} \right) \\
e_2(t)&=sin\left(2 \cdot 10^9 \cdot t + \pi \right) \\
L_1&=1 n \henr \\
L_2&=0,5 n \henr \\
R_1&=R_2=R_3=1 \ohm \\
C_1&=10,5 n \farad \\
C_2&=1 n \farad 
\end{align*}

\begin{schemat}
\label{schemat:03:15:kw:Z}
\draw
 (0,0) to[Vsrc,l=$e_1(t)$] (0,2)
       to[C,l=$C_1$] (2,2)
       to[R,l=$R_2$] (6,2)
 (0,0) to[C,l=$C_2$] (2,0)
       to[Vsrc,l_=$e_2(t)$] (4,0)
       to[L,l_=$L_2$] (6,0)
 (2,2) to[R,l=$R_1$] (2,0)
 (0,0) to[short] (0,-2)
       to[Vsrc,l_=$e_3(t)$] (3,-2)
       to[R,l=$R_3$] (6,-2)
       to[short] (6,0)
       to[L,l=$L_1$] (6,2)
;
\end{schemat}


\subsubsection{Rozwiązanie}

TBD
\end{task}





















%\input{04_metoda_wezlowa/04_metoda_wezlowa.tex}
\chapter{Metoda prądów oczkowych}

\section{Zadania}
\begin{task}
Oblicz rozpływ prądów metodą prądów oczkowych

\begin{schemat} \draw
(0,0)  to [Vsrc,l=$E_1$] (0,2)
       to [C,l=$C$] (0,4)
       to [R,l=$R_3$] (2,4)
       to [L,l=$L$] (4,4)
       to [Vsrc,l=$E_2$] (4,2)
       to [short] (4,0)
       to [short] (2,0)
       to [short] (0,0)
(2,0)  to [short] (2,2)
       to [C,l=$C$] (2,4)
;\end{schemat}

%\subsubsection{Rozwiązanie}
%TBD
\end{task}
\begin{task}
Oblicz rozpływ prądów metodą prądów oczkowych

\begin{schemat} \draw
(0,0)  to [short] (0,2)
       to [R,l=$R_1$] (0,4)
       to [R,l=$R_2$] (2,4)
       to [short]     (4,4)
       to [short]     (6,4)
       to [R,l=$R_3$] (8,4)
       to [Vsrc,l=$E_2$] (8,2)
       to [short] (8,0)
       to [short] (8,-2)
       to [short] (6,-2)
       to [R,l=$R_4$] (4,-2)
       to [Vsrc,l=$E_3$] (2,-2)
       to [short] (0,-2)
       to [short] (0,0)
(0,0)  to [short] (2,0)
       to [R,l=$R_5$] (4,0)
       to [R,l=$R_6$] (6,0)
       to [short] (8,0)
(4,0)  to [R,l=$R_7$] (4,2)
       to [Vsrc,l=$E_1$] (4,4)
;\end{schemat}

%\subsubsection{Rozwiązanie}
%TBD
\end{task}
\begin{task}
Oblicz rozpływ prądów metodą prądów oczkowych

\begin{align}
e_1(t) = 2 sin(2t+\frac{\pi}{2}) \\
e_2(t) = sin(2t)\\
e_3(t) =\sqrt{8} sin(2t) \\
e_4(t) = 2 sin(2t+\frac{\pi}{4}) \\
L = 1 \henr \\
C = 1 \farad \\
R = 1 \ohm \\
\end{align}

\begin{schemat} \draw
(0,0)  to [C,l=$C$] (0,2)
       to [Vsrc,l=$e_1(t)$] (0,4)
       to [R,l=$R_3$] (2,4)
       to [Vsrc,l=$e_3(t)$] (4,4)
       to [C,l=$C$] (4,2)
       to [Vsrc,l=$e_4(t)$] (4,0)
       to [R,l=$R$] (2,0)
       to [L,l=$L$] (0,0)
(2,0)  to [R,l=$R$] (2,2)
       to [Vsrc,l=$e_2(t)$] (2,4)
;\end{schemat}

%\subsubsection{Rozwiązanie}
%TBD
\end{task}
\begin{task}
Oblicz rozpływ prądów metodą prądów oczkowych

\begin{align}
e_1(t) = 2 sin(2t+\frac{\pi}{2}) \\
e_2(t) = sin(2t)\\
e_3(t) =\sqrt{8} sin(2t) \\
e_4(t) = 2 sin(2t+\frac{\pi}{4}) \\
L = 1 \henr \\
C = 1 \farad \\
R = 1 \ohm \\
\end{align}

\begin{schemat} \draw
(0,0)  to [C,l=$C$] (0,2)
       to [Vsrc,l=$e_1(t)$] (0,4)
       to [R,l=$R_3$] (4,4)
       to [Vsrc,l=$e_3(t)$] (6,4)
       to [L, l=$L$] (8,4)
       to [C,l=$C$] (8,2)
       to [Vsrc,l=$e_4(t)$] (8,0)
       to [R,l=$R$] (4,0)
       to [L,l=$L$] (0,0)
(4,0)  to [R,l=$R$] (4,2)
       to [Vsrc,l=$e_2(t)$] (4,4)
;\end{schemat}

%\subsubsection{Rozwiązanie}
%TBD
\end{task}
\begin{task}
Oblicz rozpływ prądów metodą prądów oczkowych

\begin{schemat} \draw
(0,0)  to [Vsrc>=$E_1$] (0,4)
       to [R,l=$R$] (3,4)
       to [short] (4,4)
       to [R,l=$R$] (7,4)
       to [L,l=$L$] (7,0)
       to [short] (7,-2)
       to [C,l=$C$] (4,-2)
       to [Vsrc,l=$E_2$] (3,-2)
       to [L,l=$L$] (0,-2)
       to [short] (0,0)
(0,0)  to [R,l=$R$] (3,0)
       to [short] (4,0)
       to [L,l=$L$] (7,0)
(3,0)  to [Isrc<=$J$] (3,4)
(4,0)  to [R,l_=$R$] (4,4)
;\end{schemat}

%\subsubsection{Rozwiązanie}
%TBD
\end{task}























\chapter{Metody źródeł zastępczych}
\section{Teoria}
\subsection{Twierdzenie Thevenina}
\subsection{Twierdzenie Nortona}

\section{Zadania}
\begin{task}
Rozwiąż metodą potencjałów węzłowych

\begin{schemat} \draw
(0,0)  to [Vsrc,l=$2V$] (0,2)
(0,2)  to [R,l=$10\Omega$] (0,4)

(0,4)  to [R,l=$10\Omega$] (2,4)
(2,4)  to [Vsrc,l=$2V$] (4,4)

(2,0)  to [Vsrc,l=$2V$] (0,0)
(2,0)  to [R,l=$10\Omega$] (4,0)

(4,2)  to [Vsrc,l=$2V$] (4,0)
(4,2)  to [R,l=$10\Omega$] (4,4)

(4,0)  to [short] (6,0)
(4,4)  to [short] (6,4)

node[ocirc] (A) at (6,0) {}
node[ocirc] (B) at (6,4) {}


;\end{schemat}

%\subsubsection{Rozwiązanie}
%TBD
\end{task}
\begin{task}
Rozwiąż metodą potencjałów węzłowych

\begin{schemat} \draw
(0,0)  to [Vsrc,l=$3V$] (0,2)
(0,2)  to [R,l=$3\Omega$] (0,4)

(-1,4)  to [R,l=$7\Omega$] (-1,6)
( 1,4)  to [R,l=$2\Omega$] ( 1,6)
(-1,4)  to [short] (1,4)

(-1,6)  to [Vsrc,l=$2V$] (1,6)

(0,0)  to [short] (6,0)
(1,6)  to [short] (6,6)

node[ocirc] (B) at (6,0) {}
node[ocirc] (a) at (6,6) {}


;\end{schemat}

%\subsubsection{Rozwiązanie}
%TBD
\end{task}
\input{06_thevenin_norton/zad_kw_06_03.tex}
\begin{task}
Rozwiąż metodą potencjałów węzłowych

\begin{schemat} \draw

(0,0)  to [Vsrc,l=$40V$] (0,4)
(8,0)  to [Vsrc,l=$12V$] (8,4)

(0,4)  to [R,l=$60\Omega$] (4,4)
(4,4)  to [R,l=$64\Omega$] (8,4)

(0,0)  to [short] (8,0)

(4,0)  to [short] (4,1)
(4,3)  to [short] (4,4)

node[ocirc] (B) at (4,1) {}
node[ocirc] (a) at (4,3) {}
;\end{schemat}

%\subsubsection{Rozwiązanie}
%TBD
\end{task}
\begin{task}
Rozwiąż metodą potencjałów węzłowych

\begin{schemat} \draw
(0,0)  to [short] (0,4)

(0,4)  to [R,l=$R$,v=$U_1$] (2,4)

(2,0)  to [cVsrc,l=$k \cdot U_1$] (2,2)
(2,2)  to [R,l=$R$] (2,4)

(0,0)  to [short] (6,0)
(2,4)  to [short] (6,4)

node[ocirc] (B) at (6,0) {}
node[ocirc] (a) at (6,4) {}
;\end{schemat}

%\subsubsection{Rozwiązanie}
%TBD
\end{task}






















\chapter{Moc i dopasowanie na moc}

\section{Zadania}
\begin{task}
Moc wydzielana na R

\begin{schemat} \draw
(0,0)  to [Vsrc,l=$2V$] (0,2)
(0,2)  to [R,l=${R_w=10\Omega}$] (0,4)

(4,0)  to [R,l=${R=20\Omega}$] (4,4)

(0,0)  to [short] (4,0)
(0,4)  to [short] (4,4)



;\end{schemat}

%\subsubsection{Rozwiązanie}
%TBD
\end{task}
\begin{task}
Moc wydzielana na R

\begin{schemat} \draw
(0,0)  to [Isrc,l=$0.1A$] (0,4)
(3,0)  to [R,l=${R_w=10\Omega}$,*-*] (3,4)

(8,0)  to [R,l=${R=20\Omega}$] (8,4)

(0,0)  to [short] (8,0)
(0,4)  to [short] (8,4)



;\end{schemat}

%\subsubsection{Rozwiązanie}
%TBD
\end{task}
\begin{task}
Dopasowanie na moc

\begin{schemat} \draw
(0,0)  to [Vsrc,l=${E=0.1A}$] (0,2)
(0,2)  to [R,l=${R_w=2k\Omega}$] (2,2)
(0,0)  to [short] (2,0)
node[ocirc] (A) at (2,0) {}
node[ocirc] (B) at (2,2) {}

(2,0)  to [short] (5,0)
(2,2)  to [short] (3,2)
(3,0)  to [R,l=${R_1=3k\Omega}$] (3,2)
(5,0)  to [R,l=${R_3=?}$] (5,2)
(3,2)  to [R,l=${R_2=1k\Omega}$] (5,2)


;\end{schemat}

%\subsubsection{Rozwiązanie}
%TBD
\end{task}
\begin{task}
Oblicz moc zespoloną wydzielaną na kondensatorze

\begin{schemat} \draw
(0,0)  to [R,l=$R_1$] (0,2)
       to [Vsrc,l=$e(t)$] (2,2)
       to [short] (4,2)
       to [R,l=$R_3$] (6,2)
       to [C,l=$C$] (6,0)
       to [short] (0,0)
(2,0)  to [R,l=$R_2$] (2,2)
(4,0)  to [Isrc,l=$j(t)$] (4,2)


;\end{schemat}

%\subsubsection{Rozwiązanie}
%TBD
\end{task}
\begin{task}
W obwodzie prądu stałego oblicz moc czynną wydzielaną na rezystorze $R_x$

\begin{schemat} \draw
(0,0)  to [R,l=$R$] (0,2)
       to [Vsrc,l=$E$] (2,2)
       to [R,l=$R$] (4,2)
       to [short] (6,2)
       to [R,l=$R_x$] (6,0)
       to [short] (2,0)
       to [R,l=$R$] (0,0)
(2,0)  to [R,l=$R$] (2,2)
(4,0)  to [Isrc,l=$J$] (4,2)


;\end{schemat}

%\subsubsection{Rozwiązanie}
%TBD
\end{task}
\begin{task}
Ile wynosi $R_w$ w obwodzie przedstawionym poniżej jeśli wiadomo że następuje dopasowanie na moc czynną.

\begin{schemat} \draw
(0,0)  to [R,l=$R_w$] (0,2)
       to [Vsrc,l=$E$] (0,4)
       to [short,-o] (1,4)
       to [short] (4,4)
       to [R,l=$R$] (6,4)
       to [short] (8,4)
       to [R,l=$R$] (8,0)
       to [short,-o] (1,0)
       to [short] (0,0)
(2,0)  to [R,l=$R$] (2,2)
       to [R,l=$R$] (2,4)
(4,0)  to [R,l=$R$] (4,4)
(6,0)  to [R,l=$R$] (6,4)

;\end{schemat}

%\subsubsection{Rozwiązanie}
%TBD
\end{task}
\begin{task}
Ile wynosi $R_w$ w obwodzie przedstawionym poniżej jeśli wiadomo że następuje dopasowanie na moc czynną.

\begin{align}
X_{C_1}=X_{C_2}=X_L
\end{align}

\begin{schemat} \draw
(0,0)  to [Z,l=$Z_w$] (0,2)
       to [Vsrc,l=$E$] (0,4)
       to [short,-o] (1,4)
       to [short] (4,4)
       to [R,l=$R$] (6,4)
       to [short] (8,4)
       to [C,l=$X_{C_2}$] (8,0)
       to [short,-o] (1,0)
       to [short] (0,0)
(2,0)  to [C,l=$X_{C_1}$] (2,4)
(4,0)  to [R,l=$R$] (4,4)
(6,0)  to [L,l=$X_L$] (6,4)

;\end{schemat}

%\subsubsection{Rozwiązanie}
%TBD
\end{task}
\begin{task}
Korzystając z dzielników napięć i prądów, sprawdzić czy suma mocy wydzielanej na poszczególnych rezystancjach równa się mocy odebranej na zaciskach układu.

\begin{align}
X_{C_1}=X_{C_2}=X_L
\end{align}

\begin{schemat} \draw
(0,0)  to [R,l=$R$] (0,2)
       to [Vsrc,l=$E$] (0,4)
       to [short,-o] (1,4)
       to [R,l=$R$] (3,4)
       to [short] (4,4)
       to [R,l=$R$] (4,0)
       to [short] (3,0)
       to [R,l=$R$,-o] (1,0)
       to [short] (0,0)
(3,0)  to [R,l=$R$] (3,4)
;\end{schemat}

%\subsubsection{Rozwiązanie}
%TBD
\end{task}
\begin{task}
Udowodnij że w układzie prądu stałego, dopasowanie na moc czynna następuje gdy $R_w=R$

\begin{schemat} \draw
(0,0)  to [R,l=$R_w$] (0,2)
       to [Vsrc,l=$E$] (0,4)
       to [short,-o] (1,4)
       to [short] (2,4)
       to [R,l=$R$] (2,0)
       to [short,-o] (1,0)
       to [short] (0,0)
;\end{schemat}

%\subsubsection{Rozwiązanie}
%TBD
\end{task}
\begin{task}
Udowodnij że w układzie przedstawiony poniżej, dopasowanie na moc czynna następuje gdy $Z_w=Z$

\begin{schemat} \draw
(0,0)  to [Z,l=$Z_w$] (0,2)
       to [Vsrc,l=$E$] (0,4)
       to [short,-o] (1,4)
       to [short] (2,4)
       to [Z,l=$Z$] (2,0)
       to [short,-o] (1,0)
       to [short] (0,0)
;\end{schemat}

%\subsubsection{Rozwiązanie}
%TBD
\end{task}
\begin{task}
Układzie przedstawiony poniżej, dopierz $Z_1$ w taki sposób aby nastąpiło dopasowanie na moc czynna.

\begin{align}
Z_w=2+2 \jmath \\
C = 5 \farad \\
R = 10 \ohm \\
j(t) = 2 sin(t)
\end{align}

\begin{schemat} \draw
(0,0)  to [Z,l=$Z_w$] (0,2)
       to [short,-o] (2,2)
       to [Z,l=$Z_1$] (4,2)
       to [short] (5,2)
       to [C,l=$C$] (5,0)
       to [short,-o] (2,0)
       to [short] (0,0)
(1,0)  to [Isrc,l_=$j(t)$] (1,2)
(4,0)  to [R,l=$R$] (4,2)
;\end{schemat}

%\subsubsection{Rozwiązanie}
%TBD
\end{task}























\chapter{Metoda Superpozycji}
\section{Teoria}

\section{Zadania}
\begin{task}
Oblicz rozpływ prądów metodą superpozycji. 
Do obliczeń przyjmij: $E_1=40V$, $E_2=12V$, $R_1=60\Omega$, $R_2=64\Omega$, $R_3=40\Omega$.

\begin{schemat} \draw
(0,0)  to [R,l=$R_1$] (2,0)
(2,0)  to [R,l=$R_2$] (4,0)
(2,-2)  to [R,l=$R_3$] (2,0)
(0,-2) to [Vsrc,l=$E_1$] (0,0)
(4,-2) to [Vsrc,l=$E_1$] (4,0)
(0,-2) to [short] (4,-2)
;\end{schemat}

%\subsubsection{Rozwiązanie}
%TBD
\end{task}
\begin{task}
Oblicz rozpływ prądów metodą superpozycji. 
Do obliczeń przyjmij: $E_1=45V$, $E_2=30V$, $J_1=1mA$, $R_1=6k\Omega$, $R_2=2k\Omega$, $R_3=4k\Omega$, $R_3=12k\Omega$.

\begin{schemat} \draw
(0,0)  to [short] (0,2)
(4,0)  to [short] (4,2)
(0,2)  to[Vsrc,l=$E_2$] (2,2)
(2,2)  to [R,l=$R_2$] (4,2)

(0,0)  to [R,l=$R_1$] (2,0)
(2,0)  to [R,l=$R_3$] (4,0)
(2,-2)  to [R,l=$R_4$] (2,0)
(0,-2) to [Vsrc,l=$E_2$] (0,0)
(4,-2) to [Isrc,l=$J_3$] (4,0)
(0,-2) to [short] (4,-2)
;\end{schemat}

%\subsubsection{Rozwiązanie}
%TBD
\end{task}
\begin{task}
Oblicz rozpływ prądów metodą superpozycji. 
Do obliczeń przyjmij: $E_1=20V$, $J_2=0.1A$, $R_1=10\Omega$, $R_2=40\Omega$, $R_3=50\Omega$, $R_4=20\Omega$.

\begin{schemat} \draw
(0,0)  to [short] (0,1)
(4,0)  to [short] (4,1)
(0,1)  to [R,l=$R_4$] (4,1)

(0,0)  to [R,l=$R_1$] (2,0)
(2,0)  to [R,l=$R_2$] (4,0)
(2,-2)  to [R,l=$R_3$] (2,0)
(0,-2) to [Vsrc,l=$E_1$] (0,0)
(4,-2) to [Isrc,l=$J_2$] (4,0)
(0,-2) to [short] (4,-2)
;\end{schemat}

%\subsubsection{Rozwiązanie}
%TBD
\end{task}






















\chapter{Układy nieliniowe}

\section{Zadania}
\begin{task}
Napięcie na ${R_n}$. Gdzie ${R_n: U=k \cdot J^2}$

\begin{schemat} \draw
(0,0)  to [Vsrc,l=$2V$] (0,2)
(0,2)  to [R,l=${R_1=1\Omega}$] (0,4)
(2,0)  to [R,l=${R_2=2\Omega}$] (2,4)
(6,0)  to [thR,l=${R_n(J)}$] (6,4)
(0,0)  to [short] (6,0)
(0,4)  to [short] (6,4)

;\end{schemat}

%\subsubsection{Rozwiązanie}
%TBD
\end{task}
\begin{task}
Napięcie na ${R_n}$. Gdzie ${R_n: U=k \cdot J^2}$

\begin{schemat} \draw
(0,0)  to [Vsrc,l=${E=2V}$] (0,2)
(0,2)  to [R,l=${R_1=14\Omega}$] (2,2)
(2,0)  to [R,l_=${R_2=14\Omega}$] (2,2)
(2,2)  to [R,l=${R_3=13\Omega}$] (4,2)
(6,0)  to [thR,l=${R_n(J)}$] (6,2)
(0,0)  to [short] (6,0)
(4,2)  to [short] (6,2)

;\end{schemat}

%\subsubsection{Rozwiązanie}
%TBD
\end{task}
\begin{task}
Napięcie na ${R_n}$. Gdzie ${R_n: U=2 \cdot J^3}$

\begin{schemat} \draw
(0,0)  to [Vsrc,l=${E=5V}$] (0,2)
(0,2)  to [R,l=${R=10\Omega}$] (2,2)
(4,0)  to [thR,l=${R_n(J)}$] (4,2)
(8,0)  to [R,l=${R=10\Omega}$] (8,2)
(10,0) to [Vsrc,l_=${E=\beta \cdot J_1}$] (10,2)
(0,0)  to [short] (10,0)
(2,2)  to [short] (10,2)

;\end{schemat}

%\subsubsection{Rozwiązanie}
%TBD
\end{task}























\chapter{Impedancja zastępcza}

\section{Zadania}
\begin{task}
Oblicz impedancje zastepcza

\begin{schemat} \draw
(0,2)  to [L,l=${Z_1}$] (2,2)
(2,2)  to [C,l=${Z_3}$] (4,2)
(4,2)  to [short]       (6,2)

(2,0)  to [R,l=${Z_2}$] (2,2)
(4,0)  to [L,l=${Z_4}$] (4,2)
(6,0)  to [R,l=${Z_5}$] (6,2)

(0,0)  to [short]       (6,0)
;\end{schemat}

%\subsubsection{Rozwiązanie}
%TBD
\end{task}
\begin{task}
Oblicz impedancje zastepcza

\begin{schemat} \draw
node[ocirc] (A) at (0,0) {}
node[ocirc] (B) at (10,0) {}

(0,0)  to [L,l=${Z_1}$] (4, 0)
(4,-1) to [short]       (4, 1)
(4,1)  to [R,l=$R$]     (8, 1)
(4,-1) to [C,l=$C$]     (8,-1)
(8,-1) to [short]       (8, 1)
(8,0)  to [short]       (10,0)
;\end{schemat}

%\subsubsection{Rozwiązanie}
%TBD
\end{task}
\begin{task}
Oblicz impedancje zastepcza

\begin{schemat} \draw
(0,2)  to [L,l=${L}$] (2,2)
(2,2)  to [R,l=${R}$] (4,2)
(4,2)  to [C,l=${C}$,v=${U_C}$] (6,2)
(6,2)  to [short]     (11,2)

(8,0)  to [R,l=${R}$] (8,2)
(11,0)  to [cIsrc,l=${g \cdot U_C}$] (11,2)

(0,0)  to [short]       (11,0)
;\end{schemat}

%\subsubsection{Rozwiązanie}
%TBD
\end{task}
























\chapter{Metoda potencjałów węzłowych i metoda prądów oczkowych}

\section{Zadania}
\begin{task}

\begin{schemat} \draw

( 0,-4)  to[R,l=$R_1$]        (-2,-4)
(-2,-4)  to[Vsrc,l=$E_1$]     (-4,-4)
(-4,-4)  to[short]            (-5, 0)

( 1, 4)  to [Isrc,l_=${J_2}$] (-1, 4)
(-1, 2)  to [C,l=${C_2}$]     ( 1, 2)
(-1, 4)  to [short]           (-1, 2)
( 1, 4)  to [short]           ( 1, 2)
( 1, 3)  to [short]           ( 2, 3)
(-1, 3)  to [short]           (-2, 3)
( 2, 3)  to [short]           ( 5, 0)
(-2, 3)  to [short]           (-5, 0)

( 0,-4)  to[L,l=$L_3$]     ( 4,-4)
( 4,-4)  to[short]            ( 5, 0)

(-5, 0) to[R,l=$R_4$] ( 0, 0)

( 0, 0) to[C,l=$C_5$] ( 5, 0)

( 0,-4)  to [Vsrc,l=${E_6}$] ( 0, 0)

;\end{schemat}

%\subsubsection{Rozwiązanie}
%TBD
\end{task}
\begin{task}

\begin{schemat} \draw

( 0, 0) to[L,l=$L_1$]        ( 0, 2)
( 0, 2) to[R,l=$R_1$]        ( 0, 4)
( 0, 4) to[Vsrc,l=$E_1$]     ( 0, 6)
( 0, 0) to[short]            ( 2, 0)
( 0, 6) to[short]            ( 2, 6)

( 2, 6) to[Vsrc,l_=$E_2$]    ( 2, 3)
( 2, 0) to[C,l=$C_2$]        ( 2, 3)

( 2, 6) to[L,l=$L_3$]        ( 5, 6)
( 5, 6) to[Vsrc,l=$E_3$]     ( 8, 6)

( 2, 0) to[L,l=$L_4$]        ( 5, 0)
( 8, 0) to[Vsrc,l_=$E_4$]    ( 5, 0)

( 8, 3) to[C,l=$C_5$]        ( 8, 6)
( 8, 0) to[L,l=$L_5$]        ( 8, 3)

(10, 0) to[L,l=$L_6$]        (10, 6)
( 8, 0) to[short]            (10, 0)
( 8, 6) to[short]            (10, 6)
;\end{schemat}

%\subsubsection{Rozwiązanie}
%TBD
\end{task}
\begin{task}

\begin{schemat} \draw

( 0, 0) to[R,l=$Z$]         ( 0, 4)
( 0, 4) to[Vsrc,l=$E_1$]    ( 0, 8)

( 4, 0) to[R,l=$2Z$]        ( 4, 2)
( 4, 4) to[Vsrc,l_=$E_2$]   ( 4, 2)

( 4, 4) to[R,l=$3Z$]        ( 4, 6)
( 4, 6) to[Vsrc,l=$E_3$]    ( 4, 8)

( 8, 0) to[R,l=$Z$]         ( 8, 4)
( 8, 4) to[Vsrc,l=$E_6$]    ( 8, 8)

( 0, 4) to[R,l=$2Z$]        ( 4, 4)

( 0, 8) to[R,l=$2Z$]        ( 4, 8)

( 0, 0) to[short]           ( 8, 0)
( 4, 8) to[short]           ( 8, 8)

;\end{schemat}

%\subsubsection{Rozwiązanie}
%TBD
\end{task}
























\chapter{Moc i dopasowanie na moc}

\section{Zadania}
\begin{task}
Obliczyć moc czynną pobieraną za źródła ${j=J_m \cdot cos(\omega \cdot t)}$  jeżeli amplituda prądu $i$ wynosi $1mA$, ${\omega \cdot L = \frac{1}{2 \cdot \omega \cdot C} = 1k\Omega}$ a impedancja obciążająca źródło ${Z_{AB} = (6-2j)k\Omega}$.

\begin{schemat} \draw

( 0, 0) to[Isrc,l=$j$]       ( 0, 4)
( 2, 0) to[C,l=$C$]          ( 2, 4)

( 4, 3) to[L,l=$L$]          ( 6, 3)
( 4, 5) to[C,l=$C$]          ( 6, 5)
( 4, 3) to[short]            ( 4, 5)
( 6, 3) to[short]            ( 6, 5)

( 8, 0) to[C,l=$C$]          ( 8, 4)
(10, 0) to[R,l=$R$]          (10, 4)

(10, 4) to[L,l=$L$]          (13, 4)
(13, 0) to[R,l=$R$]          (13, 4)


( 0, 0) to[short]            (13, 0)
( 0, 4) to[short]            ( 4, 4)
( 6, 4) to[short]            (10, 4)


;\end{schemat}

%\subsubsection{Rozwiązanie}
%TBD
\end{task}
\begin{task}

Dopasowanie na moc czynną. $E_1=10V$, $\omega = 5 \cdot 10^5$, $R_1 = 1k\Omega$, $C_1 = 1nF$, $L_1 = 1mH$, $R_2 = ???$, $C_2 = ???$

\begin{schemat} \draw

( 0, 0) to[Vsrc,l=$E_1$]   ( 0, 2)
( 0, 2) to[R,l=$R_1$]      ( 0, 4)
( 0, 4) to[C,l=$C_1$]      ( 0, 6)

( 7, 1) to[C,l=$C_2$]      ( 7, 3)
( 5, 1) to[L,l=$L_2$]      ( 5, 3)
( 6, 3) to[R,l=$R_2$]      ( 6, 6)
( 5, 1) to[short]          ( 7, 1)
( 5, 3) to[short]          ( 7, 3)
( 6, 0) to[short]          ( 6, 1)

( 0, 0) to[short]          ( 6, 0)
( 0, 6) to[short]          ( 6, 6)

;\end{schemat}

%\subsubsection{Rozwiązanie}
%TBD
\end{task}



























%====================================================
% Ostatna stona z ISBN'em
%====================================================
\label{page:lastpage}
\input{LastPage.tex}
%====================================================
\end{document}

%Tom pierwszy zadania
%Tom drugi rozwiązania

%TODO
