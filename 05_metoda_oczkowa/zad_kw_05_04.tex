\begin{task}
Oblicz rozpływ prądów metodą prądów oczkowych

\begin{align}
e_1(t) = 2 sin(2t+\frac{\pi}{2}) \\
e_2(t) = sin(2t)\\
e_3(t) =\sqrt{8} sin(2t) \\
e_4(t) = 2 sin(2t+\frac{\pi}{4}) \\
L = 1 \henr \\
C = 1 \farad \\
R = 1 \ohm \\
\end{align}

\begin{schemat} \draw
(0,0)  to [C,l=$C$] (0,2)
       to [Vsrc,l=$e_1(t)$] (0,4)
       to [R,l=$R_3$] (4,4)
       to [Vsrc,l=$e_3(t)$] (6,4)
       to [L, l=$L$] (8,4)
       to [C,l=$C$] (8,2)
       to [Vsrc,l=$e_4(t)$] (8,0)
       to [R,l=$R$] (4,0)
       to [L,l=$L$] (0,0)
(4,0)  to [R,l=$R$] (4,2)
       to [Vsrc,l=$e_2(t)$] (4,4)
;\end{schemat}

%\subsubsection{Rozwiązanie}
%TBD
\end{task}