\begin{task}
Oblicz rozpływ prądów metodą praw Kirchhoffa

\begin{center}\begin{circuitikz}[european] 
\ctikzset{voltage/distance from line=0.2} 
\ctikzset{voltage/european label distance=2.00}  
%przesynąć etykiety węzłów dalej od węzłów
\draw
 (0,0) to[R,l=$R_1$] (3,3)
       to[R,l_=$R_2$] (6,0)
       to[R,l_=$R_3$] (3,-3)
       to[R,l_=$R_4$] (0,0)
 (0,0) to[short,*-] (1,0)
       to[R,l_=$R_5$] (3,0) 
       to[american current source,l_=$E_5$] (5,0)
       to[short,-*] (6,0)
 (3,-3) to[short,*-] (8,-3)
        to[american current source,l_=$E_6$] (8,0)
        to[R,l_=$R_6$] (8,3) 
        to[short,-*] (3,3)
;\end{circuitikz}\end{center}

Do obliczeń przyjmij: $E_5=6V$, $E_6=4V$, $R_1=3\Omega$, $R_2=2\Omega$, $R_3=4\Omega$, $R_4=5\Omega$, $R_5=1\Omega$, $R_6=2\Omega$.

\subsubsection{Rozwiązanie}
TBD
\end{task}