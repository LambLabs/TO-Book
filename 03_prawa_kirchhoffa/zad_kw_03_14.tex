\begin{task}
Wyznacz wartość napięcia $u_c(t)$ na kondensatorze w obwodzie na rysunku poniżej.

\begin{align*}
e_1(t)&=sin\left(2 \cdot 10^9 \cdot t + \frac{3\pi}{4} \right) \\
e_2(t)&=sin\left(2 \cdot 10^9 \cdot t + \frac{3\pi}{2} \right) \\
e_2(t)&=sin\left(2 \cdot 10^9 \cdot t + \pi \right) \\
L_1&=1 n \henr \\
L_2&=0,5 n \henr \\
R_1&=R_2=R_3=1 \ohm \\
C_1&=10,5 n \farad 
\end{align*}

\begin{schemat}
\label{schemat:03:14:kw:Z}
\draw
 (0,0) to[Vsrc,l=$e_1(t)$] (0,2)
       to[L,l=$L_1$] (2,2)
       to[L,l=$L_2$] (6,2)
 (0,0) to[C,l=$C_1$] (2,0)
       to[Vsrc,l_=$e_2(t)$] (4,0)
       to[R,l_=$R_2$] (6,0)
 (2,2) to[R,l=$R_1$] (2,0)
 (0,0) to[short] (0,-2)
       to[R,l=$R_3$] (6,-2)
       to[short] (6,0)
       to[Vsrc,l=$e_3(t)$] (6,2)
;
\end{schemat}


\subsubsection{Rozwiązanie}

TBD
\end{task}