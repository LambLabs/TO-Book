\begin{task}
Wzynacy wartość siłz elektromotorycznej $E$, która na oporze $R_7$ powoduje spadek napięcia $U$

\begin{center}\begin{circuitikz}[european] 
\ctikzset{voltage/distance from line=0.2} 
\ctikzset{voltage/european label distance=2.00}  
%przesynąć etykiety węzłów dalej od węzłów
\draw
 (0,0) to[R,l_=$R_1$] (2,0)
       to[R,l_=$R_2$] (2,-2)
       to[short] (0,-2)
       to[american current source,l=$E$] (0,0)
 (2,0) to[R,l_=$R_3$,*-] (4,0)
       to[R,l_=$R_4$] (4,-2)
       to[R,l_=$R_5$,-*] (2,-2)
 (4,0) to[R,l_=$R_6$,*-] (6,0)
       to[R,l_=$R_7$] (6,-2)
       to[short,-*] (4,-2)       
;\end{circuitikz}\end{center}

Do obliczeń przyjmij że $U=2V$, $R_1=5 \Omega$, $R_2=6 \Omega$, $R_3=7 \Omega$, $R_4=8 \Omega$, $R_5=9 \Omega$, $R_6=10 \Omega$, $R_7=5 \Omega$, 

\subsubsection{Rozwiązanie}
TBD
\end{task}