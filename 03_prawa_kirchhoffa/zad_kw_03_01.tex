\begin{task}
W układzie dzielnika prądowego pokazanego na rysunku dobrać tak wartość oporu $R_2$, aby przez opór $R_1$ płynął prąd o natężeniu $p\cdot I$

\begin{center}\begin{circuitikz}[european] 
\ctikzset{voltage/distance from line=0.2} 
\ctikzset{voltage/european label distance=2.00}  
%przesynąć etykiety węzłów dalej od węzłów
\draw
 (0,0) to[short,i>_=$I$,-*] (1,0) 
       to[short] (1,-0.5) 
       to[R,l_=$R_2$] (3,-0.5)
       to[short,-*] (3,0)
       to[short,-*] (4,0)       
 (1,0) to[short] (1,0.5) 
       to[R,l_=$R_1$] (3,0.5)
       to[short] (3,0)       
;
\end{circuitikz}
\end{center}
\subsubsection{Rozwiązanie}
TBD
\end{task}