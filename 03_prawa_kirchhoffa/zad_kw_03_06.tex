\begin{task}
W układzie przedstawionym poniżej określ rozpływ prądów i rozkład napięć.

\begin{center}\begin{circuitikz}[european] 
\ctikzset{voltage/distance from line=0.2} 
\ctikzset{voltage/european label distance=2.00}  
%przesynąć etykiety węzłów dalej od węzłów
\draw
 (0,0) to[R,l=$R_2$] (2,0)
       to[american current source,l=$E$] (4,0)
       to[R,l_=$R_3$] (4,-3)
       to[short] (0,-3)
       to[R,l_=$R_1$] (0,0) 
 (0,0) to[short] (0,1)
       to[R,l=$R_4$] (4,1)
       to[short] (4,0)    
 (0,-3) to[short] (-1,-3)
        to[american current source,l=$I_1$] (-1,0)
        to[short] (0,0)
 (4,-3) to[short] (5,-3)
        to[american current source,l_=$I_2$] (5,0)
        to[short] (4,0)        
;\end{circuitikz}\end{center}

Do obliczeń przyjmij $R_1=1\Omega$, $R_2=2\Omega$, $R_3=3\Omega$, $R_4=4\Omega$, $E=10V$, $I_1=2A$, $I_2=5A$.

\subsubsection{Rozwiązanie}

Zastosować przekształcenie źródeł

TBD
\end{task}