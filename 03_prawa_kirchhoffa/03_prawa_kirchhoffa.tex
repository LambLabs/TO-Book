\chapter{Prawa Kirchhoffa}
\section{Teoria}
\subsection{I prawo Kirchhoffa - prądowe prawo Kirchhoffa}

\begin{center}
\begin{circuitikz}[european] 
\ctikzset{voltage/distance from line=0.2} % można jeszcze poprawić wygląd
\ctikzset{voltage/european label distance=2.00}  
%\ctikzset{voltage/distance from node=.1}
\draw
 (0,0) to[short,i>_=$I_1$,-*] (2,0) 
       to[short,i>_=$I_3$,*-] (2.5,1)
 (2,0) to[short,i<_=$I_5$,*-] (2.5,-1)       
 (2,0) to[short,i>_=$I_4$,*-] (4,0)
 (2,0) to[short,i>_=$I_2$,*-] (1,1) 
;
\end{circuitikz}
\end{center}

Suma prądów wpływających i wypływających z węzła jest równa 0

\begin{equation}
\sum_{i}I_i=0
\end{equation}

Prądy wpływające do węzła ($I_1$, $I_5$) oznaczamy ze znakiem plus, prądy wypływające z węzła ($I_2$, $I_3$, $I_4$) oznaczamy ze znakiem minus.

\begin{equation}
I_1-I_2-I_3-I_4+I_5=0
\end{equation}

\subsection{II prawo Kirchhoffa - napięciowe prawo Kirchhoffa}

\begin{center}
\begin{circuitikz}[european] 
\ctikzset{voltage/distance from line=0.2} % można jeszcze poprawić wygląd
\ctikzset{voltage/european label distance=2.00}  
%\ctikzset{voltage/distance from node=.1}
\draw
 (-2,0) to[R,l_=$R_2$,i>_=$I_2$,v^=$U_2$] (2,0) 
       to[R,l_=$R_3$,v^=$U_3$]  (2,-2)  
       to[american current source,l_=$E_3$] (2,-3)  
       to[short,i>_=$I_3$] (2,-4)
       to[american current source,l_=$E_4$] (0,-4)  
       to[R,l_=$R_4$,v^>=$U_4$]  (-2,-4)
       to[R,l_=$R_1$,i<_=$I_1$,v^>=$U_1$]  (-2,0)  ;
\draw[->] (-1,-2) arc (180:-135:1cm) 
;
\end{circuitikz}
\end{center}

W każdym obwodzie zamkniętym suma spadków napięć na poszczególnych elementach obwodu jest w każdej chwili równa zero

\begin{equation}
\sum_{i}U_i=0
\end{equation}

Należy założyć sobie kierunek obiegu obwodu. Napięcia zgodne z kierunkiem obiegu obwodu mają znak dodatki, napięcia przeciwne z kierunkiem obiegu obwodu mają znak ujemny. 

\begin{equation}
U_1-U_2-U_3+E_3+E_4+U_4=0
\end{equation}

\section{Zadania}
\begin{task}
Wyznacz spadek napięcia na rezystorze $R_1$
\begin{center}\begin{circuitikz}[european] 
\ctikzset{voltage/distance from line=0.2} 
\ctikzset{voltage/european label distance=2.00}  
\draw
 (0,0) to[R,l=$R_1$] (2,0) 
       to[R,l=$R_2$] (2,-2)
       to[short] (0,-2)
       to[american current source,l=$E$] (0,0) 
;
\end{circuitikz}
\end{center}
\subsubsection{Rozwiązanie}
TBD
\end{task}
\begin{task}
W układzie dzielnika prądowego pokazanego na rysunku dobrać tak wartość oporu $R_2$, aby przez opór $R_1$ płynął prąd o natężeniu $p\cdot I$

\begin{center}\begin{circuitikz}[european] 
\ctikzset{voltage/distance from line=0.2} 
\ctikzset{voltage/european label distance=2.00}  
%przesynąć etykiety węzłów dalej od węzłów
\draw
 (0,0) to[short,i>_=$I$,-*] (1,0) 
       to[short] (1,-0.5) 
       to[R,l_=$R_2$] (3,-0.5)
       to[short,-*] (3,0)
       to[short,-*] (4,0)       
 (1,0) to[short] (1,0.5) 
       to[R,l_=$R_1$] (3,0.5)
       to[short] (3,0)       
;
\end{circuitikz}
\end{center}
\subsubsection{Rozwiązanie}
TBD
\end{task}
\begin{task}
Wzynacy wartość siłz elektromotorycznej $E$, która na oporze $R_7$ powoduje spadek napięcia $U$

\begin{center}\begin{circuitikz}[european] 
\ctikzset{voltage/distance from line=0.2} 
\ctikzset{voltage/european label distance=2.00}  
%przesynąć etykiety węzłów dalej od węzłów
\draw
 (0,0) to[R,l_=$R_1$] (2,0)
       to[R,l_=$R_2$] (2,-2)
       to[short] (0,-2)
       to[american current source,l=$E$] (0,0)
 (2,0) to[R,l_=$R_3$,*-] (4,0)
       to[R,l_=$R_4$] (4,-2)
       to[R,l_=$R_5$,-*] (2,-2)
 (4,0) to[R,l_=$R_6$,*-] (6,0)
       to[R,l_=$R_7$] (6,-2)
       to[short,-*] (4,-2)       
;\end{circuitikz}\end{center}

Do obliczeń przyjmij że $U=2V$, $R_1=5 \Omega$, $R_2=6 \Omega$, $R_3=7 \Omega$, $R_4=8 \Omega$, $R_5=9 \Omega$, $R_6=10 \Omega$, $R_7=5 \Omega$, 

\subsubsection{Rozwiązanie}
TBD
\end{task}
\begin{task}
Oblicz rozpływ prądów w układzie podanym na rysunku metodą praw Kirchhoffa. Wyznacz napięcia na rezystorach.

\begin{center}\begin{circuitikz}[european] 
\ctikzset{voltage/distance from line=0.2} 
\ctikzset{voltage/european label distance=2.00}  
%przesynąć etykiety węzłów dalej od węzłów
\draw
 (0,0) to[R,l=$R_1$] (2,0)
       to[R,l_=$R_3$] (2,-2)
       to[short] (0,-2)
       to[american current source,l=$E_1$] (0,0)
 (2,-2) to[short,*-] (4,-2)
       to[american current source,l=$E_2$] (4,0)
       to[R,l_=$R_2$,-*] (2,0)
;\end{circuitikz}\end{center}

Do obliczeń przyjmij że: $E_1=4V$, $E_2=6V$, $R_1=2\Omega$, $R_2=12\Omega$, $R_3=4\Omega$.

\subsubsection{Rozwiązanie}
TBD
\end{task}
\begin{task}
Oblicz rozpływ prądów metodą praw Kirchhoffa

\begin{center}\begin{circuitikz}[european] 
\ctikzset{voltage/distance from line=0.2} 
\ctikzset{voltage/european label distance=2.00}  
%przesynąć etykiety węzłów dalej od węzłów
\draw
 (0,0) to[R,l=$R_1$] (3,3)
       to[R,l_=$R_2$] (6,0)
       to[R,l_=$R_3$] (3,-3)
       to[R,l_=$R_4$] (0,0)
 (0,0) to[short,*-] (1,0)
       to[R,l_=$R_5$] (3,0) 
       to[american current source,l_=$E_5$] (5,0)
       to[short,-*] (6,0)
 (3,-3) to[short,*-] (8,-3)
        to[american current source,l_=$E_6$] (8,0)
        to[R,l_=$R_6$] (8,3) 
        to[short,-*] (3,3)
;\end{circuitikz}\end{center}

Do obliczeń przyjmij: $E_5=6V$, $E_6=4V$, $R_1=3\Omega$, $R_2=2\Omega$, $R_3=4\Omega$, $R_4=5\Omega$, $R_5=1\Omega$, $R_6=2\Omega$.

\subsubsection{Rozwiązanie}
TBD
\end{task}
\begin{task}
W układzie przedstawionym poniżej dobierz opór $R$ w taki sposób aby prąd $I$ był równy zero.

\begin{center}\begin{circuitikz}[european] 
\ctikzset{voltage/distance from line=0.2} 
\ctikzset{voltage/european label distance=2.00}  
%przesynąć etykiety węzłów dalej od węzłów
\draw
 (0,0) to[R,l=$40\Omega$] (2,0)
       to[R,l=$30\Omega$] (2,-3)
       to[short] (0,-3)
       to[american current source,l=$100V$] (0,0)
 (2,0) to[short,*-*] (3,0)
        to[short] (3,0.5)
        to[R,l=$40\Omega$] (5,0.5) 
        to[short] (5,0)
        to[short,*-*] (6,0)
 (3,0)  to[short] (3,-0.5)
        to[R,l=$40\Omega$] (5,-0.5) 
        to[short] (5,0)        
 (2,-3) to[short,*-*] (6,-3)          
 (6,-3) to[short] (8,-3)
        to[short] (8,-2)
        to[american current source,l_=$35V$] (8,-1) 
        to[short,i>_=$I$] (8,0)
        to[R,l_=$40\Omega$] (6,0)
        to[R,l=$R$] (6,-3)
       
;\end{circuitikz}\end{center}

\subsubsection{Rozwiązanie}
TBD
\end{task}
\begin{task}
W układzie przedstawionym poniżej określ rozpływ prądów i rozkład napięć.

\begin{center}\begin{circuitikz}[european] 
\ctikzset{voltage/distance from line=0.2} 
\ctikzset{voltage/european label distance=2.00}  
%przesynąć etykiety węzłów dalej od węzłów
\draw
 (0,0) to[R,l=$R_2$] (2,0)
       to[american current source,l=$E$] (4,0)
       to[R,l_=$R_3$] (4,-3)
       to[short] (0,-3)
       to[R,l_=$R_1$] (0,0) 
 (0,0) to[short] (0,1)
       to[R,l=$R_4$] (4,1)
       to[short] (4,0)    
 (0,-3) to[short] (-1,-3)
        to[american current source,l=$I_1$] (-1,0)
        to[short] (0,0)
 (4,-3) to[short] (5,-3)
        to[american current source,l_=$I_2$] (5,0)
        to[short] (4,0)        
;\end{circuitikz}\end{center}

Do obliczeń przyjmij $R_1=1\Omega$, $R_2=2\Omega$, $R_3=3\Omega$, $R_4=4\Omega$, $E=10V$, $I_1=2A$, $I_2=5A$.

\subsubsection{Rozwiązanie}

Zastosować przekształcenie źródeł

TBD
\end{task}
\begin{task}
W układzie przedstawionym poniżej oblicz wartość prądu $\underline{I_1}$

\begin{schemat}
\label{schemat:03:07:kw:Z}
\draw
 (0,0) to[Z,l=$Z$] (0,2)
       to[Vsrc,l=$\underline{E}$] (0,4)
       to[Z,l_=$Z$] (2,4)
       to[Z,l_=$Z$] (4,4)
       to[short,i=$\underline{I_1}$] (6,4)
       to[Z,l_=$2 \cdot Z$] (6,0) 
       to[short] (2,0)
       to[Z,l_=$Z$] (0,0)
 (2,0) to[Z,l=$2\cdot Z$,*-*] (2,4)
 (4,0) to[Z,l=$Z$,*-] (4,2)
       to[Z,l=$Z$,-*] (4,4)
;
\end{schemat}

Do obliczeń przyjmij $\underline{E}=3\cdot e^{\jmath \cdot \frac{\pi}{4}} \volt$, $Z=2+\jmath\ohm$.

\subsubsection{Rozwiązanie}

TBD
\end{task}
\begin{task}
W układzie przedstawionym poniżej dobierz wartość zespolonej amplitudy napięcia $\underline{E}$ aby wartość zespolonej amplitudy natężenia prądu $\underline{I_x}$ była równa $1+\jmath\amper$.

\begin{schemat}
\label{schemat:03:08:kw:Z}
\draw
 (0,0) to[Z,l=$Z$] (0,2)
       to[Vsrc,l=$\underline{E}$] (0,4)
       to[Z,l_=$Z$] (2,4)
       to[Z,l_=$Z$] (4,4)
       to[short,i=$\underline{I_1}$] (6,4)
       to[Z,l_=$Z$] (6,2) 
       to[Z,l_=$Z$] (6,0)        
       to[short] (2,0)
       to[Z,l_=$Z$] (0,0)
 (2,0) to[Z,l=$3\cdot Z$,*-*] (2,4)
 (4,0) to[Z,l=$2\cdot Z$,*-*] (4,4)
;
\end{schemat}

Do obliczeń przyjmij $Z=1+2\cdot \jmath\ohm$.

\subsubsection{Rozwiązanie}

TBD
\end{task}
\begin{task}
W układzie przedstawionym poniżej wyznacz wartość zespolonej amplitudy natężenia prądu $\underline{I_1}$.

\begin{schemat}
\label{schemat:03:09:kw:Z}
\draw
 (0,0) to[Z,l=$Z_1$] (0,2)
       to[Vsrc,l=$e_1(t)$] (0,4)
       to[short,i=$\underline{I_1}$] (2,4)
       to[short] (4,4)
       to[Vsrc,l_=$e_2(t)$] (6,4)
       to[Z,l_=$Z_4$] (6,0) 
       to[Z,l_=$Z_3$] (4,0)        
       to[short] (0,0)
 (2,0) to[Isrc,l=$j(t)$,*-*] (2,4)
 (4,0) to[Z,l=$Z_2$,*-*] (4,4)
;
\end{schemat}

\begin{align*}
e_1(t)&=2 \cdot sin(2 \cdot t)\volt\\
e_2(t)&= sin(2 \cdot t)\volt\\
j(t)&= cos(2 \cdot t)\amper\\
Z_1&=\jmath\ohm\\
Z_2&=2\jmath+1\ohm\\
Z_3&=\jmath\ohm\\
Z_4&=1\ohm\\
\end{align*}

\subsubsection{Rozwiązanie}

TBD
\end{task}
\begin{task}
W układzie przedstawionym poniżej dobierz wartość zespolonej amplitudy natężenia prądu źródła $\underline{J_2}$ w taki sposób aby wartość zespolonej amplitudy natężenia prądu $\underline{I_1}$ była równa $0$.

\begin{schemat}
\label{schemat:03:10:kw:Z}
\draw
 (0,0) to[Z,l=$Z_1$] (0,2)
       to[Vsrc,l=$e_1(t)$] (0,4)
       to[short] (2,4)
       to[short,i=$\underline{I_1}$] (4,4)
       to[short] (6,4)
 (6,0) to[Isrc,l=$j_2(t)$] (6,4) 
 (6,0) to[short] (4,0)        
       to[Z,l_=$Z_2$] (2,0)
       to[short] (0,0)
 (2,0) to[Isrc,l=$j(t)$,*-*] (2,4)
 (4,0) to[Z,l=$Z_3$,*-*] (4,4)
;
\end{schemat}

\begin{align*}
e_1(t)&=2 \cdot sin(2 \cdot t)\volt\\
j(t)&= cos(2 \cdot t)\amper\\
Z_1&=\jmath\ohm\\
Z_2&=2\jmath+1\ohm\\
Z_3&=\jmath\ohm\\
\end{align*}

\subsubsection{Rozwiązanie}

TBD
\end{task}
\begin{task}
W układzie przedstawionym poniżej dobierz wartość impedancji $Z_1$ w taki sposób aby ...

\begin{align*}
e(t)&=2 \cdot sin(2 \cdot t)\volt\\
j(t)&=2 \cdot sin(2 \cdot t + \frac{\pi}{2})\volt\\
R_1&=2\ohm\\
R_2&=1\ohm\\
L_1&=3\henr\\
C_1&=0.5\farad
\end{align*}


\begin{schemat}
\label{schemat:03:11:kw:Z}
\draw
 (0,0) to[Vsrc,l=$e_1(t)$] (0,3)
       to[short] (0,6)
       to[Vsrc,l=$e_2(t)$] (2,6)
       to[R,l=$R_1$] (4,6)       
       to[short] (4,3)
       to[short] (5,3)
       to[L,l=$L_1$] (5,0)
       to[short] (4,0)
       to[C,l=$C_1$] (0,0)
 (4,0) to[Isrc,l=$j(t)$] (4,3)
 (4,0) to[R,l=$R_1$] (0,3)
       to[Z,l_=$Z_1$] (4,3)
 (4,4.5) to[Isrc,l=$j(t)$] (0,4.5) 
;
\end{schemat}


\subsubsection{Rozwiązanie}

TBD
\end{task}
\begin{task}
Oblicz przesunięcie fazowe pomiędzy natężeniem prądu $\underline{I_1}$ a napięciem $\underline{U_1}$ w poniższym układzie.  

\begin{schemat}
\label{schemat:03:12:kw:Z}
\draw
node[ocirc,label=A] (A) at (2, 4) {}
node[ocirc,label=below:B] (B) at (2,0) {}

 (0,0) to[R,l=$R_1$] (0,2) 
       to[Vsrc,l=$e_1(t)$] (0,4)
       to[short,i>=$\underline{I_1}$] (2,4)
       to[L,l=$L_1$] (4,4)       
       to[short] (5,4)
       to[C,l=$C$] (5,0)
       to[short] (2,0)
       to[L,l=$L_2$] (0,0)
 (4,0) to[R,l=$R_2$] (4,4)
 (B)   to[open,v^>=$\underline{U_1}$] (A)

;
\end{schemat}


\subsubsection{Rozwiązanie}

TBD
\end{task}
\begin{task}
Wyznacz wartość natężenia prądu $i_1(t)$ oraz $i_1(t)$ w obwodzie na rysunku poniżej.

\begin{align*}
L&=1 n \henr \\
C&=250 p \farad \\
e_1(t)&=100 \cdot sin\left(10^9 \cdot t + \frac{\pi}{4} \right) \\
e_2(t)&=100 \cdot sin\left(10^9 \cdot t + \frac{\pi}{2} \right)
\end{align*}

\begin{schemat}
\label{schemat:03:13:kw:Z}
\draw
 (0,0) to[Vsrc,l=$e_1(t)$] (0,3)
       to[L,l=$L$,i>^=$\underline{I_1}$] (3,3)
       to[L,l=$L$] (6,3)
       to[C,l=$C$] (6,0)       
       to[Vsrc,l=$e_2(t)$] (3,0)
       to[C,l=$C$] (0,0)
 (3,0) to[L,l=$L$,i<=$\underline{I_2}$] (3,3)
;
\end{schemat}


\subsubsection{Rozwiązanie}

TBD
\end{task}
\begin{task}
Wyznacz wartość napięcia $u_c(t)$ na kondensatorze w obwodzie na rysunku poniżej.

\begin{align*}
e_1(t)&=sin\left(2 \cdot 10^9 \cdot t + \frac{3\pi}{4} \right) \\
e_2(t)&=sin\left(2 \cdot 10^9 \cdot t + \frac{3\pi}{2} \right) \\
e_2(t)&=sin\left(2 \cdot 10^9 \cdot t + \pi \right) \\
L_1&=1 n \henr \\
L_2&=0,5 n \henr \\
R_1&=R_2=R_3=1 \ohm \\
C_1&=10,5 n \farad 
\end{align*}

\begin{schemat}
\label{schemat:03:14:kw:Z}
\draw
 (0,0) to[Vsrc,l=$e_1(t)$] (0,2)
       to[L,l=$L_1$] (2,2)
       to[L,l=$L_2$] (6,2)
 (0,0) to[C,l=$C_1$] (2,0)
       to[Vsrc,l_=$e_2(t)$] (4,0)
       to[R,l_=$R_2$] (6,0)
 (2,2) to[R,l=$R_1$] (2,0)
 (0,0) to[short] (0,-2)
       to[R,l=$R_3$] (6,-2)
       to[short] (6,0)
       to[Vsrc,l=$e_3(t)$] (6,2)
;
\end{schemat}


\subsubsection{Rozwiązanie}

TBD
\end{task}
\begin{task}
Wyznacz wartość napięcia $u_{c1}(t)$ na kondensatorze w obwodzie na rysunku poniżej.

\begin{align*}
e_1(t)&=sin\left(2 \cdot 10^9 \cdot t + \frac{3\pi}{4} \right) \\
e_2(t)&=sin\left(2 \cdot 10^9 \cdot t + \frac{3\pi}{2} \right) \\
e_2(t)&=sin\left(2 \cdot 10^9 \cdot t + \pi \right) \\
L_1&=1 n \henr \\
L_2&=0,5 n \henr \\
R_1&=R_2=R_3=1 \ohm \\
C_1&=10,5 n \farad \\
C_2&=1 n \farad 
\end{align*}

\begin{schemat}
\label{schemat:03:15:kw:Z}
\draw
 (0,0) to[Vsrc,l=$e_1(t)$] (0,2)
       to[C,l=$C_1$] (2,2)
       to[R,l=$R_2$] (6,2)
 (0,0) to[C,l=$C_2$] (2,0)
       to[Vsrc,l_=$e_2(t)$] (4,0)
       to[L,l_=$L_2$] (6,0)
 (2,2) to[R,l=$R_1$] (2,0)
 (0,0) to[short] (0,-2)
       to[Vsrc,l_=$e_3(t)$] (3,-2)
       to[R,l=$R_3$] (6,-2)
       to[short] (6,0)
       to[L,l=$L_1$] (6,2)
;
\end{schemat}


\subsubsection{Rozwiązanie}

TBD
\end{task}




















