\subsection{Źródła idealne}

\subsubsection{Źródło napięciowe}
\begin{schemat}
\draw
 ( 0, 0) to[polish voltage source, l=$U$]       ( 2, 0) 
;
\end{schemat}

\subsubsection{Źródło prądowe}
\begin{schemat}
\draw
 ( 0, 0) to[polish current source, l=$I$]       ( 2, 0) 
;
\end{schemat}

\subsection{Źródła rzeczywiste}

\subsubsection{Źródło napięciowe}
\begin{schemat}
\draw
 ( 0, 0) to[polish voltage source, l=$e$]       ( 2, 0) 
 ( 2, 0) to[R, l=$R_w$]                         ( 4, 0) 
;
\end{schemat}

\subsubsection{Źródło prądowe}
\begin{schemat}
\draw
 ( 0, 0) to[polish current source, l=$j$]       ( 2, 0) 
 ( 0, 2) to[R, l=$R_w$]                         ( 2, 2) 
 ( 0, 0) to[short]( 0, 2)
 ( 2, 0) to[short]( 2, 2)
 ( -1, 1) to[short]( 0, 1)
 ( 2, 1) to[short]( 3, 1)
;
\end{schemat}

\subsection{Źródła sterowane}

\subsubsection{Źródło napięciowe sterowane napięciem}
\begin{schemat}
\draw
 node[ocirc] (A) at ( 0, 0) {}
 node[ocirc] (B) at ( 0, 2) {}
 ( 0, 0) to[open, v=$U_1$]                                ( 0, 2)
 ( 4, 0) to[polish controlled voltage source, l_=${e=k \cdot U_1}$]      ( 4, 2) 
;
\end{schemat}

\subsubsection{Źródło napięciowe sterowane prądem}

\begin{schemat}
\draw
 node[ocirc] (A) at ( 0, 0) {}
 node[ocirc] (B) at ( 0, 2) {}
 ( 0, 0) to[short, i=$I_1$]                                ( 0, 2)
 ( 4, 0) to[polish controlled voltage source, l_=${e=r \cdot I_1}$]      ( 4, 2) 
;
\end{schemat}

\subsubsection{Źródło prądowe sterowane napięciem}

\begin{schemat}
\draw
 node[ocirc] (A) at ( 0, 0) {}
 node[ocirc] (B) at ( 0, 2) {}
 ( 0, 0) to[open, v=$U_1$]                              ( 0, 2)
 ( 4, 0) to[polish controlled current source, l_=${j=g \cdot U_1}$]      ( 4, 2) 
;
\end{schemat}

\subsubsection{Źródło prądowe sterowane prądem}

\begin{schemat}
\draw
 node[ocirc] (A) at ( 0, 0) {}
 node[ocirc] (B) at ( 0, 2) {}
 ( 0, 0) to[short, i=$I_1$]                                ( 0, 2)
 ( 4, 0) to[polish controlled current source, l_=${j= \alpha \cdot I_1}$]      ( 4, 2) 
;
\end{schemat}

\subsection{Zamiana źródeł}

\begin{schemat}
\draw
 ( 0, 0) to[polish voltage source, l=$e$]       ( 2, 0) 
 ( 2, 0) to[R, l=$R_{w1}$]                         ( 4, 0) 
;
\end{schemat}

\begin{schemat}
\draw
 ( 0, 0) to[polish current source, l=$j$]       ( 2, 0) 
 ( 0, 2) to[R, l=$R_{w2}$]                         ( 2, 2) 
 ( 0, 0) to[short]( 0, 2)
 ( 2, 0) to[short]( 2, 2)
 ( -1, 1) to[short]( 0, 1)
 ( 2, 1) to[short]( 3, 1)
;
\end{schemat}

\begin{equation}
R_{w1} = R_{w2} = R_w
\end{equation}
\begin{equation}
e = R_w \cdot j
\end{equation}
\begin{equation}
j = \frac{e}{R_w}
\end{equation}

\subsection{Łączenie źródeł}

\subsection{Szeregowe łączenie źródeł napięciowych}

\begin{schemat}
\draw
 ( 0,  0) to[polish voltage source, l=$e_1$]     ( 0, -2) 
 ( 0, -2) to[R, l=$R_1$]                         ( 0, -4)
 ( 0, -4) to[polish voltage source, l=$e_2$]     ( 0, -6) 
 ( 0, -6) to[R, l=$R_2$]                         ( 0, -8)  
 ( 0, 0) to[short]( 4, 0)
 ( 0,-8) to[short]( 4,-8)
 node[ocirc] (A) at ( 4, 0) {}
 node[ocirc] (B) at ( 4,-8) {} 
;
\end{schemat}

\begin{schemat}
\draw
 ( 0,  0) to[polish voltage source, l=$e_1$]     ( 0, -2) 
 ( 0, -4) to[R, l=$R_1$]                         ( 0, -6)
 ( 0, -2) to[polish voltage source, l=$e_2$]     ( 0, -4) 
 ( 0, -6) to[R, l=$R_2$]                         ( 0, -8)  
 ( 0, 0) to[short]( 4, 0)
 ( 0,-8) to[short]( 4,-8)
 node[ocirc] (A) at ( 4, 0) {}
 node[ocirc] (B) at ( 4,-8) {} 
;
\end{schemat}

\begin{equation}
e = e_1 + e_2
\end{equation}
\begin{equation}
R = R_1 + R_2
\end{equation}

\begin{schemat}
\draw
 ( 0,  0) to[polish voltage source, l=$e$]     ( 0, -2) 
 ( 0, -2) to[R, l=$R$]                         ( 0, -4)
 ( 0, 0) to[short]( 4, 0)
 ( 0,-4) to[short]( 4,-4)
 node[ocirc] (A) at ( 4, 0) {}
 node[ocirc] (B) at ( 4,-4) {} 
;
\end{schemat}

\subsection{Równoległe łączenie źródeł pradowych}

\begin{schemat}
\draw
 ( 0, 0) to[polish current source, l=$j_1$]     ( 0, 2) 
 ( 2, 0) to[R, l=$R_1$]                         ( 2, 2) 
 ( 4, 0) to[polish current source, l=$j_2$]     ( 4, 2) 
 ( 6, 0) to[R, l=$R_2$]                         ( 6, 2)  
 ( 0, 0) to[short]( 8, 0)
 ( 0, 2) to[short]( 8, 2)
 node[ocirc] (A) at ( 8, 0) {}
 node[ocirc] (B) at ( 8, 2) {} 
;
\end{schemat}

\begin{schemat}
\draw
 ( 0, 0) to[polish current source, l=$j_1$]     ( 0, 2) 
 ( 4, 0) to[R, l=$R_1$]                         ( 4, 2) 
 ( 2, 0) to[polish current source, l=$j_2$]     ( 2, 2) 
 ( 6, 0) to[R, l=$R_2$]                         ( 6, 2)  
 ( 0, 0) to[short]( 8, 0)
 ( 0, 2) to[short]( 8, 2)
 node[ocirc] (A) at ( 8, 0) {}
 node[ocirc] (B) at ( 8, 2) {} 
;
\end{schemat}

\begin{equation}
j = j_1 + j_2
\end{equation}
\begin{equation}
R = R_1 || R_2 =\frac{R_1 \cdot R_2}{R_1+R_2}
\end{equation}

\begin{schemat}
\draw
 ( 0, 0) to[polish current source, l=$j$]     ( 0, 2) 
 ( 2, 0) to[R, l=$R$]                         ( 2, 2) 
 ( 0, 0) to[short]( 8, 0)
 ( 0, 2) to[short]( 8, 2)
 node[ocirc] (A) at ( 8, 0) {}
 node[ocirc] (B) at ( 8, 2) {}
;
\end{schemat}

\subsection{Równoległe łączenie źródeł napięciowych}

\begin{schemat}
\draw
 ( 0,  0) to[polish voltage source, l=$e_1$]     ( 0, -2) 
 ( 0, -2) to[R, l=$R_1$]                         ( 0, -4)
 ( 2, -0) to[polish voltage source, l=$e_2$]     ( 2, -2) 
 ( 2, -2) to[R, l=$R_2$]                         ( 2, -4)  
 ( 0, 0) to[short]( 4, 0)
 ( 0,-4) to[short]( 4,-4)
 node[ocirc] (A) at ( 4, 0) {}
 node[ocirc] (B) at ( 4,-4) {} 
;
\end{schemat}

\begin{equation}
e = \frac{e_1 \cdot R_2 + e_2 \cdot R_1}{R_1+R_2}
\end{equation}
\begin{equation}
R = R_1 || R_2 =\frac{R_1 \cdot R_2}{R_1+R_2}
\end{equation}

\begin{schemat}
\draw
 ( 0,  0) to[polish voltage source, l=$e$]     ( 0, -2) 
 ( 0, -2) to[R, l=$R$]                         ( 0, -4)
 ( 0, 0) to[short]( 4, 0)
 ( 0,-4) to[short]( 4,-4)
 node[ocirc] (A) at ( 4, 0) {}
 node[ocirc] (B) at ( 4,-4) {} 
;
\end{schemat}



\subsection{Szeregowe łączenie źródeł pradowych}

\begin{schemat}
\draw
( 0, 0) to[short] ( 0,-1)
(-1,-1) to[short] ( 1,-1)
(-1,-1) to[Isrc, l=$j_1$] (-1,-3)
( 1,-1) to[R, l=$R_1$] ( 1,-3)
(-1,-3) to[short] ( 1,-3)
( 0,-3) to[short] ( 0,-5)
(-1,-5) to[short] ( 1,-5)
(-1,-5) to[Isrc, l=$j_2$] (-1,-7)
( 1,-5) to[R, l=$R_2$] ( 1,-7)
(-1,-7) to[short] ( 1,-7)
( 0,-7) to[short] ( 0,-8)
 ( 0, 0) to[short]( 4, 0)
 ( 0,-8) to[short]( 4,-8)
 node[ocirc] (A) at ( 4, 0) {}
 node[ocirc] (B) at ( 4,-8) {} 
;
\end{schemat}

\begin{schemat}
\draw
 ( 0,  0) to[polish voltage source, l=$e_1$]     ( 0, -2) 
 ( 0, -2) to[R, l=$R_1$]                         ( 0, -4)
 ( 0, -4) to[polish voltage source, l=$e_2$]     ( 0, -6) 
 ( 0, -6) to[R, l=$R_2$]                         ( 0, -8)  
 ( 0, 0) to[short]( 4, 0)
 ( 0,-8) to[short]( 4,-8)
 node[ocirc] (A) at ( 4, 0) {}
 node[ocirc] (B) at ( 4,-8) {} 
;
\end{schemat}

\begin{schemat}
\draw
 ( 0,  0) to[polish voltage source, l=$e_1$]     ( 0, -2) 
 ( 0, -4) to[R, l=$R_1$]                         ( 0, -6)
 ( 0, -2) to[polish voltage source, l=$e_2$]     ( 0, -4) 
 ( 0, -6) to[R, l=$R_2$]                         ( 0, -8)  
 ( 0, 0) to[short]( 4, 0)
 ( 0,-8) to[short]( 4,-8)
 node[ocirc] (A) at ( 4, 0) {}
 node[ocirc] (B) at ( 4,-8) {} 
;
\end{schemat}

\begin{equation}
R = R_1 + R_2
\end{equation}
\begin{equation}
e = e_1 + e_2 = j_1 \cdot R_1 + j_2 \cdot R_2
\end{equation}
\begin{equation}
j = \frac{e}{R} = \frac{j_1 \cdot R_1 + j_2 \cdot R_2}{R_1 + R_2}
\end{equation}


\begin{schemat}
\draw
 ( 0,  0) to[polish voltage source, l=$e$]     ( 0, -2) 
 ( 0, -2) to[R, l=$R$]                         ( 0, -4)
 ( 0, 0) to[short]( 4, 0)
 ( 0,-4) to[short]( 4,-4)
 node[ocirc] (A) at ( 4, 0) {}
 node[ocirc] (B) at ( 4,-4) {} 
;
\end{schemat}











